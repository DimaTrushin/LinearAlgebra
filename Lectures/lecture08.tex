\ProvidesFile{lecture08.tex}[Лекция 8]


\subsection{Мультипликативные отображения}

Давайте подытожим, что мы показали.
Утверждение~\ref{claim::DetPolyAnti} вместе с утверждением~\ref{claim::DetTranspose} объясняют почему определитель является полилинейной кососимметрической функцией как строк, так и столбцов.
Далее утверждение~\ref{claim::PolyAntiUnique} доказывает, что любая полилинейная кососимметричная функция по строкам, принимающая значение $1$ на единичной матрице, должна быть определителем.
С помощью утверждения~\ref{claim::DetTranspose} мы получаем аналогичный результат для столбцов.
Таким образом мы показали эквивалентность подхода (I) подходам (II) и (II').

Теперь, утверждение~\ref{claim::DetMulti} показывает, что определитель обязательно мультипликативен, а свойство (III)~(2) следует из явных вычислений для элементарных матриц.
Тем самым мы показали, что (I) и (II) влекут (III).
Осталось показать, что (III) влечет (I), т.е. что определитель является единственной функцией с такими свойствами.

\begin{claim}
\label{claim::DetMultiUnique}
Пусть $\psi\colon \Matrix{n}\to \mathbb R$ -- отображение, удовлетворяющее свойствам:
\begin{enumerate}
\item $\psi(AB) = \psi(A)\psi(B)$ для любых $A,B\in\Matrix{n}$.

\item 
$
\psi
\begin{pmatrix}
{1}&{}&{}&{}\\
{}&{\ddots}&{}&{}\\
{}&{}&{1}&{}\\
{}&{}&{}&{d}\\
\end{pmatrix}
= 
d
$ для любого ненулевого $d\in\mathbb R$.
\end{enumerate}
Тогда $\psi = \det$.
\end{claim}

Доказательство этого утверждения разобьем в несколько этапов.
В начале докажем элементарные свойства мультипликативных отображений.

\begin{claim}
Пусть $\psi\colon \Matrix{n}\to \mathbb R$ отображение со свойством $\psi(AB) = \psi(A)\psi(B)$ для всех $A,B\in\Matrix{n}$.
Тогда
\begin{enumerate}
\item Если $P\in \Matrix{n}$ такая, что $P^2 = P$, то $\psi(P)$ равно либо $0$, либо $1$.

\item В частности, значение $\psi(0)$ и $\psi(E)$ равно либо $0$, либо $1$.

\item Если $\psi(E) = 0$, то $\psi(A) = 0$ для любой матрицы $A\in\Matrix{n}$.

\item Если $\psi(0) = 1$, то $\psi(A) = 1$ для любой матрицы $A\in\Matrix{n}$.

\item Если $\psi(E) = 1$, то $\psi(A^{-1}) = \psi(A)^{-1}$ для любой обратимой матрицы $A\in\Matrix{n}$.
\end{enumerate}
\end{claim}
\begin{proof}

(1) Применим $\psi$ к тождеству $P^2 = P$, получим $\psi(P) = \psi(P P) = \psi(P)\psi(P)$.
То есть число $\psi(P)$ в квадрате равно самому себе.
Значит либо $\psi(P) = 0$, либо $\psi(P) = 1$.

(2) Заметим, что $E^2 = E$ и $0^2 = 0$ и воспользуемся предыдущим пунктом.

(3) Применим $\psi$ к тождеству $A = A E$, получим $\psi(A) = \psi(A)\psi(E) = 0$.

(4) Применим $\psi$ к тождеству $0 = A 0$, получим $\psi(0) = \psi(A) \psi(0)$.
И так как $\psi(0) =1$ по предположению, то $\psi(A) = 1$.

(5) Применим $\psi$ к тождеству $A A^{-1} = E$, получим $1 = \psi(E) = \psi(AA^{-1})=\psi(A)\psi(A^{-1})$.
Значит число $\psi(A^{-1})$ является обратным к числу $\psi(A)$, что и требовалось показать.
\end{proof}

\begin{claim}
\label{claim::MultiOnElementary}
Пусть $\psi\colon \Matrix{n}\to \mathbb R$ -- отображение, удовлетворяющее свойствам:
\begin{enumerate}
\item $\psi(AB) = \psi(A)\psi(B)$ для любых $A,B\in\Matrix{n}$.

\item $\psi(D_n(\lambda)) = \lambda$ для любого ненулевого $\lambda\in\mathbb R$.
\end{enumerate}
Тогда
\begin{enumerate}
\item $\psi(S_{ij}(\lambda)) = 1 = \det(S_{ij}(\lambda))$.

\item $\psi(U_{ij}) = -1 = \det(U_{ij})$.

\item $\psi(D_i(\lambda)) = \lambda = \det(D_i(\lambda))$.
\end{enumerate}
\end{claim}
\begin{proof}
В начале заметим, что $\psi(E) = 1$.
Потому что иначе $\psi(A) = 0$ для любой матрицы, что противоречит второму свойству.
А раз $\psi(E)=1$, то можно пользоваться пунктом~(5) предыдущего утверждения.

(1) Для доказательства воспользуемся следующим замечанием: если $A,B\in\Matrix{n}$ -- произвольные обратимые матрицы, то $\psi(ABA^{-1}B^{-1}) = 1$.
Действительно, 
\[
\psi(ABA^{-1}B^{-1}) = \psi(A)\psi(B)\psi(A)^{-1}\psi(B)^{-1}=\psi(A)\psi(A)^{-1}\psi(B)\psi(B)^{-1} = 1
\]
Для доказательства нам достаточно представить $S_{ij}(\lambda)$ в таком виде.
Давайте проверим, что 
\[
S_{ij}(\lambda) = D_i(2) S_{ij}(\lambda)D_i^{-1}(2)S_{ij}(\lambda)^{-1}
\]
Это равенство проверяется непосредственно глядя на матрицы.
Давайте для простоты проверим в случае $2$ на $2$, когда все наглядно:
\[
\begin{pmatrix}
{2}&{0}\\
{0}&{1}\\
\end{pmatrix}
\begin{pmatrix}
{1}&{\lambda}\\
{0}&{1}\\
\end{pmatrix}
\begin{pmatrix}
{\frac{1}{2}}&{0}\\
{0}&{1}\\
\end{pmatrix}
\begin{pmatrix}
{1}&{-\lambda}\\
{0}&{1}\\
\end{pmatrix}
=
\begin{pmatrix}
{1}&{2\lambda}\\
{0}&{1}\\
\end{pmatrix}
\begin{pmatrix}
{1}&{-\lambda}\\
{0}&{1}\\
\end{pmatrix}
=
\begin{pmatrix}
{1}&{\lambda}\\
{0}&{1}\\
\end{pmatrix}
\]

(3) Для доказательства этого пункта воспользуемся следующим наблюдением: если $A,B\in\Matrix{n}$ причем $A$ обратима, тогда $\psi(ABA^{-1}) = \psi(B)$.
Действительно, 
\[
\psi(ABA^{-1}) = \psi(A)\psi(B)\psi(A)^{-1} =  \psi(B)\psi(A)\psi(A)^{-1} = \psi(B)
\]
Мы уже знаем, что $\psi(D_n(\lambda)) = \lambda$ по условию.
Надо лишь доказать, что для всех $i$ выполнено $\psi(D_i(\lambda)) = \lambda$.
Для этого достаточно представить $D_{i}(\lambda) = A D_{i+1}(\lambda)A^{-1}$.
Возьмем в качестве $A = U_{i, i+1}$ элементарную матрицу переставляющую $i$ и $i+1$ строки.
Тогда $A^{-1} = A$.
Более того, легко видеть, что $D_{i}(\lambda) = U_{i, i+1} D_{i+1}(\lambda)U_{i, i+1}^{-1}$.
Действительно, умножение на $U_{i, i+1}$ слева переставляет $i$ и $i+1$ строки, а умножение на $U_{i, i+1}$ справа равносильно умножению на $U_{i, i+1}^{-1}$ и оно переставляет $i$ и $i+1$ столбцы.
Для наглядности двумерный случай:
\[
\begin{pmatrix}
{0}&{1}\\
{1}&{0}\\
\end{pmatrix}
\begin{pmatrix}
{1}&{0}\\
{0}&{\lambda}\\
\end{pmatrix}
\begin{pmatrix}
{0}&{1}\\
{1}&{0}\\
\end{pmatrix}
=
\begin{pmatrix}
{\lambda}&{0}\\
{0}&{1}\\
\end{pmatrix}
\]

(2) Здесь мы воспользуемся тем, что элементарные преобразования второго типа можно выразить через элементарные преобразования первого и третьего типа, а именно, давайте проверим, что 
\[
U_{ij} = D_i(-1)S_{ji}(1)S_{ij}(-1)S_{ji}(1)
\]
Применив $\psi$ к этому равенству и воспользовавшись предыдущими двумя пунктами мы получаем требуемое.
Однако, остается законный вопрос: а как вообще можно догадаться до такого и проверить?
Вот вам рассуждение приводящее к такому ответу.
Давайте последовательно применять элементарные преобразования первого и третьего типа к единичной матрице, пока не получим из нее матрицу $U_{ij}$.
Написанное равенство означает, что надо сделать так: (1) прибавить $i$ строку к $j$, (2) вычесть $j$ строку из $i$, (3) прибавить $i$ строку к $j$, (4) умножить $i$ строку на $-1$.
Давайте для наглядности это проделаем на матрицах $2$ на $2$.
Ниже мы последовательно умножаем матрицу с левой стороны слева на матрицу, написанную над стрелкой:
\[
\xymatrix@R=10pt@C=40pt{
 	{
 	\begin{pmatrix}
	{1}&{0}\\
	{0}&{1}\\
	\end{pmatrix}
	}\ar[r]^-{
	\begin{pmatrix}
	{1}&{0}\\{1}&{1}\\
	\end{pmatrix}
	}&{
 	\begin{pmatrix}
	{1}&{0}\\
	{1}&{1}\\
	\end{pmatrix}
	}\ar[r]^-{
 	\begin{pmatrix}
	{1}&{-1}\\
	{0}&{1}\\
	\end{pmatrix}
	}&{
 	\begin{pmatrix}
	{0}&{-1}\\
	{1}&{1}\\
	\end{pmatrix}
	}\ar[r]^{
 	\begin{pmatrix}
	{1}&{0}\\
	{1}&{1}\\
	\end{pmatrix}
	}&{
 	\begin{pmatrix}
	{0}&{-1}\\
	{1}&{0}\\
	\end{pmatrix}
	}\ar[r]^{
 	\begin{pmatrix}
	{-1}&{0}\\
	{0}&{1}\\
	\end{pmatrix}
	}&{
 	\begin{pmatrix}
	{0}&{1}\\
	{1}&{0}\\
	\end{pmatrix}
	}
}
\]
\end{proof}

\begin{claim}
\label{claim::MultiOnZeroRow}
Пусть $\psi\colon \Matrix{n}\to \mathbb R$ -- отображение, удовлетворяющее свойствам:
\begin{enumerate}
\item $\psi(AB) = \psi(A)\psi(B)$ для любых $A,B\in\Matrix{n}$.

\item $\psi(D_n(\lambda)) = \lambda$ для любого ненулевого $\lambda\in\mathbb R$.
\end{enumerate}
И пусть $P\in\Matrix{n}$ матрица с нулевой строкой.
Тогда $\psi(P) = 0$.
\end{claim}
\begin{proof}
Пусть в матрице $P$ нулевой является $i$-я строка.
Тогда $D_i(\lambda) P = P$ при любом ненулевом $\lambda \in \mathbb R$.
Применим к этому равенству $\psi$ и получим
\[
\lambda \psi(P) = \psi(D_i(\lambda)) \psi(P) = \psi(P)
\]
Выберем любое ненулевое число $\lambda$ отличное от $1$, тогда получим, что $\psi(P)$ обязано быть нулем.
\end{proof}

\begin{proof}
[Доказательство Утверждения~\ref{claim::DetMultiUnique}]
В начале пусть $A\in\Matrix{n}$ -- невырожденная матрица.
Тогда мы знаем, что она является произведением элементарных матриц $A = U_1\ldots U_k$.
Применим $\psi$ к этому равенству, получим $\psi(A) = \psi(U_1)\ldots \psi(U_k)$.
С другой стороны по утверждению~\ref{claim::MultiOnElementary} получаем $\psi(A) = \det(U_1)\ldots \det(U_k)$.
А из мультипликативности определителя следует, что правая часть равна $\det A$.
То есть $\psi$ совпадает с $\det$ на невырожденных матрицах.

Теперь покажем, что $\psi$ совпадает с $\det$ на всех матрицах.
Пусть $A\in\Matrix{n}$ -- вырожденная матрица.
Тогда элементарными преобразованиями строк она приводится к ступенчатому виду, то есть $A$ можно представить в виде $TB$, где $T$ -- обратимая, а $B$ имеет улучшенный ступенчатый вид.
Так как $A$ вырождена, матрица $B$ имеет нулевую строку.
Теперь применим к равенству $A = TB$ отображение $\psi$ и $\det$.
Получим
\[
\psi(A) = \psi(T) \psi(B)\quad\text{и}\quad \det(A)=\det(T) \det(B)
\]
Но мы знаем по утверждению~\ref{claim::MultiOnZeroRow}, что $\psi(B) = 0$.
Кроме того, мы знаем, что определитель от матриц с нулевой строкой тоже равен нулю по утверждению~\ref{claim::DetZero}.
Значит $\psi(A) = 0 = \det(A)$.
\end{proof}

\subsection{Миноры и алгебраические дополнения}

\paragraph{Определения}

Пусть $B\in\Matrix{n}$ -- некоторая матрица с $b_{ij}$.
Рассмотрим матрицу $D_{ij}\in\Matrix{n-1}$ полученную из $B$ вычеркиванием $i$-ой строки и $j$-го столбца.
Определитель матрицы $D_{ij}$ обозначается $M_{ij}$ и называется {\it минором} матрицы $B$ или $i\,j$-минором для определенности.
Число $A_{ij} = (-1)^{i+j}M_{ij}$ называется {\it алгебраическим дополнением} элемента $b_{ij}$ или $i\,j$-алгебраическим дополнением матрицы $B$.

Покажем как все это выглядит на картинках.
Если мы представим матрицу $B$ в виде
\[
B =
\left(
\begin{array}{c|c|c}
\cline{2-2}
{X_{ij}}&{
\begin{array}{c}
{*}\\{\vdots}
\end{array}
}&{Y_{ij}}\\
\hline
\multicolumn{1}{|c|}{
\begin{array}{cc}
{*}&{\ldots}
\end{array}
}&{b_{ij}}&\multicolumn{1}{c|}{
\begin{array}{cc}
{\ldots}&{*}
\end{array}
}\\
\hline
{Z_{ij}}&{
\begin{array}{c}
{\vdots}\\{*}
\end{array}
}&{W_{ij}}\\
\cline{2-2}
\end{array}
\right)
\]
Тогда
\[
D_{ij} =
\begin{pmatrix}
{X_{ij}}&{Y_{ij}}\\
{Z_{ij}}&{W_{ij}}\\
\end{pmatrix},\quad
M_{ij} = 
\det
\begin{pmatrix}
{X_{ij}}&{Y_{ij}}\\
{Z_{ij}}&{W_{ij}}\\
\end{pmatrix}\quad\text{и}\quad
A_{ij} =
(-1)^{i+j}
\det
\begin{pmatrix}
{X_{ij}}&{Y_{ij}}\\
{Z_{ij}}&{W_{ij}}\\
\end{pmatrix}
\]
{\it Присоединенная матрица} $\hat B$ для $B$ определяется как
\[
\hat B = 
\begin{pmatrix}
{A_{11}}&{A_{21}}&{\ldots}&{A_{n1}}\\
{A_{12}}&{A_{22}}&{\ldots}&{A_{n2}}\\
{\vdots}&{\vdots}&{\ddots}&{\vdots}\\
{A_{1n}}&{A_{2n}}&{\ldots}&{A_{nn}}\\
\end{pmatrix}
\]
То есть надо в матрице $B$ каждый элемент $b_{ij}$ заменить на его алгебраическое дополнение $A_{ij}$, а потом полученную матрицу транспонировать.
Полезно держать перед глазами формулу для элемента присоединенной матрицы $\hat B_{ij} = A_{ji}$.

\paragraph{Формула разложения по строке}

\begin{claim}
\label{claim::DetExpand}
Пусть $B\in\Matrix{n}$ -- произвольная матрица.
Тогда%
\footnote{Всюду в формулах $A_{ij}$ обозначает алгебраическое дополнение.}
\begin{enumerate}
\item Для любой строки $i$ верно разложение
\[
\det B = \sum_{j=1}^n b_{ij} A_{ij}
\]

\item Для любого столбца $j$ верно разложение
\[
\det B = \sum_{i=1}^n b_{ij} A_{ij}
\]
\end{enumerate}
\end{claim}
\begin{proof}
Мы докажем формулу для строки, для столбца она получается аналогично либо применением транспонирования к матрице.
Рассмотрим $i$-ю строку в матрице $B$
\[
B =
\left(
\begin{array}{ccc}
{X_{ij}}&{
\begin{array}{c}
{*}\\{\vdots}
\end{array}
}&{Y_{ij}}\\
\hline
\multicolumn{1}{|c}{
\begin{array}{cc}
{b_{i1}}&{\ldots}
\end{array}
}&{b_{ij}}&\multicolumn{1}{c|}{
\begin{array}{cc}
{\ldots}&{b_{in}}
\end{array}
}\\
\hline
{Z_{ij}}&{
\begin{array}{c}
{\vdots}\\{*}
\end{array}
}&{W_{ij}}\\
\end{array}
\right)
\]
Эту строку можно разложить в сумму следующих строк 
\[
(b_{i1},\ldots,b_{in}) = \sum_{j=1}^n(0,\ldots,0,b_{ij},0,\ldots,0)
\]
Теперь вычислим определитель $B$ пользуясь линейностью по $i$-ой строке
\[
\det B = 
\sum_{j=1}^n
\det
\left(
\begin{array}{c|c|c}
\cline{2-2}
{X_{ij}}&{
\begin{array}{c}
{*}\\{\vdots}
\end{array}
}&{Y_{ij}}\\
\hline
\multicolumn{1}{|c|}{
\begin{array}{cc}
{0}&{\ldots}
\end{array}
}&{b_{ij}}&\multicolumn{1}{c|}{
\begin{array}{cc}
{\ldots}&{0}
\end{array}
}\\
\hline
{Z_{ij}}&{
\begin{array}{c}
{\vdots}\\{*}
\end{array}
}&{W_{ij}}\\
\cline{2-2}
\end{array}
\right)
\]
Теперь отдельно посчитаем следующий определитель
\[
\det 
\left(
\begin{array}{c|c|c}
\cline{2-2}
{X_{ij}}&{
\begin{array}{c}
{*}\\{\vdots}
\end{array}
}&{Y_{ij}}\\
\hline
\multicolumn{1}{|c|}{
\begin{array}{cc}
{0}&{\ldots}
\end{array}
}&{b_{ij}}&\multicolumn{1}{c|}{
\begin{array}{cc}
{\ldots}&{0}
\end{array}
}\\
\hline
{Z_{ij}}&{
\begin{array}{c}
{\vdots}\\{*}
\end{array}
}&{W_{ij}}\\
\cline{2-2}
\end{array}
\right)
=
(-1)^{j-1}
\det
\left(
\begin{array}{|c|cc}
\cline{1-1}
{
\begin{array}{c}
{*}\\{\vdots}
\end{array}
}&{X_{ij}}&{Y_{ij}}\\
\hline
{b_{ij}}&{\ldots}&\multicolumn{1}{c|}{0}\\
\hline
{
\begin{array}{c}
{\vdots}\\{*}
\end{array}
}&{Z_{ij}}&{W_{ij}}\\
\cline{1-1}
\end{array}
\right)
=
(-1)^{j-1}(-1)^{i-1}
\det
\left(
\begin{array}{|c|cc}
\hline
{b_{ij}}&{\ldots}&\multicolumn{1}{c|}{0}\\
\hline
{\vdots}&{X_{ij}}&{Y_{ij}}\\
{*}&{Z_{ij}}&{W_{ij}}\\
\cline{1-1}
\end{array}
\right)
\]
В первом равенстве мы переставили $j$-ый столбец $j-1$ раз, чтобы переместить его на место первого столбца.
Во втором равенстве мы переставили $i$-ю строку $i-1$ раз, чтобы переставить ее на место первой строки.
Последняя матрица является блочно нижнетреугольной, а следовательно, равенство можно продолжить так
\[
(-1)^{i+j} b_{ij}
\det
\begin{pmatrix}
{X_{ij}}&{Y_{ij}}\\
{Z_{ij}}&{W_{ij}}\\
\end{pmatrix}
= b_{ij}(-1)^{i+j}M_{ij} = b_{ij}A_{ij}
\]
\end{proof}

\paragraph{Явные формулы для обратной матрицы}

\begin{claim}
\label{claim::InvMatExplicite}
Для любой матрицы $B\in \Matrix{n}$ верно 
\[
\hat B B = B\hat B = \det(B) E
\]
\end{claim}
\begin{proof}
Нам надо отдельно доказать два равенства $\hat B B = \det (B) E$ и $B\hat B = \det (B) E$.
Давайте докажем второе равенство, а первое показывается аналогично (или через трюк с транспонированием).

Для доказательства $B\hat B = \det (B) E$ нам надо показать две вещи:
(1) все диагональные элементы матрицы $B\hat B$ равны $\det (B)$, (2) все внедиагональные элементы равны нулю.

(1) Рассмотрим $i$-ый диагональный элемент в матрице $B\hat B$:
\[
(B\hat B)_{ii} = \sum_{j=1}^n b_{ij}\hat B_{ji} = \sum_{j=1}^n b_{ij}A_{ij}=\det(B)
\]
Последняя формула является разложением определителя $\det (B)$ по $i$-ой строке из утверждения~\ref{claim::DetExpand}.

(2) Рассмотрим элемент на позиции $i\,j$ для $i\neq j$:
\[
(B\hat B)_{ij} = \sum_{k=1}^n b_{ik}\hat B_{kj} = \sum_{k=1}^n b_{ik}A_{jk}
\]
Нам надо показать, что последнее выражение равно нулю.
Давайте рассмотрим матрицу $B$ и заменим в ней $j$-ю строку на $i$-ю, все остальные оставим нетронутыми.
Обозначим полученную матрицу через $B'$.
Тогда
\[
B' =
\left(
\begin{array}{ccccc}
{*}&{\ldots}&{*}&{\ldots}&{*}\\
\hline
\multicolumn{1}{|c}{b_{i1}}&{\ldots}&{b_{ik}}&{\ldots}&\multicolumn{1}{c|}{b_{in}}\\
\hline
{*}&{\ldots}&{*}&{\ldots}&{*}\\
\hline
\multicolumn{1}{|c}{b_{i1}}&{\ldots}&{b_{ik}}&{\ldots}&\multicolumn{1}{c|}{b_{in}}\\
\hline
{*}&{\ldots}&{*}&{\ldots}&{*}\\
\end{array}
\right)
\]
Давайте посчитаем определитель $B'$ двумя способами.
С одной стороны $\det(B') = 0$ так как в матрице есть две одинаковые строки.
С другой стороны, давайте разложим определитель $\det(B')$ по $j$-ой строке
\[
\det (B') = \sum_{k=1}^n b_{ik}A_{jk}
\]
Что и требовалось доказать.
\end{proof}

В качестве непосредственного следствия этого утверждения получаем явные формулы обратной матрицы.%
\footnote{Заметим, что для формулы требуется условие $\det (B)\neq 0$.
Однако, матрица обратима тогда и только тогда, когда $\det(B)\neq 0$.
Один из способов это показать -- применить $\det$ к равенству $B B^{-1} = E$ и увидеть, что $\det(B) \det(B^{-1}) = 1$.
А в обратную сторону -- явные формулы.}

\begin{claim}
[Явные формулы обратной матрицы]
Пусть $B\in\Matrix{n}$ -- обратимая матрица, тогда 
\[
B^{-1} = \frac{1}{\det(B)}\hat B
\]
\end{claim}

Заметим, что в случае матрицы $2$ на $2$ формулы принимают следующий вид
\[
\begin{pmatrix}
{a}&{b}\\
{c}&{d}\\
\end{pmatrix}^{-1}
=
\frac{1}{ad - bc}
\begin{pmatrix}
{d}&{-b}\\
{-c}&{a}\\
\end{pmatrix}
\]

\subsection{Формулы Крамера}

Пусть $A\in\Matrix{n}$ -- произвольная матрица и $b\in\mathbb R^n$ -- столбец.
Рассмотрим систему линейных уравнений $Ax = b$.
Давайте в матрице $A$ $i$-ый столбец заменим на $b$, а остальные столбцы оставим как есть.
Обозначим полученную матрицу через $\bar A_i$.
Определим $\Delta = \det (A)$ и $\Delta_i = \det (\bar A_i)$.

Мы знаем, что данная система имеет единственное решение для любого $b$ тогда и только тогда, когда матрица $A$ обратима.
Следующее утверждение дает явные формулы для координат решения системы в этом случае.

\begin{claim}
[Формулы Крамера]
Пусть $A\in\Matrix{n}$, $x,b\in \mathbb R^n$ и выполнено равенство $Ax = b$.
Тогда $\Delta \cdot x_i = \Delta_i$ для любого $i$.%
\footnote{Здесь $x_i$ -- координаты вектора $x$.}
\end{claim}
\begin{proof}
Рассмотрим матрицу $A$ как строку из столбцов $A = (A_1|\ldots|A_n)$, где $A_i$ -- столбцы матрицы $A$.
Тогда равенство $Ax = b$, пользуясь блочными формулами, можно переписать так $x_1 A_1 + \ldots + x_n A_n = b$.
Давайте посчитаем определитель $\bar A_i$, пользуясь последним равенством.
\[
\det (\bar A_i) = \det(A_1|\ldots|b|\ldots|A_n) = \det(A_1|\ldots|\sum_{k=1}^n x_k A_k|\ldots|A_n) = \sum_{k=1}^n x_k \det (
\stackrel{i}{A_1|\ldots|A_k|\ldots|A_n})
\]
В последней формуле, если $k\neq i$, то слагаемое имеет два одинаковых столбца $A_i$.
Потому остается только одно слагаемое для $k = i$.
Получаем
\[
\det(\bar A_i) = x_i \det(A_1|\ldots|A_i|\ldots|A_n) = x_i \det(A)
\]
Что и требовалось.
\end{proof}

Заметим, что если $\Delta = \det (A) \neq 0$, то имеется единственное решение системы $Ax = b$ для любой правой части $b$ и координаты этого решения заданы по формулам $x_i =\frac{\Delta_i}{\Delta}$.
Однако, если $\Delta = \det(A) = 0$, то либо решений бесконечное число, либо их вообще нет.
В этом случае единственная информация из формул Крамера это: $\Delta_i = 0$.
