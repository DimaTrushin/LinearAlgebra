\ProvidesFile{lecture23.tex}[Лекция 23]


\subsection{Функции на функциях}

Давайте рассмотрим следующую последовательность векторных пространств
\begin{center}
\begin{tabular}{c|c|c}

{$V$}&{$V^*$}&{$V^{**}$}\\

\hline

{$v$}&{$\xi$}&{$?$}\\

\end{tabular}
\end{center}
В $V$ у нас живут векторы, в $V^*$ функции на векторах, а в $V^{**}$ -- функции на функциях на векторах.
Оказывается, каждый вектор можно рассматривать как функцию на функциях на векторах.
Давайте вспомним наш удобный формализм: если $v\in V$ и $\xi \in V^*$, то $\xi(v)$ надо обозначать так $\xi v$, то есть как произведение.
Тогда зафиксировав $\xi$ мы получим правило линейное по $v$, то есть отображение $V\to F$.
Но, мы с таким же успехом можем зафиксировать $v$ и начать менять левый аргумент.
Тогда получится линейное отображение $V^* \to F$.
Это ломает мозг в записи $\xi(v)$, но когда и функционал и вектор записаны равноправно в виде умножения $\xi v$, такие конструкции становится проще понимать.
Теперь аккуратно.

Для произвольного векторного пространства $V$ построим отображение $\phi\colon V\to V^{**}$, где $v\mapsto \phi_v$.
Чтобы задать это отображение нам надо определить элемент $\phi_v\in V^{**}$, то есть нам надо задать $\phi_v\colon V^*\to F$.
То есть нам надо определить число из поля $\phi_v(\xi)$ для каждого $\xi\in V^*$.
Для этого положим по определению $\phi_v(\xi) = \xi(v)$ -- функционал вычисления на векторе.
Обратим внимание, что отображение $\phi\colon V\to V^{**}$ является линейным отображением между векторными пространствами.

\begin{claim}
\label{claim::DoubleDuoIsom}
Пусть $V$ -- произвольное векторное пространство и $\phi\colon V\to V^{**}$ отображение по правилу $v\mapsto \phi_v$, где $\phi_v(\xi) = \xi(v)$.
Тогда 
\begin{enumerate}
\item Отображение $\phi\colon V\to V^{**}$ инъективно.

\item Если $V$ конечно мерно, то отображение $\phi$ является изоморфизмом.
\end{enumerate}
\end{claim}
\begin{proof}
(1) Так как $\phi$ линейно, нам достаточно показать, что у него нулевое ядро.
Давайте расшифруем, что это значит.
Пусть $v\in V$ такой, что $\phi_v = 0$.
Это значит, что $\phi_v \colon V^* \to F$ является нулевым отображением.
То есть $\phi_v(\xi) = 0$ для любого $\xi \in V^*$.
То есть $\xi(v) = 0$ для любого $\xi \in V^*$.
Но утверждение~\ref{claim::VecPointwiseEq} в точности говорит, что такое бывает только для нулевого вектора.%
\footnote{Формально мы доказали утверждение~\ref{claim::VecPointwiseEq} в конечномерном случае, но в сносках к доказательству сказано как доказательство переносится на бесконечномерный случай.}

(2) Так как $V$ вкладывается в $V^{**}$, то нам достаточно убедиться, что они имеют одинаковую размерность.
А это следует из утверждения~\ref{claim:DualBasis}, так как $\dim V = \dim V^* = \dim V^{**}$.
\end{proof}

Я конечно же постарался привести доказательство в бесконечно мерном случае.
Однако, если все эти бесконечномерные пакости так претят вашей ранимой тонкой душевной организации, можете доказывать это утверждение только для конечно мерных пространств.

\paragraph{Замечания}

\begin{itemize}
\item Если мы стартовали с векторного пространства $V$, то можем начать строить цепочку пространств вида $V, V^*, V^{**}, \ldots$.
Утверждение~\ref{claim:DualBasis} гласит, что с абстрактной точки зрения, все эти пространства одинаковые -- изоморфны и мы ничего не получили нового.

\item Однако, этот абстрактный изоморфизм ничего не знает про <<семантику>> наших пространств, а именно он ничего не знает про операцию применения функционалов к векторам.
Изоморфизм $\phi\colon V\to V^{**}$ из утверждения~\ref{claim::DoubleDuoIsom} согласован с этой семантикой следующим образом.
На паре пространств $V^*$, $V$ есть операция $V^*\times V\to F$ вычисления функционала и на паре пространств $V^{**}$, $V^*$ есть операция $V^{**}\times V^*\to F$ вычисления функционала.
Рассмотрим следующую картинку
\begin{center}
\begin{tabular}{c|c|c}

{$V$}&{$V^*$}&{$V^{**}$}\\

\hline

{$v$}&{$\xi$}&{$\phi_v$}\\

\end{tabular}
\end{center}
Тогда мы можем применить $\xi$ к $v$ и получим $\xi(v)$, можем применить $\phi_v$ к $\xi$ и получим $\phi_v(\xi) = \xi(v)$.
То есть рассматривать пару $(v,\xi)$ можно как пару $(\text{вектор}, \text{функция})$, а можно рассматривать как пару $(\text{функция}, \text{вектор})$.
При этом операция вычисления функции на векторе будет одной и той же и задается правилом $\xi v$ (вот видите, как полезна симметричная запись).%
\footnote{Таким образом нам необходимо знать только про векторы и функционалы.
Кроме того, оказывается, что если определить такую операцию над векторными пространствами как тензорное произведение, то все, что только можно определить в линейной алгебре, можно выразить через векторы и функционалы с помощью тензорного произведения, например, линейные отображения, операторы, билинейные формы (которые будут чуть позже) и много других объектов.
Это все ведет к некоторому единому удобному тензорному языку.}

\item Теперь еще раз рассмотрим последовательность $V, V^*, V^{**}$.
Если мы выберем в $V$ некоторый базис $e_1,\ldots,e_n$, то в $V^*$ можем найти к нему двойственный $\xi_1,\ldots,\xi_n$.
После чего найдем к последнему двойственный $\eta_1,\ldots,\eta_n$ в $V^{**}$.
Но теперь вспомним, что $V^{**}$ изоморфно $V$.
А так как изоморфизм между $V$ и $V^{**}$ согласован с применением функционалов, то базис $\eta_1,\ldots,\eta_n$ перейдет в $e_1,\ldots,e_n$.

\item Чуть ниже я покажу более аккуратно, что значит, что изоморфизм $\phi\colon V\to V^{**}$ в некотором смысле канонический, а между $V$ и $V^*$ таких не бывает.
\end{itemize}


\subsection{Сопряженное линейное отображение}

Выше мы показали, что в конечномерном случае все три пространства $V$, $V^*$ и $V^{**}$ изоморфны между собой.
Однако, в случае $V$ и $V^{**}$ мы не просто показали это, мы построили некоторый замечательнейший изоморфизм между ними.
Ниже очень хочется объяснить, а что же такого замечательного в этом изоморфизме и почему подобного замечательного изоморфизма нет между $V$ и $V^*$.
Оказывается, что изоморфизм $\phi\colon V \to V^{**}$ в некотором смысле согласован с линейными отображениями, а подобного согласованного изоморфизма между $V$ и $V^*$ просто не бывает.
Для того, чтобы объяснить, что все это значит, мне нужно для начала определить сопряженное линейное отображение.

Пусть $\varphi\colon V\to U$ -- некоторое линейное отображение.
Мы хотим определить другое линейное отображение $\varphi^* \colon U^*\to V^*$ следующим образом:
\[
\xymatrix@R=6pt{
	{V}\ar[r]^{\varphi}\ar@{-->}@/^16pt/[rr]^{\xi\circ \varphi}&{U}\ar[r]^{\xi}&{F}\\
	{V^*}&{U^*}\ar[l]_{\phi^*}&{}\\
	{\xi\circ \varphi}&{\xi}\ar@{|->}[l]&{}\\
}
\]
Давайте поясним, что нарисовано на диаграмме.
Нам надо определить $\varphi^*\colon U^*\to V^*$.
То есть для любого $\xi\in U^*$ нам надо задать $\varphi^*(\xi)\in V^*$.
Последнее означает, что по линейному функционалу на $U$, нам надо как-то построить линейный функционал на $V$.
Предлагается определить $\varphi^*(\xi) = \xi \varphi$ как композицию.

\begin{definition}
Для линейного отображения $\varphi\colon V\to U$ сопряженным (или двойственным) линейным отображением называется $\varphi^*\colon U^*\to V^*$ по правилу $\xi \mapsto \varphi^*(\xi) = \xi \varphi$.
\end{definition}

\paragraph{Матрица сопряженного линейного отображения}

\begin{claim}
\label{claim::DualHomMatrix}
Пусть $V$ -- векторное пространство с базисом $e_1,\ldots,e_n$, $U$ -- векторное пространство с базисом $f_1,\ldots,f_m$.
Пусть $e^1,\ldots,e^n$ -- двойственный базис в $V^*$ и $f^1,\ldots,f^m$ -- двойственный базис в $U^*$.
Пусть $\varphi \colon V\to U$ -- некоторый линейное отображение с матрицей $A_\varphi$ в базисах $e_i$ и $f_j$ и пусть $\varphi^*\colon U^*\to V^*$ -- сопряженное линейное отображение с матрицей $A_{\varphi^*}$ в базисах $f^i$ и $e^j$.
Тогда $A_{\varphi^*} = A_{\varphi}^t$.
\end{claim}
\begin{proof}
Для лучшего понимания, я приведу два доказательства: без координатное и координатное.
Выбирайте любое, какое вам больше нравится.

\paragraph{Абстрактное доказательство}

По определению матрицы линейного отображения
\[
\varphi(e_1,\ldots,e_n) = (f_1,\ldots,f_m)A_{\varphi}\quad\text{и}\quad
\varphi^*(f^1,\ldots,f^m) = (e^1,\ldots,e^n)A_{\varphi^*}
\]
Кроме того, по определению двойственного базиса нам дано
\[
\begin{pmatrix}
{e^1}\\{\vdots}\\{e^n}
\end{pmatrix}
\begin{pmatrix}
{e_1}&{\ldots}&{e_n}
\end{pmatrix}
=E\quad\text{и}\quad
\begin{pmatrix}
{f^1}\\{\vdots}\\{f^m}
\end{pmatrix}
\begin{pmatrix}
{f_1}&{\ldots}&{f_m}
\end{pmatrix}
=E
\]
В начале распишем следующее равенство
\[
(f^1\varphi,\ldots,f^m\varphi)=\varphi^*(f^1,\ldots,f^m) = (e^1,\ldots,e^n)A_{\varphi^*}
\]
Теперь транспонируем его и получим
\[
\begin{pmatrix}
{f^1 \varphi}\\{\vdots}\\{f^m\varphi}
\end{pmatrix}
=
\begin{pmatrix}
{f^1}\\{\vdots}\\{f^m}
\end{pmatrix} \varphi
=
A_{\varphi^*}^t
\begin{pmatrix}
{e^1}\\{\vdots}\\{e^n}
\end{pmatrix}
\]
Умножим левую и праву часть полученного равенства на $(e_1,\ldots,e_n)$, получим
\begin{gather*}
\begin{pmatrix}
{f^1}\\{\vdots}\\{f^m}
\end{pmatrix} 
\varphi (e_1,\ldots,e_n)
=
A_{\varphi^*}^t
\begin{pmatrix}
{e^1}\\{\vdots}\\{e^n}
\end{pmatrix}
\begin{pmatrix}
{e_1}&{\ldots}&{e_n}
\end{pmatrix}\\
\begin{pmatrix}
{f^1}\\{\vdots}\\{f^m}
\end{pmatrix} 
(f_1,\ldots,f_m)A_{\varphi}
=
A_{\varphi^*}^t\\
A_{\varphi} = A_{\varphi^*}^t
\end{gather*}
что и требовалось.

\paragraph{Координатное доказательство}

Так как в пространствах $V$ и $V^*$ фиксированы базис и двойственный к нему, то можно считать, что $V = F^n$ -- пространство столбцов, $V^* = F^n$ -- пространство строк, а применение функции к вектору -- умножение строки на столбец.
Аналогично, можно считать, что $U = F^m$ -- пространство столбцов, $U^*=F^m$ -- пространство строк.
Тогда линейное отображение $\varphi$ действует по правилу $\varphi(x) = A_{\varphi} x$.
Значит по определению строка $\xi \in F^m$ переходит в $\varphi^*(\xi) = \xi \circ \varphi = \xi A_{\varphi}$, то есть строка $\xi\in F^m$ переходит в строку $\xi A_{\varphi}$.
Но чтобы получить матрицу линайного отображения, надо записать координаты вектора в столбцы, тогда получим $\xi^t \mapsto A_{\varphi}^t \xi^t$.
То есть отображение $\varphi^*$ действует с помощью матрицы $A_{\varphi}^t$.
\end{proof}

\paragraph{Замечание}

То есть транспонирование матриц имеет следующий философский смысл: это переход к сопряженному линейному отображению в двойственном пространстве.
Заметили, что транспонирование и звездочка меняют местами порядок отображений?
Последнее утверждение показывает, что это не случайное совпадение.

\paragraph{Функториальность звездочки}

Обратите внимание, что мы теперь построили очень любопытный математический агрегат.
А именно, пусть у нас есть мешок всех векторных пространств и линейных отображений между ними.
Тогда для каждого векторного пространства $V$ мы можем построить новое векторное пространство $V^*$.
Кроме того, мы умеем действовать не только на векторных пространствах, но и на отображениях между ними.
То есть каждому отображению $\varphi\colon V\to U$ мы ставим в соответствие $\varphi^*\colon U^*\to V^*$.
Про это надо думать так: все векторные пространства образуют (охренительно огроменнейший) граф, у которого вершины -- векторные пространства, а ребра -- линейные отображения.
Мы построили отображение из этого графа в себя, которое разворачивает стрелки.
Кроме того, это отображение согласовано с композицией в следующем смысле:%
\footnote{Я не хочу вводить формальные определения, но мы реально только что построили контровариантный функтор из категории векторных пространств в себя, что бы это ни значило.}
\begin{itemize}
\item Если $\Identity\colon V\to V$, тогда $\Identity^*\colon V^*\to V^*$ является тождественным отображением, то есть $\Identity^* = \Identity$.

\item Если $\varphi\colon V\to U$ и $\psi\colon U\to W$, тогда $(\psi\varphi)^* = \varphi^*\psi^*$, то есть звездочка меняет местами порядок отображений.
Графически это можно изобразить так:
\[
\text{Если коммутативна диаграмма }
\xymatrix@R=6pt{
	{V}\ar[r]^-{\varphi}\ar@{-->}@/^16pt/[rr]^{\psi\circ \varphi}&{U}\ar[r]^-{\psi}&{W,}\\
}
\text{ то коммутативна диаграмма }
\xymatrix@R=6pt{
	{V^*}&{U^*}\ar[l]_-{\varphi^*}&{W^*}\ar[l]_-{\psi^*}\ar@{-->}@/_16pt/[ll]_{(\psi\circ \varphi)^*}\\
}
\]
Коммутативность диаграммы, означает, что любые два пути ведущие из одной вершины в другую приводят к одному результату.

\item Так же отметим, что звездочка согласована со структурой векторного пространства.
А именно для любых $\varphi, \psi \colon V\to U$ и числе $\alpha, \beta\in F$ верно $(\alpha \varphi + \beta \psi)^* = \alpha \varphi^* + \beta\psi^* \colon U^* \to V^*$.
\end{itemize}

\paragraph{Канонический изоморфизм}

\begin{claim}
\label{claim::CanonicalIsomorphism}
Пусть $\varphi\colon V\to U$ -- линейное отображение и $\phi\colon V\to V^{**}$ и $\phi\colon U\to U^{**}$ -- канонические изоморфизмы на второе сопряженное.
В этом случае коммутативна следующая диаграмма:
\[
\xymatrix{
	{V}\ar[r]^{\varphi}\ar[d]^{\phi}&{U}\ar[d]^{\phi}\\
	{V^{**}}\ar[r]^{\varphi^{**}}&{U^{**}}
}
\]
то есть $\phi \varphi = \varphi^{**}\phi$.
\end{claim}
\begin{proof}
Давайте распишем, как вектор $v\in V$ двигается по этой диаграмме
\[
\xymatrix{
	{v}\ar@{|->}[rr]\ar@{|->}[d]&{}&{\varphi(v)}\ar@{|->}[d]\\
	{\phi_v}\ar@{|->}[r]&{\phi_v\circ \varphi^*}\ar@{=}[r]^{?}&{\phi_{\varphi(v)}}
}
\]
И нам надо проверить равенство с вопросиком.
Для этого надо сравнить действие левой и правой части на произвольном $\xi\in U^*$.
Получим
\[
(\phi_v\circ \varphi^*)(\xi) = \phi_v(\varphi^*(\xi)) = \phi_v(\xi\circ\varphi) = (\xi\circ \varphi) (v) = \xi(\varphi(v)) = \phi_{\varphi(v)}(\xi)
\]
что и требовалось.
\end{proof}

\paragraph{Замечания}

\begin{itemize}
\item Последнее утверждение показывает, что изоморфизмы $\phi\colon V\to V^{**}$ согласованы с линейными отображениями в том смысле, что коммутативны некоторые диаграммы.
Как надо думать про это?
Смысл $\varphi$ в том, что мы можем считать, что $V$ и $V^{**}$ -- это одно и то же.
Согласованность с отображениями означает, что при таком отождествлении $V$ с $V^{**}$ и $U$ с $U^{**}$ отображение $\varphi$ превращается в $\varphi^{**}$.

\item С другой стороны, если мы захотим потребовать нечто подобное для $V$ и $V^*$ то мы получим диаграмму вида
\[
\xymatrix{
	{V}\ar[r]^{\varphi}\ar[d]^{\phi}&{U}\ar[d]^{\phi}\\
	{V^*}&{U^*}\ar[l]^{\varphi^*}
}
\]
Причем эта диаграмма должна быть коммутативной (то есть $\phi = \varphi^*\phi\varphi$) для всех отображений $\varphi\colon V\to U$, а вертикальные стрелки $\phi$ все должны быть изоморфизмами.
Но выберем тогда в качестве $\varphi$ нулевое отображение и получим, что $\phi = 0$, противоречие.
\end{itemize}



\newpage
\section{Билинейные формы}

Всем хороши векторные пространства.
Их элементы можно складывать и умножать на числа.
Одна беда -- векторы нельзя перемножать.
А очень хочется.
А мы знаем, что когда нельзя, но очень хочется, то можно, а скорее даже нужно.%
\footnote{На самом деле вся математика устроена именно так.}
Наша следующая задача научиться перемножать векторы.
Оказывается, что способов перемножать векторы много и результаты этих перемножений могут лежать где угодно.
Большие дяди и тети для этих целей используют тензорные произведения, потому что это лучший способ перемножить векторы.
Мы же пока еще маленькие и будем заниматься всем этим запретным делом понарошку.
Потому для наших целей сгодятся билинейные формы.
Оказывается, что с помощью них можно будет уметь решать два вида задач: исследовать искривление гладких поверхностей или задавать углы и расстояния в вещественных пространствах.
Так что изучаем мы их не из праздного любопытства, они нам еще пригодятся, но позже.

\subsection{Определение и примеры}

\begin{definition}
\label{def::BilinearForms}
Пусть $V$ и $U$ -- векторные пространства над полем $F$.
Билинейная форма на паре пространств $V$ и $U$ -- это билинейное отображение $\beta\colon V\times U \to F$, то есть такое отображение, что выполнены следующие свойства:
\begin{enumerate}
\item $\beta(v_1 + v_2, u) = \beta(v_1,u) + \beta(v_2,u)$ для всех $v_1,v_2\in V$ и $u\in U$.

\item $\beta(\lambda v, u) = \lambda \beta(v,u)$ для всех $v\in V$, $u\in U$ и $\lambda\in F$.

\item $\beta(v, u_1+u_2) = \beta(v, u_1) + \beta(v,u_2)$ для всех $v\in V$ и $u_1,u_2\in U$.

\item $\beta(v,\lambda u) = \lambda\beta(v,u)$ для всех $v\in V$, $u\in U$ и $\lambda\in F$.
\end{enumerate}
\end{definition}

Таким образом, билинейная форма на $V$ и $U$ -- это правило, которому скармливают два вектора (один из $V$, другой из $U$), а на выходе оно выдает нам число.
Причем это число линейно зависит от каждого из аргументов.
Множество всех билинейных форм на паре пространств $V$ и $U$ будем обозначать через $\Bil(V,U)$.

Билинейная форма -- это функция от двух аргументов.
А про функции от двух аргументов можно думать, как про оператор.
Давайте запишем нашу билинейную форму $\beta\colon V\times U \to F$ в следующем виде $\beta(v,u) = v \cdot_\beta u$, где выражение справа -- это все та же билинейная форма но в операторной записи.
Тогда определение билинейной формы можно переписать так:
\begin{definition}
Пусть $V$ и $U$ -- векторные пространства над полем $F$.
Билинейная форма на паре пространств $V$ и $U$ -- это отображение $\cdot_\beta\colon V\times U \to F$ такое, что
\begin{enumerate}
\item $(v_1 + v_2)\cdot_\beta u = v_1\cdot_\beta u + v_2\cdot_\beta u$ для всех $v_1,v_2\in V$ и $u\in U$.

\item $(\lambda v)\cdot_\beta u = \lambda (v\cdot_\beta u)$ для всех $v\in V$, $u\in U$ и $\lambda\in F$.

\item $v\cdot_\beta( u_1+u_2) = v\cdot_\beta u_1 + v\cdot_\beta u_2$ для всех $v\in V$ и $u_1,u_2\in U$.

\item $v\cdot_\beta(\lambda u) = \lambda (v\cdot_\beta u)$ для всех $v\in V$, $u\in U$ и $\lambda\in F$.
\end{enumerate}
\end{definition}

То есть наше определение превращается в определение умножения дистрибутивного по обоим аргументам и согласованное с умножением на скаляр.

\paragraph{Примеры}

\begin{enumerate}
\item Начнем с самого популярного примера: $\beta\colon F^n\times F^n \to F$ по правилу $\beta(x,y) = x^t y$.
Этот товарищ нам известен в случае $F = \mathbb R$ и $n\leqslant 3$ как скалярное произведение.
Над произвольным полем у него уже нет такой выделенной роли (даже над $\mathbb C$ оно уже не так хорошо как над $\mathbb R$), однако, это достаточно общий пример, как будет видно из классификационной теоремы.
% TO DO
% добавить ссылку на будущую классификационную теорему

\item Пусть $C[a,b]$ -- множество непрерывных функций на отрезке $[a,b]$, тогда рассмотрим следующую форму $\beta\colon C[a,b]\times C[a,b]\to \mathbb R$ по правилу $(f,g)\mapsto \int_a^b f(x)g(x)\,dx$.

\item Отображение $\operatorname{M}_n(F)\times \operatorname{M}_n(F)\to F$ по правилу $(A,B)\mapsto \tr(A^tB)$ так же является билинейной формой.
Тут можно было бы использовать $\tr(AB)$ или другие разновидности.
Однако, версия $\tr(A^tB)$ в случае поля $\mathbb R$ обладает весьма замечательными свойствами, как и пример из пункта~(1).

\item До сих пор у нас были примеры на одном векторном пространстве.
Пусть $V$ -- некоторое векторное пространство, тогда отображение $\langle,\rangle\colon V^*\times V\to F$ по правилу $(\xi,v)\mapsto \langle\xi,v\rangle := \xi(v)$ называется естественной билинейной формой.%
\footnote{Это один из главных примеров, ради которого и вводится понятие билинейной формы на паре разных пространств.
Основной плюс от этого подхода в том, что выражение $\langle \xi,v\rangle$ симметрично относительно своих аргументов в отличие от $\xi(v)$ и позволяет думать и работать с векторами и функциями на равных правах.}%
${}^{,}$%
\footnote{Про эту симметрию мы уже говорили в замечании после определения~\ref{def::DualBasis}.}
\end{enumerate}

\subsection{Матрица билинейной формы}

При изучении любого объекта один из первых вопросов: <<а как этот объект задавать?>> Сейчас мы коснемся этого вопроса для билинейных форм и начнем со следующего.

\begin{definition}
Пусть $\beta\colon V\times U\to F$ -- некоторая билинейная форма,  $e_1,\ldots,e_n\in V$ -- базис пространства $V$ и $f_1,\ldots,f_m\in U$ -- базис пространства $U$.
Тогда матрица $B_\beta$ с коэффициентами $b_{ij} = \beta(e_i,f_j)$ называется матрицей билинейной формы $\beta$ в паре базисов $e_1,\ldots,e_n$ и $f_1,\ldots,f_m$.
\end{definition}

\begin{claim}
\label{claim::BilinearBasis}
Пусть $\beta\colon V\times U\to F$ -- некоторая билинейная форма,  $e = (e_1,\ldots,e_n)$ -- базис пространства $V$ и $f=(f_1,\ldots,f_m)$ -- базис пространства $U$.
Пусть $v = ex$, $x\in F^n$, $u =fy$, $y\in F^m$ и $B$ -- матрица билинейной формы $\beta$ в базисах $e$ и $f$.
Тогда $\beta(v,u) = x^t B y$.
\end{claim}
\begin{proof}
Действительно, 
\[
\beta(v,u) = \beta(\sum_{i=1}^n x_i e_i, \sum_{j=1}^m y_j f_j) = \sum_{i,j} x_iy_j\beta(e_i, f_j) = x^t B y
\]
\end{proof}

Таким образом, когда вы работаете с парой пространств $V$ и $U$, после выбора базиса они превращаются в $F^n$ и $F^m$, соответственно, а билинейная форма $\beta\colon V\times U\to F$ превращается в отображение $\beta\colon F^n \times F^m \to F$ по правилу $(x,y)\mapsto x^t B y$.

\begin{claim}
\label{claim::BilinearMatrices}
Пусть $\beta\colon V\times U\to F$ -- некоторая билинейная форма,  $e = (e_1,\ldots,e_n)$ -- базис пространства $V$ и $f=(f_1,\ldots,f_m)$ -- базис пространства $U$.
Тогда отображение $\operatorname{Bil}(V,U)\to \operatorname{M}_{n\,m}(F)$ по правилу $\beta\mapsto B_\beta$ является биекцией.
\end{claim}
\begin{proof}
Из утверждения~\ref{claim::BilinearBasis} следует, что $\beta(x,y) = x^t B_\beta y$.
Значит, билинейная форма восстанавливается по своей матрице и отображение $\beta\mapsto B_\beta$ инъективно.
Обратно, если $B\in \operatorname{M}_{n\,m}(F)$ -- произвольная матрица, то рассмотрим форму $\beta(x,y) = x^t B y$.
Тогда по определению $B = B_\beta$.
\end{proof}

\paragraph{Матричный формализм}

Пусть $\beta\colon V\times U\to F$ -- билинейная форма и пусть $v = (v_1,\ldots,v_s)$ -- некоторый набор векторов из $V$ и $u = (u_1,\ldots,u_t)$ -- набор векторов из $U$.
Тогда рассмотрим следующую конструкцию
\[
v^t\cdot_\beta u = 
\begin{pmatrix}
{v_1}\\{\vdots}\\{v_s}
\end{pmatrix}
\cdot_\beta
\begin{pmatrix}
{u_1}&{\ldots}&{u_t}
\end{pmatrix}
=
\begin{pmatrix}
{v_1\cdot_\beta u_1}&{\ldots}&{v_1\cdot_\beta u_t}\\
{\vdots}&{\ddots}&{\vdots}\\
{v_s\cdot_\beta u_1}&{\ldots}&{v_s\cdot_\beta u_t}\\
\end{pmatrix}
\]
То есть мы умножаем столбец из векторов из $V$ на строку из векторов из $U$ с помощью билинейной формы, рассматриваемой как оператор умножения.
Тогда результат будет матрица из $\operatorname{M}_{s\,t}(F)$.
Причем умножение происходит по тем же самым формальным правилам, что и обычное матричное умножение, только с использованием $\cdot_\beta$ вместо обычного умножения (которое для векторов даже не определено).

Тогда, если выбрать $e = (e_1,\ldots,e_n)$ -- базис $V$ и $f = (f_1,\ldots,f_m)$ -- базис $U$, то 
\[
B_\beta = e^t\cdot_\beta f = 
\begin{pmatrix}
{e_1}\\{\vdots}\\{e_n}
\end{pmatrix}
\cdot_\beta
\begin{pmatrix}
{f_1}&{\ldots}&{f_m}
\end{pmatrix}
\]
Мы привыкли, что в случае линейных отображений вычисления можно вести в удобной матричной форме.
Последнее равенство позволяет вычисления с билинейными формами сводить к матричной.

\begin{claim}
Пусть $V$ и $U$ -- векторные пространства над полем $F$, $e_1,\ldots,e_n \in V$ -- базис $V$ и $f_1,\ldots,f_m\in U$ -- базис $U$.
Тогда для любого набора чисел $b_{ij}\in F$, где $1\leqslant i \leqslant n$ и $1\leqslant j \leqslant m$, существует единственная билинейная форма $\beta\colon V\times U\to F$ такая, что $\beta(e_i,f_j) = b_{ij}$.
\end{claim}
\begin{proof}
По сути -- это переформулировка утверждения~\ref{claim::BilinearMatrices}.
\end{proof}

\begin{claim}
Пусть $\beta\colon V\times U\to F$ -- билинейная форма.
Пусть в пространстве $V$ зафиксировано два базиса $e=(e_1,\ldots,e_n)$ и $e' = (e_1',\ldots,e_n')$ с матрицей перехода $C\in \operatorname{M}_n(F)$ такой, что $e'=eC$, пусть в пространстве $U$ также зафиксированы два базиса $f = (f_1,\ldots,f_m)$ и $f'=(f_1',\ldots,f_m')$ с матрицей перехода $D\in \operatorname{M}_m(F)$ такой, что $f' = fD$.
Если $B_\beta$ -- матрица $\beta$ в базисах $e$ и $f$ и $B_\beta'$ -- матрица $\beta$ в базисах $e'$ и $f'$, тогда $B_\beta' = C^t B_\beta D$.
\end{claim}
\begin{proof}
Пользуясь только что введенным формализмом можно проделать следующие вычисления:%
\footnote{Обратите внимание, что тут у нас присутствует два умножения: матричное с числами и матричное с билинейной формой.
Порядок этих операций (то есть расстановка скобок) не важны, это следует просто из определения билинейной формы, если присмотреться внимательно.}
\[
B_\beta' = (e')^t\cdot_\beta f' = (eC)^t \cdot_\beta (f D) = (C^t e^t) \cdot_\beta (f D) = C^t(e^t \cdot_\beta f) D = C^t B_\beta D
\]
\end{proof}

\paragraph{Замечания}

\begin{itemize}
\item Заметим, что если билинейная форма определена на одном пространстве $\beta\colon V\times V\to F$, то достаточно выбрать один базис $e=(e_1,\ldots,e_n)$, после чего коэффициенты $B_\beta$ считаются по правилу $b_{ij} = \beta(e_i,e_j)$.
При этом, если $e'=(e_1',\ldots,e_n')$ -- другой базис и $e'=eC$, где $C$ -- матрица перехода, то $B_\beta' = C^t B C$.

\item Пусть у нас есть два векторных пространства $V$ и $U$.
Тогда на них могут жить два разного рода объектов: линейные отображения и билинейные формы, например, $\phi\colon V\to U$ и $\beta\colon U\times V \to F$.
Если мы выберем базис в $V$ и базис в $U$, то $V$ превращается в $F^n$, а $U$ -- в $F^m$.
В этом случае, линейное отображение $\phi$ описывается некоторой матрицей $A\in\operatorname{M}_{m\,n}(F)$, при этом $\phi(x) = Ax$.
С другой стороны, билинейная форма тоже описывается матрицей $B\in \operatorname{M}_{m\,n}(F)$, при этом $\beta(x,y) = x^tBy$.

Таким образом, для описания и линейных отображений и билинейных форм в фиксированном базисе мы используем матрицы (причем одного и того же размера).
Возникает вопрос: <<а как понять,  когда матрица задает линейное отображение, а когда билинейную форму?>> Если  нам выдали только одну пару базисов и матрицу $S$, то ответ простой -- никак.
В фиксированном базисе они не отличимы.
Мы можем считать нашу матрицу $S$ линейным оператором или билинейной формой, в зависимости от наших предпочтений.
Однако, если нам выдали несколько базисов, например два, и в этих базисах наш объект задается матрицами $S$ и $S'$.
То отличить оператор от билинейной формы можно по формуле перехода, а именно, если задан оператор, то $S' = C^{-1}S D$, а если билинейная форма, то $S' = C^t S D$.
Конечно, если базисы трепетно подобраны (врагом или другом -- это как повезет), то мы все равно можем не заметить разницы.
Но если мы будем сравнивать во всех возможных базисах, то ответ определяется однозначно.
\end{itemize}

