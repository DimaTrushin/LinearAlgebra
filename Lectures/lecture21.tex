\ProvidesFile{lecture21.tex}[Лекция 21]


\subsection{Жорданова нормальная форма для нильпотентных операторов}

\begin{definition}
\label{def::JNF}
Матрица 
\[
J_n(\lambda) = 
\begin{pmatrix}
{\lambda}&{1}&{}&{}&{}\\
{}&{\lambda}&{1}&{}&{}\\
{}&{}&{\ddots}&{\ddots}&{}\\
{}&{}&{}&{\lambda}&{1}\\
{}&{}&{}&{}&{\lambda}\\
\end{pmatrix}
\in \operatorname{M}_n(F)
\]
называется жордановой клеткой размера $n$.
Если матрица $A$ блочно диагональная, где на диагонали стоят жордановы клетки
\[
A = 
\begin{pmatrix}
{J_{n_1}(\lambda_1)}&{}&{}&{}\\
{}&{J_{n_2}(\lambda_2)}&{}&{}\\
{}&{}&{\ddots}&{}\\
{}&{}&{}&{J_{n_r}(\lambda_r)}\\
\end{pmatrix}
\in \operatorname{M}_{n_1 + \ldots + n_r}(F)
\]
то говорят, что $A$ имеет жорданову нормальную форму.
\end{definition}

Рассмотрим оператор $\phi\colon F^n \to F^n$, по правилу $x\mapsto J_n(0)x$.
Если $e_1, \ldots ,e_n$ -- стандартный базис в $F^n$, то мы видим, что 
\[
0\stackrel{\phi}{\longleftarrow}e_1\stackrel{\phi}{\longleftarrow}e_2 \stackrel{\phi}{\longleftarrow}\ldots\stackrel{\phi}{\longleftarrow}e_{n-1}\stackrel{\phi}{\longleftarrow}e_n
\]
Наоборот, пусть у нас нашелся базис для оператора $\phi\colon V\to V$ с таким свойством.
Тогда метод пристального взгляда нам подсказывает, что его матрица в этом базисе будет $J_n(0)$.
Таким образом мы описали на геометрическом языке как понять, что в некотором базисе матрица оператора задана жордановой клеткой с нулем на диагонали.

\begin{claim}
\label{claim::LinIndepModKer}
Пусть $\varphi\colon V\to V$ -- некоторый оператор и $v_1,\ldots,v_r\in \ker \varphi^{k+1}$ линейно независимы по модулю $\ker \varphi^k$.
Тогда векторы $\varphi v_1,\ldots,\varphi v_r$ лежат в $\ker \varphi^k$ и при $k \geqslant 1$%
\footnote{При $k = 0$ все векторы $\varphi v_i = 0$.
Формально доказательство не сработает, так как при $k = 0$ не определено $\ker \varphi^{0 - 1}$.}
они линейно независимы по модулю $\ker \varphi^{k-1}$.
\end{claim}
\begin{proof}
Тот факт что $\varphi v_i\in \ker \varphi^{k}$ следует из того, что они зануляются оператором $\varphi^k$.
Теперь покажем, что они линейно независимы по модулю $\ker \varphi^{k-1}$.
Пусть 
\[
\alpha_1 \varphi v_1 + \ldots + \alpha_r \varphi v_r = 0 \pmod{\ker \varphi^{k-1}}
\]
Тогда
\begin{gather*}
\varphi(\alpha_1 v_1 + \ldots + \alpha_r v_r)\in \ker \varphi^{k-1}\Rightarrow\\
\varphi^k(\alpha_1 v_1 + \ldots + \alpha_r v_r) = 0\Rightarrow\\
\alpha_1v_1 +\ldots + \alpha_r v_r \in \ker\varphi^k\Rightarrow\\
\alpha_1 v_1 + \ldots + \alpha_r v_r = 0 \pmod{\ker \varphi^k}
\end{gather*}
Значит все $\alpha_i = 0$, что и требовалось.
\end{proof}

\begin{claim}
[ЖНФ для нильпотентов]
\label{claim::NilJNF}
Пусть $\varphi\colon V\to V$ -- нильпотентный оператор.
Тогда
\begin{enumerate}
\item Для оператора $\varphi$ существует жорданов базис, то есть в некотором базисе матрица $\varphi$ имеет вид
\[
A_\varphi = 
\begin{pmatrix}
{J_{n_1}(0)}&{}&{}&{}\\
{}&{J_{n_2}(0)}&{}&{}\\
{}&{}&{}&{}\\
{}&{}&{}&{J_{n_r}(0)}\\
\end{pmatrix}
\]

\item Количество клеток размера $r$ вычисляется по формуле
\[
2\dim \ker \varphi^r - \dim \ker \varphi^{r+1} - \dim \ker \varphi^{r-1}
\]
Так как эти числа не зависят от базиса, то в любом жордановом базисе количество клеток размера $r$ одинаковое.
А значит жордановы формы в разных базисах могут отличаться лишь перестановкой клеток.
\end{enumerate}
\end{claim}
\begin{proof}
(1) Пусть $\varphi^k = 0$, причем $k$ -- наименьшее возможное.
Для того, чтобы доказать теорему, мне надо найти базис в пространстве $V = \ker \varphi^k$, состоящий из цепочек вида:
\[
0\stackrel{\varphi}{\longleftarrow}e_1\stackrel{\varphi}{\longleftarrow}e_2 \stackrel{\varphi}{\longleftarrow}\ldots\stackrel{\varphi}{\longleftarrow}e_{n-1}\stackrel{\varphi}{\longleftarrow}e_n
\]
Каждая такая цепочка будет давать одну клетку размера $n$.
Так как вектор $e_n$ в такой цепочке имеет высоту $n$ его надо искать в $\ker \varphi^{n}\setminus\ker\varphi^{n-1}$.
Значит, чтобы получить самые длинные цепочки я должен как-то выбрать векторы $v_1,\ldots, v_r$ в $\ker\varphi^k \setminus \ker \varphi^{k-1}$.
При этом, я хочу, чтобы все векторы вида $\varphi^i v_j$ были между собой линейно независимы.
То есть выбирать надо аккуратно.
Для этого мне и понадобится понятие линейной независимости по модулю подпространства.
Итак, приступим.

Возьмем $v_1,\ldots,v_{r_1}\in\ker\varphi^k$ -- базис $\ker \varphi^k$ по модулю $\ker \varphi^{k-1}$.
Последнее означает, что $\ker \varphi^k = \langle v_1,\ldots,v_{r_1}\rangle \oplus \ker \varphi^{k-1}$.
Из утверждения~\ref{claim::LinIndepModKer} следует, что векторы $\varphi v_1,\ldots,\varphi v_{r_1}$ лежат в $\ker \varphi^{k-1}$ и линейно независимы по модулю $\ker \varphi^{k-2}$.%
\footnote{Либо, что $k = 1$, то есть $V = \ker \varphi$, а значит $\varphi = 0$ и доказывать нечего.}
Значит их можно дополнить до базиса пространства $\ker \varphi^{k-1}$ по модулю подпространства $\varphi^{k-2}$ векторами $v_{r_1+1},\ldots,v_{r_2}$.
Данный процесс можно изобразить на следующей диаграмме:
\[
\xymatrix@R=15pt@C=15pt{
  {\ker \varphi^k}&
  {v_1}\ar@{|->}[d]&{\ldots}&{v_{r_1}}\ar@{|->}[d]&
  {}&{}&{}\\
  {\ker \varphi^{k-1}}\ar@{}[u]|{\cup}&
  {\varphi v_1}&{\ldots}&{\varphi v_{r_1}}&
  {v_{r_1 + 1}}&{\ldots}&{v_{r_2}}\\
}
\]
Кроме того, проделанное означает, что 
\begin{gather*}
\ker \varphi^k = \langle v_1,\ldots, v_{r_1}\rangle \oplus \ker \varphi^{k-1}\\
\ker \varphi^{k-1} = \langle \varphi v_1,\ldots, \varphi v_{r_1}\rangle \oplus \langle v_{r_1 + 1},\ldots,v_{r_2}\rangle \oplus \ker \varphi^{k-2}
\end{gather*}
Мы можем продолжать этот процесс далее.
Он остановится, когда мы дойдем до $\ker \varphi$, так как следующее подпространство будет уже нулевым.
Весь процесс можно изобразить на следующей диаграмме (здесь я векторы обозначил точками, чтобы не загромождать обозначения):
\[
\xymatrix@R=15pt@C=15pt{
  {\ker\varphi^k}&{\bullet}\ar[d]
  {\save
   [].[rrr]*+[F--]\frm{}
  \restore}
  &
  {\bullet}\ar[d]&{\ldots}&{\bullet}\ar[d]&
  {}&{}&{}&
  {}&
  {}&{}&{}&
  {}&{}&{}\\
  {\ker\varphi^{k-1}}\ar@{}[u]|{\cup}&{\bullet}\ar[d]
  {\save
   [].[rrrrrr]*+[F--]\frm{}
  \restore}
  &
  {\bullet}\ar[d]&{\ldots}&{\bullet}\ar[d]&
  {\bullet}\ar[d]
  &{\ldots}&{\bullet}\ar[d]&
  {}&
  {}&{}&{}&
  {}&{}&{}\\
  {\vdots}\ar@{}[u]|{\cup}&
  {\vdots}\ar[d]&{\vdots}\ar[d]&{\vdots}&{\vdots}\ar[d]&
  {\vdots}\ar[d]&{\vdots}&{\vdots}\ar[d]&
  {\ddots}&
  {}&{}&{}&
  {}&{}&{}\\
  {\ker\varphi^2}\ar@{}[u]|{\cup}&{\bullet}\ar[d]
  {\save
   [].[rrrrrrrrrr]*+[F--]\frm{}
  \restore}
  &
  {\bullet}\ar[d]&{\ldots}&{\bullet}\ar[d]&
  {\bullet}\ar[d]&{\ldots}&{\bullet}\ar[d]&
  {\ldots}&
  {\bullet}\ar[d]
  &{\ldots}&{\bullet}\ar[d]&
  {}&{}&{}\\
  {\ker\varphi}\ar@{}[u]|{\cup}&{\bullet}\ar[d]
  {\save
   [].[rrrrrrrrrrrrr]*+[F--]\frm{}
  \restore}
  &
  {\bullet}\ar[d]&{\ldots}&{\bullet}\ar[d]&
  {\bullet}\ar[d]&{\ldots}&{\bullet}\ar[d]&
  {\ldots}&
  {\bullet}\ar[d]&{\ldots}&{\bullet}\ar[d]&
  {\bullet}\ar[d]
  &{\ldots}&{\bullet}\ar[d]\\
  {0}\ar@{}[u]|{\cup}&
  {0}&{0}&{\ldots}&{0}&
  {0}&{\ldots}&{0}&
  {\ldots}&
  {0}&{\ldots}&{0}&
  {0}&{\ldots}&{0}\\
}
\]
Кроме того, мы будем иметь равенства вида:
\[
\ker \varphi^{k-s} = \langle \varphi^s v_1,\ldots,\varphi^s v_{r_1}\rangle \oplus \langle \varphi^{s-1}v_{r_1 + 1},\ldots,\varphi^{s-1}v_{r_2}\rangle \oplus \ldots \oplus \langle v_{r_s + 1},\ldots,v_{r_{s+1}} \rangle \oplus \ker \varphi^{k - s -1}
\]
То есть все векторы расположенные в заштрихованных прямоугольниках на диаграмме выше являются линейно независимыми между собой и со всеми векторами, которые лежат ниже них.
Значит все построенные вектора (точки на диаграмме выше) являются базисом пространства $\ker \varphi^k = V$.
А это то, что и надо было сделать.

(2) Теперь нам надо доказать формулу для количества клеток.
Для этого предлагается сделать так: выберем произвольную жорданову нормальную форму для оператора, для нее посчитаем количество клеток фиксированного размера и поймем, что оно задается нужной формулой.
Пусть $e_1,\ldots,e_n$ -- жорданов базис и пусть матрица оператора имеет вид
\[
A =
\begin{pmatrix}
{J_{k_1}(0)}&{}&{}&{}\\
{}&{J_{k_2}(0)}&{}&{}\\
{}&{}&{\ddots}&{}\\
{}&{}&{}&{J_{k_s}(0)}\\
\end{pmatrix}
\]
Переставив базисные векторы, мы можем считать, что $k_1 \geqslant k_2 \geqslant\ldots\geqslant k_s$.
Тогда мы можем расположить базисные векторы в виде диаграммы
\[
\xymatrix@R=15pt@C=15pt{
  {e_k}\ar[d]
  {\save
   [].[rrr]*[F-:<3pt>]\frm{}
  \restore}
  {\save
   [].[rrr]*+[F--]\frm{}
  \restore}&
  {\bullet}\ar[d]&{\ldots}&{\bullet}\ar[d]&
  {}&{}&{}&
  {}&
  {}&{}&{}&
  {}&{}&{}\\
  {e_{k-1}}\ar[d]
  {\save
   [].[rrrrrr]*+[F--]\frm{}
  \restore}&
  {\bullet}\ar[d]&{\ldots}&{\bullet}\ar[d]&
  {\bullet}\ar[d]
  {\save
   [].[rr]*[F-:<3pt>]\frm{}
  \restore}&{\ldots}&{\bullet}\ar[d]&
  {}&
  {}&{}&{}&
  {}&{}&{}\\
  {\vdots}\ar[d]&{\vdots}\ar[d]&{\vdots}&{\vdots}\ar[d]&
  {\vdots}\ar[d]&{\vdots}&{\vdots}\ar[d]&
  {\ddots}&
  {}&{}&{}&
  {}&{}&{}\\
  {e_2}\ar[d]
  {\save
   [].[rrrrrrrrrr]*+[F--]\frm{}
  \restore}&
  {\bullet}\ar[d]&{\ldots}&{\bullet}\ar[d]&
  {\bullet}\ar[d]&{\ldots}&{\bullet}\ar[d]&
  {\ldots}&
  {\bullet}\ar[d]
  {\save
   [].[rr]*[F-:<3pt>]\frm{}
  \restore}&{\ldots}&{\bullet}\ar[d]&
  {}&{}&{}\\
  {e_1}\ar[d]
  {\save
   [].[rrrrrrrrrrrrr]*+[F--]\frm{}
  \restore}&
  {\bullet}\ar[d]&{\ldots}&{\bullet}\ar[d]&
  {\bullet}\ar[d]&{\ldots}&{\bullet}\ar[d]&
  {\ldots}&
  {\bullet}\ar[d]&{\ldots}&{\bullet}\ar[d]&
  {\bullet}\ar[d]
  {\save
   [].[rr]*[F-:<3pt>]\frm{}
  \restore}&{\ldots}&{e_n}\ar[d]\\
  {0}&{0}&{\ldots}&{0}&
  {0}&{\ldots}&{0}&
  {\ldots}&
  {0}&{\ldots}&{0}&
  {0}&{\ldots}&{0}\\
}
\]
где мы ставим вектора $e_1, e_2, \ldots, e_n$ снизу вверх и слева направо, а стрелочки означают применение оператора $\varphi$.
Теперь мы знаем, что все эти векторы линейно независимы, но вообще говоря не понятно, как они связаны с ядрами.
Давайте покажем, что нижние $r$ слоев дают базис $\ker \varphi^r$.
Действительно, рассмотрим произвольный вектор $v = a_1 e_1 + \ldots + a_n e_n$ из $\ker \varphi^r$.
Это значит, что $\varphi^r (v) = 0$.
Давайте мысленно расставим коэффициенты $a_i$ в диаграмме выше рядом с соответствующим $e_i$.
При применении $\varphi$  каждый базисный вектор спустится на ярус ниже, а вектора нижнего яруса занулятся.
Это значит, что под действием $\varphi^r$ нижние $r$ ярусов занулсятся, а вектора с ярусов $r+1, \ldots, k$ перейдут на ярусы $1,\ldots, k-r$.
Давайте изобразим это безобразие на следующем примере.%
\footnote{Здесь имеется в виду, что надо взять сумму всех указанных на диаграмме слагаемых.
Слева стоит исходная линейная комбинация $v = a_1 e_1 + \ldots + a_{13}e_{13}$, а справа результат применения к нему $\varphi^2$, то есть $\varphi^2v$.}
\[
\begin{aligned}[c]
\xymatrix@R=15pt@C=10pt{
	{a_4e_4}\ar[d]&{a_8e_8}\ar[d]&{}&{}&{}\\
	{a_3e_3}\ar[d]&{a_7e_7}\ar[d]&{a_{10}e_{10}}\ar[d]&{}&{}\\
	{a_2e_2}\ar[d]&{a_6e_6}\ar[d]&{a_9e_9}\ar[d]&{a_{12}e_{12}}\ar[d]&{}\\
	{a_1e_1}\ar[d]&{a_5e_5}\ar[d]&{a_8e_8}\ar[d]&{a_{11}e_{11}}\ar[d]&{a_{13}e_{13}}\ar[d]\\
  {0}&{0}&{0}&{0}&{0}
}
\end{aligned}
\quad\stackrel{\varphi^2}{\longrightarrow}\quad
\begin{aligned}[c]
\xymatrix@R=15pt@C=10pt{
	{0e_4}\ar[d]&{0e_8}\ar[d]&{}&{}&{}\\
	{0e_3}\ar[d]&{0e_7}\ar[d]&{0e_{10}}\ar[d]&{}&{}\\
	{a_4e_2}\ar[d]&{a_8e_6}\ar[d]&{0e_9}\ar[d]&{0e_{12}}\ar[d]&{}\\
	{a_3e_1}\ar[d]&{a_7e_5}\ar[d]&{a_{10}e_8}\ar[d]&{0e_{11}}\ar[d]&{0e_{13}}\ar[d]\\
  {0}&{0}&{0}&{0}&{0}
}
\end{aligned}
\]
Как мы видим, остаются слагаемые вида коэффициент на какой-то базисный вектор.
Чтобы результат был нулем, надо чтобы все кооэффициенты оставшиеся справа на диаграмме были нулевые.
В данном примере, это означает, что на левой диаграмме верхние два слоя были нулями, то есть линейная комбинация принадлежит нижним двум слоям.
В общем случае ситуация такая же, ядро $\varphi^r$ будет порождено первыми $r$ слоями снизу.

Теперь нам надо посчитать количество клеток размера $r$ в жордановой форме.
Это соответствует тому, чтобы посчитать количество цепочек длины $r$ на большой диаграмме выше.
То есть нам надо посчитать количество векторов обведенных в овальную рамку в $r$-ой строке.
В начале посчитаем количество векторов в заштрихованной рамке на каждом этаже.
Так как $\ker \varphi^r$ порожден всеми векторами на слое $r$ и ниже, то в $r$-ой строке количество векторов в заштрихованной рамке равно $\dim \ker \varphi^r - \dim \ker \varphi^{r-1}$.
Тогда количество векторов в овальной рамке на этаже $r$ равно количество векторов в заштрихованной рамке на этаже $r$ минус количество векторов в заштрихованной рамке на этаже $r+1$.
Значит, искомое количество клеток размера $r$ равно:
\[
(\dim \ker \varphi^r - \dim \ker \varphi^{r-1}) - (\dim \ker \varphi^{r+1} - \dim \ker \varphi^r) = 2 \dim \ker \varphi^r - \dim \ker \varphi^{r+1} - \dim \ker \varphi^{r-1}
\]
\end{proof}

\paragraph{Замечания}

Отметим специальный вид для количества клеток максимального и минимального размеров и сделаем еще пару замечаний.
\begin{itemize}
\item Максимальный размер $r = k$.
Тогда как мы видим, количество клеток равно
\[
\dim \ker \varphi^k - \dim \ker \varphi^{k-1} = 
\dim V - \dim \ker \varphi^{k-1} = \dim \Im\varphi^{k-1}
\]
То есть ранг последней ненулевой степени оператора $\varphi$ -- это количество клеток максимальной размерности.

\item Минимальный размер $r = 1$.
Тогда $\ker \varphi^{r - 1} = 0$.
Значит, количество клеток размера $1$, то есть, количество отдельно стоящих нулей в жордановой форме будет
\[
2 \dim \ker \varphi - \dim \ker \varphi^{2} 
\]
Обратите внимание на то, что это НЕ размерность ядра.

\item Размерность ядра $\dim \ker \varphi$ -- это количество всех клеток всевозможных размеров.

\item Максимальный размер клетки -- это степень минимального многочлена для $\varphi$ или что то же самое -- кратность его единственного корня $0$.
\end{itemize}

\subsection{Теорема о жордановой нормальной форме}

\begin{claim}
[Теорема о жордановой нормальной форме]
\label{claim::JNF}
Пусть $\varphi\colon V\to V$ -- линейный оператор такой, что его характеристический (или минимальный) многочлен раскладывается на линейные множители
\[
\chi_\varphi(t) = (t - \lambda_1)^{n_1} \ldots (t - \lambda_r)^{n_r}
\]
Тогда
\begin{enumerate}
\item Для оператора $\varphi$ существует жорданов базис, то есть в некотором базисе матрица $\varphi$ имеет вид
\[
A_\varphi = 
\begin{pmatrix}
{J_{k_1}(\lambda_{i_1})}&{}&{}&{}\\
{}&{J_{k_2}(\lambda_{i_2})}&{}&{}\\
{}&{}&{}&{}\\
{}&{}&{}&{J_{k_s}(\lambda_{i_s})}\\
\end{pmatrix}
\]

\item В любом жордановом базисе количество клеток размера $m$ с фиксированным числом $\lambda$ на диагонали одинаковое и равно
\[
2\dim \ker (\varphi - \lambda\Identity)^m - \dim \ker (\varphi - \lambda\Identity)^{m+1} - \dim \ker (\varphi - \lambda\Identity)^{m-1}
\]
А значит жордановы формы в разных базисах могут отличаться лишь перестановкой клеток.
\end{enumerate}
\end{claim}
\begin{proof}
(1) Так как $\chi_\varphi(t)$ (или минимальный многочлен) раскладывается на линейные множители, утверждение~\ref{claim::RootSpaceDec} говорит, что $V = V^{\lambda_1}\oplus \ldots\oplus V^{\lambda_r}$.
Тогда, если мы выберем базисы в подпространствах $V^{\lambda_i}$ объединим (они обязательно дадут базис $V$) и запишем в этом базисе матрицу $\varphi$, она будет иметь блочно диагональный вид
\[
A_\varphi = 
\begin{pmatrix}
{A_1}&{}&{}&{}\\
{}&{A_2}&{}&{}\\
{}&{}&{\ddots}&{}\\
{}&{}&{}&{A_r}\\
\end{pmatrix}
\]
где $A_i$ -- матрица $\varphi|_{V^{\lambda_i}}$.
То есть, чтобы доказать теорему, нам надо в каждом $V^{\lambda_i}$ выбрать жорданов базис для оператора $\varphi|_{V^{\lambda_i}}$.
Теперь заметим, что базис является жордановым для некоторого оператора $\phi$ тогда и только тогда, когда он является жордановым для оператора $\phi - \lambda \Identity$ (при любом выборе $\lambda$).
Потому нам надо в каждом $V^{\lambda_i}$ выбрать жорданов базис для оператора $\phi_i := \varphi|_{V^{\lambda_i}} - \lambda_i \Identity$.
Но оператор $\phi_i$ является нильпотентным и для него это следует из утверждения~\ref{claim::NilJNF}.

(2) Пусть теперь у нас выбран какой-нибудь жорданов базис, в котором матрица $\varphi$ имеет вид
\[
A_\varphi = 
\begin{pmatrix}
{A_1}&{}&{}&{}\\
{}&{A_2}&{}&{}\\
{}&{}&{\ddots}&{}\\
{}&{}&{}&{A_r}\\
\end{pmatrix},
\quad\text{где}\quad
A_i = 
\begin{pmatrix}
{J_{k_{1\,i}}(\lambda_i)}&{}&{}&{}\\
{}&{J_{k_{2\,i}}(\lambda_i)}&{}&{}\\
{}&{}&{}&{}\\
{}&{}&{}&{J_{k_{m_i\,i}}(\lambda_i)}\\
\end{pmatrix}
\]
Во-первых, числа $\lambda_i$ на диагоналях клеток будут обязательно числами из спектра, просто потому что $A_\varphi$ верхнетреугольная с этими числами на диагонали.

Во-вторых, нам надо показать, что все $A_i$ (где $A_i$ -- это блоки в которых мы сгруппировали клетки с одним и тем же числом $\lambda_i$ на диагонали) имеют одинаковый размер.
Но по определению размер этих блоков -- это кратность $\lambda_i$ в $\chi_\varphi(t)$ или что то же самое -- размерность $V^{\lambda_i}$.
А сам блок $A_i$ оказывается матрицей оператора $\varphi|_{V^{\lambda_i}}$.

В-третьих, надо показать, что внутри каждого $A_i$ количество блоков фиксированного размера одинаковое  и задано формулой 
\[
2\dim \ker (\varphi - \lambda_i\Identity)^m - \dim \ker (\varphi - \lambda_i\Identity)^{m+1} - \dim \ker (\varphi - \lambda_i\Identity)^{m-1}
\]
Но так как $A_i$ -- это матрица оператора $\varphi|_{V^{\lambda_i}}$, а оператор $\varphi|_{V^{\lambda_i}} - \lambda_i \Identity$ нильпотентен и имеет те же размеры блоков, то из пункта~(2) утверждения~\ref{claim::NilJNF} следует, что нужное количество клеток задано формулой
\[
2\dim \ker (\varphi|_{V^{\lambda_i}} - \lambda_i\Identity)^m - \dim \ker (\varphi|_{V^{\lambda_i}} - \lambda_i\Identity)^{m+1} - \dim \ker (\varphi|_{V^{\lambda_i}} - \lambda_i\Identity)^{m-1}
\]
Теперь осталось показать, что 
\[
\ker (\varphi - \lambda_i\Identity)^m = \ker (\varphi|_{V^{\lambda_i}} - \lambda_i\Identity)^m
\]
Если вспомнить определение оператора ограничения мы видим, что
\[
\ker (\varphi|_{V^{\lambda_i}} - \lambda_i\Identity)^m = V^{\lambda_i}\cap \ker (\varphi - \lambda_i\Identity)^m
\]
С другой стороны, $\ker (\varphi - \lambda_i\Identity)^m$ является подпространством $V^{\lambda_i}$ по определению, что и доказывает нужное равенство.
\end{proof}

Обратите внимание, что жорданова форма для оператора единственная с точностью до перестановки блоков.
Однако, жордановых базисов может быть много!
Например, если $\varphi = \lambda \Identity$, то любой базис является жордановым, так как в любом базисе матрица оператора будет диагональной с числом $\lambda$ на диагонали.

\paragraph{Замечания}

Давайте обсудим некоторые характеристики жордановых клеток в терминах исходного оператора.
\begin{itemize}
\item Число $\dim V^{\lambda_i}$ является суммарным размером всех клеток с заданным $\lambda_i$, то есть клеток вида $J_s(\lambda_i)$.
Действительно, по построению, блок из таких клеток возникает как матрица $\varphi|_{V^{\lambda_i}}$.
А размер матрицы оператора равен размерности пространства, на котором определен оператор.%
\footnote{Можно объяснить по-другому из явного вычисления с матрицей жордановой нормальной формы аналогично следующему пункту.}

\item Число $\dim V_{\lambda_i}$ является количеством жордановых клеток с фиксированным $\lambda_i$, то есть клеток вида $J_s(\lambda_i)$.
Действительно, по определению $V_{\lambda_i} = \ker (\varphi - \lambda_i \Identity)$.
Теперь надо посчитать правую часть равенства в жордановой форме и увидеть, что его размерность равна суммарному размеру клеток с числом $\lambda_i$.
Пусть $A$ имеет жорданову форму как в предыдущем утверждении, тогда
\[
A -\lambda_i E= 
\begin{pmatrix}
{A_1-\lambda_i E}&{}&{}\\
{}&{\ddots}&{}\\
{}&{}&{A_r - \lambda_i E}
\end{pmatrix},\text{ где }
A_k -\lambda_i E=
\begin{pmatrix}
{J_*(\lambda_k - \lambda_i)}&{}&{}\\
{}&{\ddots}&{}\\
{}&{}&{J_*(\lambda_k - \lambda_i)}
\end{pmatrix}
\]
То есть для $k\neq i$ все блоки будут верхнетреугольными с ненулевым числом на диагонали, а значит обратимыми, а для $k = i$ блок будет вида
\[
A_i -\lambda_i E=
\begin{pmatrix}
{J_*(0)}&{}&{}\\
{}&{\ddots}&{}\\
{}&{}&{J_*(0)}
\end{pmatrix}
\]
Пусть $A$ размера $n$, а $n_k$ -- размер блока $A_k$.
Тогда размерность $\ker (A - \lambda_i E)$ равна $n - \rk (A - \lambda_i E)$.
Но $\rk (A - \lambda_i E) = \sum_k \rk (A_k - \lambda_i E)$ и $n = \sum_k n_k$.
Так как все матрицы $A_k - \lambda_i E$ для $k\neq i$ обратимы, то есть имеют полный ранг, то $n - \rk (A-\lambda_i E)$ равно $n_i - \rk (A_i - \lambda_iE)$.
Каждый блок $J_s(0)$ имеет ранг $s-1$.
То есть каждая клетка вносит в ранг вклад на единицу меньше размера.
Следовательно $\rk(A_i -\lambda_i E)$ равно $n$ минус количество клеток, победа!

\item Кратность $\lambda_i$ в $f_\text{min}$ равна размерности самого большого блока вида $J_s(\lambda_i)$.
Давайте посчитаем кратность корня $\lambda_i$ в $f_\text{min}$ для $A$ в жордановой форме.
Для этого надо найти $m$ для которого наступит $\ker (A - \lambda_i E)^m = \ker(A-\lambda_i E)^{m+1}$ (утверждение~\ref{claim::RootMultGeom}).
Как и выше, блоки $A_k - \lambda_i E$ не дают вклад в ядро.
А блок $A_i - \lambda_i E$ зануляется в степени равной максимальному размеру клетки.
То есть это ровна степень стабилизации, что и требовалось.

\item Мы уже доказали, что количество клеток размера $r$ с числом $\lambda$ вычисляется по формуле
\[
2\dim \ker (\varphi - \lambda\Identity)^r - \dim \ker (\varphi - \lambda\Identity)^{r+1} - \dim \ker (\varphi - \lambda\Identity)^{r-1}
\]
Однако, можно дать совершенно другое доказательство этого факта, которое может оказаться для вас более приятным или более понятным.
Вместо всех этих дурацких рассуждений с операторами, которые были приведены во второй части предыдущего утверждения, можно поступить вот как.
Пусть мы уже привели матрицу в ЖНФ в каком-то базисе (но пока еще не знаем ее единственности).
\[
A_\varphi = 
\begin{pmatrix}
{A_1}&{}&{}&{}\\
{}&{A_2}&{}&{}\\
{}&{}&{\ddots}&{}\\
{}&{}&{}&{A_r}\\
\end{pmatrix},
\quad\text{где}\quad
A_i = 
\begin{pmatrix}
{J_{k_{1\,i}}(\lambda_i)}&{}&{}&{}\\
{}&{J_{k_{2\,i}}(\lambda_i)}&{}&{}\\
{}&{}&{}&{}\\
{}&{}&{}&{J_{k_{m_i\,i}}(\lambda_i)}\\
\end{pmatrix}
\]
А давайте для данной ЖНФ просто посчитаем число 
\[
2\dim \ker (\varphi - \lambda\Identity)^r - \dim \ker (\varphi - \lambda\Identity)^{r+1} - \dim \ker (\varphi - \lambda\Identity)^{r-1}
\]
То есть мы будем считать
\[
2\dim \ker (A_\varphi - \lambda E)^r - \dim \ker (A_\varphi - \lambda E)^{r+1} - \dim \ker (A_\varphi - \lambda E)^{r-1}
\]
Проделаем это аналогично тому, как в одном из замечаний выше и увидим, что это число дает количество клеток размера $r$ для данной ЖНФ.%
\footnote{Попробуйте довести это рассуждение до конца, это очень полезно и просто.}
Но с другой стороны, это число не зависит от ЖНФ, значит для любой ЖНФ число клеток считается по этой формуле.
\end{itemize}
