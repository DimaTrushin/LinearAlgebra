\ProvidesFile{lecture22.tex}[Лекция 22]


\paragraph{Функториальность звездочки}

Обратите внимание, что мы теперь построили очень любопытный математический агрегат.
А именно, пусть у нас есть мешок всех векторных пространств и линейных отображений между ними.
Тогда для каждого векторного пространства $V$ мы можем построить новое векторное пространство $V^*$.
Кроме того, мы умеем действовать не только на векторных пространствах, но и на отображениях между ними.
То есть каждому отображению $\varphi\colon V\to U$ мы ставим в соответствие $\varphi^*\colon U^*\to V^*$.
Про это надо думать так: все векторные пространства образуют (охренительно огроменнейший) граф, у которого вершины -- векторные пространства, а ребра -- линейные отображения.
Мы построили отображение из этого графа в себя, которое разворачивает стрелки.
Кроме того, это отображение согласовано с композицией в следующем смысле:%
\footnote{Я не хочу вводить формальные определения, но мы реально только что построили контровариантный функтор из категории векторных пространств в себя, что бы это ни значило.}
\begin{itemize}
\item Если $\Identity\colon V\to V$, тогда $\Identity^*\colon V^*\to V^*$ является тождественным отображением, то есть $\Identity^* = \Identity$.

\item Если $\varphi\colon V\to U$ и $\psi\colon U\to W$, тогда $(\psi\varphi)^* = \varphi^*\psi^*$, то есть звездочка меняет местами порядок отображений.
Графически это можно изобразить так:
\[
\text{Если коммутативна диаграмма }
\xymatrix@R=6pt{
	{V}\ar[r]^-{\varphi}\ar@{-->}@/^16pt/[rr]^{\psi\circ \varphi}&{U}\ar[r]^-{\psi}&{W,}\\
}
\text{ то коммутативна диаграмма }
\xymatrix@R=6pt{
	{V^*}&{U^*}\ar[l]_-{\varphi^*}&{W^*}\ar[l]_-{\psi^*}\ar@{-->}@/_16pt/[ll]_{(\psi\circ \varphi)^*}\\
}
\]
Коммутативность диаграммы, означает, что любые два пути ведущие из одной вершины в другую приводят к одному результату.

\item Так же отметим, что звездочка согласована со структурой векторного пространства.
А именно для любых $\varphi, \psi \colon V\to U$ и числе $\alpha, \beta\in F$ верно $(\alpha \varphi + \beta \psi)^* = \alpha \varphi^* + \beta\psi^* \colon U^* \to V^*$.


\end{itemize}



\paragraph{Канонический изоморфизм}

\begin{claim}
\label{claim::CanonicalIsomorphism}
Пусть $\varphi\colon V\to U$ -- линейное отображение и $\phi\colon V\to V^{**}$ и $\phi\colon U\to U^{**}$ -- канонические изоморфизмы на второе сопряженное.
В этом случае коммутативна следующая диаграмма:
\[
\xymatrix{
	{V}\ar[r]^{\varphi}\ar[d]^{\phi}&{U}\ar[d]^{\phi}\\
	{V^{**}}\ar[r]^{\varphi^{**}}&{U^{**}}
}
\]
то есть $\phi \varphi = \varphi^{**}\phi$.
\end{claim}
\begin{proof}
Давайте распишем, как вектор $v\in V$ двигается по этой диаграмме
\[
\xymatrix{
	{v}\ar@{|->}[rr]\ar@{|->}[d]&{}&{\varphi(v)}\ar@{|->}[d]\\
	{\phi_v}\ar@{|->}[r]&{\phi_v\circ \varphi^*}\ar@{=}[r]^{?}&{\phi_{\varphi(v)}}
}
\]
И нам надо проверить равенство с вопросиком.
Для этого надо сравнить действие левой и правой части на произвольном $\xi\in U^*$.
Получим
\[
(\phi_v\circ \varphi^*)(\xi) = \phi_v(\varphi^*(\xi)) = \phi_v(\xi\circ\varphi) = (\xi\circ \varphi) (v) = \xi(\varphi(v)) = \phi_{\varphi(v)}(\xi)
\]
что и требовалось.
\end{proof}

\paragraph{Замечания}

\begin{itemize}
\item Последнее утверждение показывает, что изоморфизмы $\phi\colon V\to V^{**}$ согласованы с линейными отображениями в том смысле, что коммутативны некоторые диаграммы.
Как надо думать про это?
Смысл $\varphi$ в том, что мы можем считать, что $V$ и $V^{**}$ -- это одно и то же.
Согласованность с отображениями означает, что при таком отождествлении $V$ с $V^{**}$ и $U$ с $U^{**}$ отображение $\varphi$ превращается в $\varphi^{**}$.

\item С другой стороны, если мы захотим потребовать нечто подобное для $V$ и $V^*$ то мы получим диаграмму вида
\[
\xymatrix{
	{V}\ar[r]^{\varphi}\ar[d]^{\phi}&{U}\ar[d]^{\phi}\\
	{V^*}&{U^*}\ar[l]^{\varphi^*}
}
\]
Причем эта диаграмма должна быть коммутативной (то есть $\phi = \varphi^*\phi\varphi$) для всех отображений $\varphi\colon V\to U$, а вертикальные стрелки $\phi$ все должны быть изоморфизмами.
Но выберем тогда в качестве $\varphi$ нулевое отображение и получим, что $\phi = 0$, противоречие.
\end{itemize}


\newpage
\section{Тензоры}

Всем хороши векторные пространства.
Их элементы можно складывать и умножать на числа.
Одна беда -- векторы нельзя перемножать.
А очень хочется.
А мы знаем, что когда нельзя, но очень хочется, то можно, а скорее даже нужно.
Наша следующая задача научиться перемножать векторы.
Окажется, что способов перемножать векторы много и результаты этих перемножений могут лежать где угодно.
Однако, среди всех этих способов можно найти самую лучшую операцию перемножения и для нее построить особое пространство, в котором будут лежать произведения векторов.
Вот это новое замечательное пространство будет называться тензорным произведением исходных векторных пространств.
Чтобы начать разговор о тензорных произведениях, надо немного поговорить об операциях умножения или как более прилично говорить в математике -- билинейных отображениях.

\subsection{Билинейные отображения}

\begin{definition}
Пусть $V$, $U$ и $T$ -- векторные пространства над полем $F$.
Отображение $\mu\colon V\times U \to T$ называется билинейным, если выполняется следующий набор аксиом
\begin{enumerate}
\item Для любых векторов $v_1,v_2\in V$ и $u\in U$ выполнено
\[
\mu(v_1 + v_2, u) = \mu(v_1, u) + \mu(v_2, u)
\]

\item Для любых векторов $v\in V$, $u\in U$ и чисел $\lambda \in F$ выполнено
\[
\mu(\lambda v, u) = \lambda \mu(v, u)
\]

\item Для любых векторов $v\in V$ и $u_1,u_2\in U$ выполнено
\[
\mu(v, u_1 + u_2) = \mu(v, u_1) + \mu(v, u_2)
\]

\item Для любых векторов $v\in V$, $u\in U$ и чисел $\lambda \in F$ выполнено
\[
\mu(v, \lambda u) = \lambda \mu(v, u)
\]
\end{enumerate}
\end{definition}

Обратите внимание, что отображение $\mu$ -- это функция двух аргументов.
А у любой функции двух аргументов есть операторная запись.%
\footnote{Операторная и функциональная запись функций с двумя аргументами многим должна быть знакома по языкам программирования.}
Например, на запись $\mu(v, u) $ можно смотреть как на $v * u$.
В этом случае определение выше можно переписать в эквивалентной форме, которая психологически кому-то может быть более близка.
\begin{definition}
Пусть $V$, $U$ и $T$ -- векторные пространства над полем $F$.
Отображение $*\colon V\times U \to T$ называется билинейным (или умножением), если выполняется следующий набор аксиом
\begin{enumerate}
\item Для любых векторов $v_1,v_2\in V$ и $u\in U$ выполнено
\[
(v_1 + v_2) * u = v_1 * u + v_2 * u
\]

\item Для любых векторов $v\in V$, $u\in U$ и чисел $\lambda \in F$ выполнено
\[
(\lambda v)*u = \lambda (v * u)
\]

\item Для любых векторов $v\in V$ и $u_1,u_2\in U$ выполнено
\[
v * (u_1 + u_2) = v * u_1 + v * u_2
\]

\item Для любых векторов $v\in V$, $u\in U$ и чисел $\lambda \in F$ выполнено
\[
v *(\lambda u) = \lambda (v * u)
\]
\end{enumerate}
\end{definition}
То есть билинейная операция -- это умножение, от которого мы лишь требуем дистрибутивности относительно сложения и согласованность с умножением на скаляр.%
\footnote{Когда вы чуть-чуть дорастете до курса алгебры, вы узнаете, что эти свойства определяют любую операцию умножения в алгебраических структурах.}

\paragraph{Примеры}

Давайте сделаем передышку и взглянем на несколько полезных примеров:
\begin{itemize}
\item Пусть $V$ и $U$ -- два произвольных векторных пространства.
Зададим операцию умножения следующим образом $\mu\colon V\times U \to 0$ по правилу $(v, u) \mapsto 0$.
Это самая дурацкая операция умножения, которую только можно придумать.
Она не несет в себе никакой полезной информации и именно в этом заключается ее польза.

\item Пусть $V = F^n$ и $U = F^n$ зададим операцию умножения так
\[
F^n \times F^n \to F,\quad (x, y) \mapsto x^t y
\]

\item Пусть $V = F^n$ и $U = F^n$ зададим операцию умножения так
\[
F^n \times F^n \to \operatorname{M}_{n}(F),\quad (x, y)\mapsto xy^t
\]

\item Пусть $V = U= \operatorname{M}_n(F)$, тогда рассмотрим операцию матричного умножения
\[
\operatorname{M}_n(F)\times \operatorname{M}_n(F)\to \operatorname{M}_n(F),\quad (A, B)\mapsto AB
\]

\item Пусть $\mu \colon V\times U\to T$ -- произвольная операция умножения и пусть $\phi \colon T\to E$ -- произвольное линейное отображение векторных пространств.
Тогда композиция $\phi\circ \mu \colon V\times U \to E$ будет так же операцией умножения.


Таким образом некоторые операции умножения можно выразить через другие с помощью линейных отображений.
В таком смысле операция $\mu$ интереснее операции $\phi \circ \mu$, так как вторая получается из первой.
Идея заключается в том, что если мы будем знать все про первую операцию, то мы будем знать все и про вторую, так как она выражается через первую.
\end{itemize}

Прежде всего мы видим, что бывает много разных операций умножения на одной и той же паре пространств.
Кроме того, в силу последнего замечания самый большой интерес для нас представляет операция умножения, через которую можно выразить все другие операции (при условии, что такая существует).
Оказывается, что подобная замечательная операция существует и в некотором смысле единственная.
Но начнем мы с формулировки универсального свойтства.

\begin{definition}
Пусть $V$, $U$, $T$ -- векторные пространства и $\mu \colon V\times U \to T$ -- операция умножения.
Тогда $\mu$ называется универсальной, если выполняется следующее свойство: для любого векторного пространства $E$ и любой операции умножения $\phi \colon V\times U\to E$ найдется единственное линейное отображение $\psi\colon T\to E$ такое, что $\phi = \psi \circ \mu$.
Продемонстрируем происходящее на диаграмме ниже
\[
\xymatrix{
	{V\times U}\ar[rr]^{\mu}\ar[rd]_{\phi}&{}&{T}\ar@{-->}[dl]^{\psi}\\
	{}&{E}&{}
}
\]
\end{definition}
Сделаю одно полезное замечание.
Если умножение $\mu\colon V\times U\to T$ универсальное, это значит, что любая операция умножения $\phi \colon V\times U \to E$ соответствует единственному линейному отображению $\psi\colon T\to E$.
Таким образом мы можем свести изучение билинейных отображений к линейным.
Подобная процедура позволит избежать построения теории билинейных отображений, аналогичной теории линейных отображений, и пользоваться уже готовым аппаратом.

\begin{claim}
\label{claim::UniversalBilUnique}
Пусть $V$, $U$ и $T$ -- векторные пространства над полем $F$ и пусть $\mu \colon V\times U\to T$ -- универсальное умножение.
Тогда
\begin{enumerate}
\item Если $\phi \colon T\to T$ -- линейное отображение такое, что $\mu = \phi \circ \mu$, то есть следующая диаграмма коммутативна:
\[
\xymatrix@R=10pt{
	{}&{}&{T}\ar[dd]^{\phi}\\
	{V\times U}\ar[urr]^{\mu}\ar[drr]_{\mu}&{}&{}\\
	{}&{}&{T}\\
}
\]
То $\phi = \Identity_T$, то есть $\phi$ обязательно тождественное отображение.

\item Пусть $\mu'\colon V\times U\to T'$ -- другая универсальная операция умножения.
Тогда существует единственное линейное отображение $\phi\colon T\to T'$ такое, что $\mu' = \phi \circ \mu$ и при этом $\phi$ обязательно окажется изоморфизмом.
На диаграмме это можно изобразить так
\[
\xymatrix@R=10pt{
	{}&{}&{T}\ar@{-->}[dd]^{\phi}\\
	{V\times U}\ar[urr]^{\mu}\ar[drr]_{\mu'}&{}&{}\\
	{}&{}&{T'}\\
}
\]
\end{enumerate}
\end{claim}
\begin{proof}
1) С одной стороны у нас есть отображение $\phi\colon T\to T$ такое, что $\mu = \phi \circ \mu$.
С другой стороны отображение $\Identity_T\colon T\to T$ тоже удовлетворяет условию $\mu = \Identity_T \circ \mu$.
Изобразим это на диаграмме ниже
\[
\xymatrix@R=10pt{
	{}&{}&{T}\ar@<3pt>[dd]^{\phi}\ar@<-3pt>[dd]_{\Identity_T}\\
	{V\times U}\ar[urr]^{\mu}\ar[drr]_{\mu}&{}&{}\\
	{}&{}&{T}\\
}
\]
Но так как $\mu$ было универсальным, то существует одно единственное отображение с таким свойством.
А значит $\phi = \Identity_T$.

2) Пусть теперь у нас есть две универсальные операции умножения.
\[
\xymatrix@R=10pt{
	{}&{}&{T}\\
	{V\times U}\ar[urr]^{\mu}\ar[drr]_{\mu'}&{}&{}\\
	{}&{}&{T'}\\
}
\]
Так как $\mu$ является универсальной операцией, то существует единственное линейное отображение $\phi\colon T\to T'$ такое, что $\mu' = \phi \circ \mu$.
В терминах диаграммы
\[
\xymatrix@R=10pt{
	{}&{}&{T}\ar@{-->}[dd]^{\phi}\\
	{V\times U}\ar[urr]^{\mu}\ar[drr]_{\mu'}&{}&{}\\
	{}&{}&{T'}\\
}
\]
С другой стороны, так как $\mu'$ является универсальной операцией умножения, существует единственное линейное отображение $\phi'\colon T'\to T$ такое, что $\mu = \phi' \circ \mu'$.
На диаграмме это изображается так
\[
\xymatrix@R=10pt{
	{}&{}&{T}\ar@{-->}@<3pt>[dd]^{\phi}\\
	{V\times U}\ar[urr]^{\mu}\ar[drr]_{\mu'}&{}&{}\\
	{}&{}&{T'}\ar@{-->}@<3pt>[uu]^{\phi'}\\
}
\]
Тогда композиция $\phi'\circ \phi \colon T\to T$ удовлетворяет условиям первого пункта утверждения, а значит равна тождественному отображению на $T$, то есть $\phi'\circ \phi = \Identity_T$.
Аналогично, отображение $\phi\circ \phi'\colon T'\to T'$ удовлетворяет условиям первого пункта утверждения, а значит $\phi\circ \phi' = \Identity_{T'}$.
Таким образом $\phi$ и $\phi'$ -- это два взаимнообратных изоморфизма, что и требовалось.
\end{proof}

Последнее утверждение означает, что любые две универсальные операции одинаковые.
Одинаковость понимается так: мы можем найти изоморфизм между их областями значений ($T$ и $T'$ в обозначениях предыдущего утверждения) такой, что под действием этого изоморфизма одна операция превращается в другую.
Причем это отождествление единственно, то есть нет никакой неоднозначности в определении операции.
Потому нет никакого смысла отличать одну универсальную операцию от другой.
Все свойства у этих универсальных операций автоматически будут одинаковые.


\subsection{Существование тензорного произведения}

\paragraph{Достаточное условие универсальности}

Теперь когда мы знаем, что наша вожделенная операция одна единственная на всем белом свете, мы очень хотим ее найти и пощупать руками ее свойства.
Но прежде чем отправиться на поиски, давайте сформулируем полезное достаточное условие универсальности.
Оно поможет нам разглядеть ту самую.

\begin{claim}
\label{claim::UniversalBil}
Пусть $V$, $U$ и $T$ -- векторные пространства над полем $F$, $e_1,\ldots,e_n$ -- базис $V$, $f_1,\ldots,f_m$ -- базис $U$ и $\mu\colon V\times U\to T$ -- билинейная операция.
Если векторы $\mu(e_i,f_j)$ образуют базис $T$, где $1\leqslant i\leqslant n$ и $1\leqslant j\leqslant m$, то операция $\mu$ является универсальным умножением.
\end{claim}
\begin{proof}
Пусть $\mu'\colon V\times U \to T'$ -- любая другая операция умножения.
Нам надо построить линейное отображение $\varphi\colon T\to T'$ такое, что для любых векторов $v\in V$ и $u\in U$ было выполнено $\mu'(v, u) = \varphi(\mu(v, u))$.
В частности должно выполняться условие $\mu'(e_i, f_j) = \varphi(\mu(e_i, f_j))$.
Так как векторы $\mu(e_i, f_j)$ являются базисом $T$, то существует единственное линейное отображение $\varphi\colon T\to T'$ такое, что $\varphi(\mu(e_i, f_j)) = \mu'(e_i, f_j)$ (утверждение~\ref{claim::LinMapExist}).

Давайте проверим, что $\varphi$ удовлетворяет необходимому требованию.
Пусть $v\in V$ и $u\in U$, разложим их по базисам, получим $v = \sum_{i=1}^n a_i e_i$ и $u = \sum_{j=1}^m b_j f_j$ для некоторых $a_i, b_j\in F$.
Тогда
\[
\varphi(\mu(v, u)) = \varphi(\mu(\sum_{i=1}^n a_i e_i, \sum_{j=1}^m b_j f_j)) = \sum_{i,j}a_i b_j \varphi(\mu(e_i, f_j)) =  \sum_{i,j}a_i b_j \mu'(e_i, f_j) = \mu'(\sum_{i=1}^n a_i e_i, \sum_{j=1}^m b_j f_j) = \mu'(v, u)
\]
\end{proof}

\paragraph{Конструкция тензорного произведения}

Лучший способ  найти лучшую операцию -- построить ее явно.
Пусть $V$ и $U$ -- два векторных пространства над полем $F$.
Мы будем предполагать, что они оба конечномерны, но конструкция будет работать и в общем случае, надо лишь аккуратно произносить все заклинания.
Я предъявлю конструкцию, которая будет зависеть от базиса, но в силу единственности универсальной операции, стартуя с любого базиса, мы получим одно и то же пространство и операцию.

Пусть $e_1,\ldots,e_n$ -- базис $V$ и $f_1,\ldots,f_m$ -- базис $U$.
Рассмотрим набор букв $t_{ij}$, где $1\leqslant i\leqslant n$ и $1\leqslant j \leqslant m$.
Положим в качестве множество формальных линейных комбинаций букв $t_{ij}$ с коэффициентами из поля%
\footnote{Если вы боитесь слова <<буквы>> и сочетания <<формальная линейная комбинация>>, то можете думать про эти выражения, как про многочлены.},
то есть
\[
T = \{\sum_{i,j}a_{ij}t_{ij}\mid a_{ij}\in F\}
\]
То есть элементом $T$ является картинка вида $a_{11}t_{11} + a_{12}t_{12} + \ldots + a_{nm}t_{nm}$.
Две такие картинки совпадают тогда и только тогда, когда у них все коэффициенты одинаковые.
Здесь умножение коэффициента $a_{ij}$ на букву $t_{ij}$ чисто формальное, мы просто приписываем коэффициент к букве, никакого другого умножения коэффициента на букву нет.

Теперь надо на этом множестве определить операции сложения и умножения на скаляры.
Для полноты картины я напишу формулы:
\begin{itemize}
\item Элементы $T$ складываются покоэффициентно, то есть
\[
\sum_{ij}a_{ij}t_{ij} + \sum_{ij}b_{ij}t_{ij} = \sum_{ij}(a_{ij}+b_{ij})t_{ij}
\]

\item Элементы $T$ умножаются на числа покоэффициентно, то есть
\[
\lambda(\sum_{ij} a_{ij}t_{ij}) = \sum_{ij} \lambda a_{ij}t_{ij}
\]
\end{itemize}
Я позволю себе сказать, что аксиомы векторного пространства проверяются методом пристального взгляда и куда полезнее их проверку оставить читателю.
Но обратите внимание, что нулем в таком пространстве является картинка, где все коэффициенты нулевые.


Теперь надо построить операцию $\mu\colon V\times U\to T$.
Пусть $v\in V$ и $u\in U$.
Разложим векторы по базисам и получим $v = \sum_i x_i e_i$ и $u = \sum_j y_j f_j$.
Так как координаты векторов определены однозначно, мы можем определить отображение множеств $\mu\colon V\times U\to T$ по правилу
\[
\mu(v, u) = \sum_{ij}x_iy_j t_{ij}
\]
Теперь надо проверить билинейность этого отображения.
Действительно, пусть $v' = \sum_{i} x_i'e_i$, тогда
\[
\mu(v + v', u) = \sum_{ij}(x_i + x_i')y_j t_{ij} = \sum_{ij}x_iy_j t_{ij} +\sum_{ij}x_i'y_j t_{ij}  =\mu(v, u) + \mu(v', u)
\]
И для любого числа $\lambda \in F$ выполнено
\[
\mu(\lambda v, u) = \sum_{ij} (\lambda x_i)y_jt_{ij} = \lambda \sum_{ij}x_iy_jt_{ij} = \lambda\mu(v, u)
\]
Аналогично проверяется линейность по второму аргументу.

\begin{claim*}
Пространство $T$ вместе с операцией $\mu\colon V\times U\to T$ построенные выше дают универсальную операцию умножения на $V$ и $U$.
\end{claim*}
\begin{proof}
Заметим, что по построению у нас получилась билинейная операция такая, что $\mu(e_i, f_j) = t_{ij}$ -- базис пространства $T$.
Тогда по утверждению~\ref{claim::UniversalBil} эта операция является универсальной.
\end{proof}

\paragraph{Обозначения}

\begin{itemize}
\item Построенное пространство $T$ выше будем обозначать%
\footnote{Напомним, что такое пространство единственное с точностью до единственного изоморфизма.
Потому как бы вы его ни построили, вы всегда сможете отождествить две конструкции единственным образом}
$V\otimes_F U$ или кратко $V\otimes U$, когда понятно о каком поле идет речь.
Это пространство называется тензорным произведением пространств $V$ и $U$ над полем $F$, а его элементы называются тензорами.

\item Операция $\mu$ в операторной форме обозначается символом $\otimes$.
Тогда запись для операции имеет вид $\otimes \colon V\times U \to V\otimes U$ по правилу $(v,u) \mapsto v\otimes u$.

\item Если я возьму произвольный набор векторов $v_1,\ldots,v_k\in V$ и такой же длины набор $u_1,\ldots, u_k\in U$, то я могу построить элементы $v_1\otimes u_1,\ldots, v_k\otimes u_k\in V\otimes_F U$, просто перемножим векторы с помощью универсальной операции.
Кроме того, я могу сложить получившиеся векторы.
Потому в тензорном произведении пространств есть элементы вида $\sum_{i=1}^k v_i \otimes u_i$.
Оказывается, что любой элемент тензорного произведения имеет такой вид, это показано в следующем пункте.

\item В обозначениях выше, элементы $t_{ij}$ представляются в виде $t_{ij} = e_i \otimes f_j$.
А так как $t_{ij}$ -- это базис, то любой элемент $V\otimes_F U$ единственным образом представляется в виде $\sum_{ij}a_{ij}e_i\otimes f_j$.
В частности любой элемент $V\otimes_F U$ имеет вид $\sum_{i=1}^k v_i\otimes u_i$ для достаточно большого $k$.

\item Тензоры вида $v\otimes u\in V\otimes_F U$ называются разложимыми.
Примеры ниже покажут, что разложимых тензоров сильно меньше, чем неразложимых.%
\footnote{Один из примеров ниже показывает, что разложимые тензоры соответствуют матрицам ранга не более $1$.}
\end{itemize}

\paragraph{Замечание о базисах}

Мы построили тензорное произведение пространств исходя из соображения, что тензорное произведение двух фиксированных базисов является базисом тензорного произведения пространств.
А что будет, если взять любые другие базисы?
Оказывается, они тоже дадут базис тензорного произведения.

\begin{claim}
Пусть $V$ и $U$ -- векторные пространства над полем $F$.
Тогда для любых базисов $e_1',\ldots,e_n'\in V$ и $f_1',\ldots,f_m'\in U$ следует, что набор векторов $e_i' \otimes f_j'$ будет базисом в $V\otimes_F U$.
\end{claim}
\begin{proof}
Пусть $\otimes \colon V\times U\to V\otimes_F U$ -- тензорное произведение.
Построим другую его версию $\mu\colon V\times U \to T$ с помощью базисов $e_i'$ и $f_j'$ по конструкции выше.
Тогда по построению $\mu(e_i', f_j')$ будет базисом $T$.
Но мы знаем, что универсальная операция единственная, потому пространство $T$ и $V\otimes_F U$ изоморфны и при этом изоморфизме одна операция превращается в другую (утверждение~\ref{claim::UniversalBilUnique}).
Более формально, коммутативна следующая диаграмма
\[
\xymatrix@R=10pt{
	{}&{}&{V\otimes_F U}\\
	{V\times U}\ar[urr]^{\otimes}\ar[drr]_{\mu}&{}&{}\\
	{}&{}&{T}\ar@{-->}[uu]^{\phi}\\
}
\]
где $\phi\colon T\to V\otimes_F U$ является изоморфизмом.
С другой стороны из коммутативности диаграммы следует, что элементы $\mu(e_i', f_j')$ переходят в $e_i'\otimes f_j'$ под действием $\phi$.
А значит последние тоже являются базисом.
\end{proof}

\begin{claim}
Для любых векторных пространств $V$ и $U$ над полем $F$ верно, что $\dim_F V\otimes_F U = \dim_F V \dim_F U$.
\end{claim}
\begin{proof}
Это непосредственно следует из конструкции тензорного произведения, которая приведена выше.
\end{proof}

\subsection{Примеры тензорных произведений}
\label{section::TensorExmpl}

Мы очень долго говорили про нашу ненаглядную операцию, но что-то пока ни разу на нее не взглянули и не притронулись ни к одному из ее загадочных свойств.
Давайте  наконец-то перейдем к примерам.

\begin{itemize}
\item Давайте рассмотрим операцию умножения вектора на число $\mu \colon V\times F \to V$ по правилу $(v, \lambda)\mapsto \lambda v$.
Утверждается, что эта операция универсальная.
Действительно, для этого надо воспользоваться утверждением~\ref{claim::UniversalBil} и проверить, что она базисы отправляет в базис.
Пусть $e_1,\ldots,e_n$ -- базис $V$, а в качестве базиса $F$ выберем $1$.
Тогда $\mu(e_1, 1) = e_1, \ldots, \mu(e_n, 1) = e_n$ -- базис $V$.
Что доказывает универсальность операции.
Нарисуем соответствующую коммутативную диаграмму:
\[
\xymatrix@R=10pt{
	{}&{}&{V\otimes_F F}\ar@{-->}[dd]^{\phi}\\
	{V\times F}\ar[urr]^{\otimes}\ar[drr]_{\mu}&{}&{}\\
	{}&{}&{V}\\
}
\quad
\xymatrix@R=10pt{
	{}&{}&{v\otimes \lambda}\ar@{|-->}[dd]^{\phi}\\
	{(v,\lambda)}\ar@{|->}[urr]^{\otimes}\ar@{|->}[drr]_{\mu}&{}&{}\\
	{}&{}&{\lambda v}\\
}
\]
Таким образом тензорное произведение $V\otimes_F F = V$, при этом отождествлении тензорное произведение превращается в умножение вектора на скаляр, а разложимые тензоры $v\otimes \lambda$ переходят в $\lambda v$.

\item Теперь изучим тензорное произведение $F^n \otimes_F F^m$.
Рассмотрим билинейную операцию $\mu\colon F^n\times F^m \to \operatorname{M}_{n\,m}(F) $ по правилу $(x, y)\mapsto xy^t$, то есть пару векторов переводим в матрицу ранга не более $1$.
Пусть $e_1,\ldots,e_n$ -- стандартный базис $F^n$ и $f_1,\ldots,f_m$ -- стандартный базис $F^m$.
Тогда $\mu(e_i, f_j) = E_{ij}$ -- матричные единицы, а они образуют базис пространства $\operatorname{M}_{n\,m}(F)$.
Утверждение~\ref{claim::UniversalBil} тогда влечет, что данная операция является универсальной.
Нарисуем соответствующую коммутативную диаграмму
\[
\xymatrix@R=10pt{
	{}&{}&{F^n\otimes_F F^m}\ar@{-->}[dd]^{\phi}\\
	{F^n\times F^m}\ar[urr]^{\otimes}\ar[drr]_{\mu}&{}&{}\\
	{}&{}&{\operatorname{M}_{n\,m}(F)}\\
}
\quad
\xymatrix@R=10pt{
	{}&{}&{x\otimes y}\ar@{|-->}[dd]^{\phi}\\
	{(x, y)}\ar@{|->}[urr]^{\otimes}\ar@{|->}[drr]_{\mu}&{}&{}\\
	{}&{}&{x y^t}\\
}
\]
Таким образом тензорное произведение $F^n \otimes_F F^m$ совпадает с матрицами $\operatorname{M}_{n\,m}(F)$, при этом отождествлении операция тензорного произведения превращается в операцию умножения столбца на строку $(x, y)\mapsto xy^t$, а разложимый тензор $x\otimes y$ переходит в матрицу ранга не более $1$ вида $xy^t$.

Этот пример объясняет смысл определения тензорного ранга.
Один из подходов к понятию ранга тензора -- минимальное количество разложимых тензоров, в сумму которых раскладывается данный тензор.
Это определение превращается в определение тензорного ранга матриц, когда мы отождествляем $F^n\otimes_F F^m$ с $\operatorname{M}_{n\,m}(F)$.

\item Теперь покажем, как на тензорном языке определить линейные отображения.
Пусть $V$ и $U$ -- два векторных пространства и $\Hom_F(V, U)$ -- линейные отображения из $V$ в $U$ над полем $F$.
Давайте определим билинейную операцию $\mu \colon U\times V^* \to \Hom_F(V, U)$.
Чтобы определить такую операцию, нам надо для каждой пары $(u,\xi)$, где $\xi \in V^*$ и $u\in U$, задать линейное отображение $\phi_{u\,\xi}\colon V\to U$.
Определим это отображение по правилу $\phi_{u\,\xi}(v) = \xi(v)u$.
Это значит, что мы построили отображение $\mu\colon U\times V^* \to \Hom_F(V, U)$ по правилу $(u,\xi)\mapsto \phi_{u\,\xi}$.

Давайте проверим, что эта операция является универсальной.
Пусть $e_1,\ldots,e_n\in V$ -- некоторый базис $V$, $f_1,\ldots,f_m\in U$ -- некоторый базис $U$ и $\xi_1,\ldots,\xi_n\in V^*$ -- базис $V^*$ двойственный к $e_i$.
Нам надо показать, что $\phi_{f_i\,\xi_j}$ являются базисом $\Hom_F(V, U)$.
Для этого посчитаем матрицы этих отображений в паре базисов $e_1,\ldots,e_n$ и $f_1,\ldots,f_m$.
Заметим, что
\[
\phi_{f_i, \xi_j}(e_1,\ldots,e_n) = (f_1,\ldots,f_m)E_{ij},\quad\text{где $E_{ij}$ -- матричная единица}
\]
Что и требовалось.
Нарисуем соответствующую коммутативную диаграмму:
\[
\xymatrix@R=10pt{
	{}&{}&{U\otimes_F V^*}\ar@{-->}[dd]^{\phi}\\
	{U\times V^*}\ar[urr]^{\otimes}\ar[drr]_{\mu}&{}&{}\\
	{}&{}&{\Hom_F(V, U)}\\
}
\quad
\xymatrix@R=10pt{
	{}&{}&{u\otimes \xi}\ar@{|-->}[dd]^{\phi}\\
	{(u, \xi)}\ar@{|->}[urr]^{\otimes}\ar@{|->}[drr]_{\mu}&{}&{}\\
	{}&{}&{\phi_{u\,\xi}}\\
}
\]
Таким образом тензорное $U\otimes_F V^*$ произведение можно отождествить с линейными отображениями $\Hom_F(V, U)$, при этом отождествлении операция тензорного произведения совпадает с операцией $(u, \xi)\mapsto \phi_{u\,\xi}$, а разложимый тензор $v\otimes \xi$ соответствует отображению $\phi_{u\,\xi}$.
\end{itemize}

