\ProvidesFile{lecture06.tex}[Лекция 6]


\paragraph{Единственность}

\begin{claim*}
Пусть $\phi\colon \Sym{n}\to \{\pm1\}$ обладает свойством (1).
Тогда
\begin{enumerate}
\item $\phi(\Identity) = 1$

\item $\phi(\sigma^{-1}) = \phi(\sigma)^{-1}$

\item Значение $\phi$ совпадает на всех транспозициях.
\end{enumerate}
\end{claim*}
\begin{proof}
(1) Рассмотрим цепочку равенств
\[
\phi(\Identity) = \phi(\Identity^2) = \phi(\Identity)\phi(\Identity)
\]
Так как это числовое равенство (все числа $\pm 1$), то можно сократить на $\phi(\Identity)$ и получим требуемое.


(2) Рассмотрим цепочку равенств
\[
1 = \phi(\Identity) = \phi(\sigma \sigma^{-1}) = \phi(\sigma)\phi(\sigma^{-1})
\]
Значит число $\phi(\sigma^{-1})$ является обратным к $\phi(\sigma)$.%
\footnote{Так как все наши числа $\pm 1$, то можно было бы сказать $\phi(\sigma^{-1}) = \phi(\sigma)$.
Но в указанной форме равенство лучше запоминается и встретится вам еще не раз.}

(3) Заметим, что для любых различных $i, j\in X_n$ у нас обязательно существует перестановка $\tau\in\Sym{n}$ такая, что $\tau(1) = i$ и $\tau(2) = j$.%
\footnote{Я оставляю это как упражнение.}
Тогда по утверждению~\ref{claim:CycleRename} получаем $(i,j) = \tau (1,2)\tau^{-1}$.
А значит
\[
\phi(i,j) = \phi(\tau (1,2) \tau^{-1}) = \phi(\tau)\phi(1,2)\phi(\tau^{-1}) = \phi(1,2)\phi(\tau)\phi(\tau^{-1}) = \phi(1,2)
\]
В предпоследнем равенстве мы воспользовались тем, что числа можно переставлять.
Следовательно, значение на любой транспозиции равно значению на фиксированной транспозиции $(1,2)$.
То есть значение на всех транспозициях одинаковое.
\end{proof}

Теперь давайте докажем единственность.
Пусть у нас существуют два таких отображения $\phi, \psi\colon \Sym{n}\to \{\pm 1\}$ удовлетворяющие свойствам~(1) и~(2).
Давайте покажем, что $\phi(\sigma) = \psi(\sigma)$ для любой $\sigma\in\Sym{n}$.
Из утверждения~\ref{claim:PermutationStructure}  следует, что $\sigma$ представляется в виде $\sigma = \tau_1\ldots\tau_r$, где $\tau_i$ -- транспозиции.

Значение $\phi$ одно и то же на всех транспозициях: либо $1$, либо $-1$.
Предположим, что значение равно $1$.
Тогда $\phi(\sigma) = \phi(\tau_1\ldots\tau_r) = \phi(\tau_1)\ldots\phi(\tau_r) = 1$ для всех $\sigma\in\Sym{n}$, что противоречит свойству~(2).
А значит $\phi(\tau) = -1$ для любой транспозиции $\tau$.
Аналогично, $\psi(\tau) = -1$ для любой транспозиции $\tau$.
А следовательно
\[
\phi(\sigma) = \phi(\tau_1\ldots\tau_r) = \phi(\tau_1)\ldots\phi(\tau_r) = (-1)^r =\psi(\tau_1)\ldots\psi(\tau_r) = \psi(\tau_1\ldots\tau_r)=\psi(\sigma)
\]
То есть, на самом деле, все определяется значением на транспозиции.


\subsection{Подсчет знака}

\paragraph{Декремент}

{\it Декремент} перестановки $\sigma\in\Sym{n}$ -- это 
\[
	\dec(\sigma) = n - \text{<<количество нетривиальных циклов>>} - \text{<<количество неподвижных точек>>}
\]
Если рассматривать все неподвижные точки как тривиальные <<циклы>>, то формула превращается в
\[
	\dec(\sigma) = n - \text{<<количество циклов>>} 
\]
Декремент можно описать еще так: каждая перестановка $\sigma$ определяет граф на множестве вершин $X_n$, где $(i,j)$ -- ребро, если $\sigma(i) = j$.
Тогда 
\[\
\dec(\sigma) = \text{<<количество вершин>>} - \text{<<количество компонент графа>>}
\]

\begin{claim}
Пусть $\sigma\in \Sym{n}$, тогда $\sgn(\sigma) = (-1)^{\dec(\sigma)}$.
\end{claim}
\begin{proof}
Действительно, разложим перестановку $\sigma$ в произведение независимых циклов $\sigma = \rho_1 \ldots \rho_k$.
Пусть длины циклов $d_1,\ldots, d_k$, соответственно.
Тогда 
\[
\sgn(\sigma) = (-1)^{d_1 - 1} \ldots (-1)^{d_k - 1} = (-1)^{\sum_i d_i - k}
\]
Пусть $s$ -- количество неподвижных точек.
Тогда 
\[
\sgn(\sigma) = (-1)^{\left(\sum_i d_i + s\right) - k - s} = (-1)^{n - k - s} = (-1)^{\dec(\sigma)}
\]
\end{proof}


При подсчете знака перестановки надо пользоваться декрементом.
То есть надо разложить перестановку в произведение независимых циклов и сложить их длины без единицы.
Например:
\[
\sigma = 
\begin{pmatrix}
{1}&{2}&{3}&{4}&{5}&{6}&{7}&{8}&{9}\\
{4}&{8}&{2}&{3}&{7}&{1}&{5}&{9}&{6}\\
\end{pmatrix}
\]
Теперь видим, что
\begin{align*}
1 &\to 4 \to 3 \to 2 \to 8 \to 9 \to 6 \to 1\\
5 &\to 7 \to 5
\end{align*}
Значит $\sigma = (1,4,3,2,8,9,6)(5,7)$, а значит $\dec(\sigma) = 6 + 1 = 7$ и $\sgn(\sigma) = -1$.


\subsection{Возведение в степень}

Прежде всего сделаем два простых наблюдения:
\begin{enumerate}
\item Пусть $\sigma, \tau\in \Sym{n}$ -- две коммутирующие перестановки, тогда $(\sigma \tau)^m = \sigma^m \tau^m$.

\item Пусть $\rho \in\Sym{n}$ -- цикл длины $d$, тогда $d$ совпадает с наименьшим натуральным числом $k$ таким, что $\rho^k = \Identity$.
\end{enumerate}

Пусть теперь $\sigma\in\Sym{n}$ -- произвольная перестановка.
Мы можем разложить ее в произведение независимых циклов $\sigma = \rho_1 \ldots \rho_k$ с длинами $d_1, \ldots, d_k$, соответственно.
Тогда 
\[
\sigma^m = \rho_1^m \ldots \rho_k^m = \rho_1^{m\pmod{d_1}}\ldots \rho_k^{m \pmod{d_k}}
\]
Таким образом, расчет произвольной степени перестановки $\sigma$ сводится к возведению циклов в степень не большую их длины.

Оставим еще одно замечание в качестве упражнения.
Если $\sigma = \rho_1\ldots \rho_k$ -- разложение в произведение независимых циклов длин $d_1,\ldots,d_k$, соответственно, то наименьшее натуральное $r$ такое, что $\sigma^r = \Identity$, равно наименьшему общему кратному чисел $d_1,\ldots,d_k$.


\subsection{Произведение циклов}

В этом разделе я приведу несколько примеров того, как перемножаются между собой зависимые циклы.

\paragraph{Два цикла}

Пусть циклы $\sigma,\tau\in \Sym{n}$ пересекаются по одному элементу как на рисунке ниже
\[
\xymatrix{
	{}&{i_3}\ar[dr]^{\sigma}&{}&{i_5}\ar@{-->}[dr]^{\tau}&{}\\
	{i_2}\ar[ur]^{\sigma}&{}&{i_4}\ar[dl]^{\sigma}\ar@{-->}[ur]^{\tau}&{}&{i_6}\ar@{-->}[dl]^{\tau}\\
	{}&{i_1}\ar[ul]^{\sigma}&{}&{i_7}\ar@{-->}[ul]^{\tau}&{}\\
}
\]
Надо найти произведение $\sigma\tau$.
Нетрудно видеть, что результат имеет следующий вид:
\[
\xymatrix{
	{}&{i_3}\ar[dr]&{}&{i_5}\ar[dr]&{}\\
	{i_2}\ar[ur]&{}&{i_4}
	\ar[ur]&{}&{i_6}\ar[dl]\\
	{}&{i_1}\ar[ul]&{}&{i_7}\ar[ll]&{}\\
}
\]
Таким образом мы получили формулу $(i_1,\ldots,i_k)(i_k,\ldots,i_n) = (i_1,\ldots,i_n)$.

\paragraph{Цикл и транспозиция}

Пусть $\sigma,\tau\in\Sym{n}$, где $\sigma$ -- цикл, а $\tau$ -- транспозиция, переставляющая два элемента цикла $\sigma$ как на рисунке ниже.
\[
\xymatrix{
	{}&{i_2}\ar[r]^{\sigma}&{i_3}\ar[rd]^{\sigma}&{}\\
	{i_1}\ar[ur]^{\sigma}\ar@{-->}@/^10pt/[rrr]^{\tau}&{}&{}&{i_4}\ar[dl]^{\sigma}\ar@{-->}@/^10pt/[lll]^{\tau}\\
	{}&{i_6}\ar[ul]^{\sigma}&{i_5}\ar[l]^{\sigma}&{}\\
}
\]
Вот так выглядят композиции для $\sigma\tau$ и $\tau\sigma$ соответственно
\[
\xymatrix{
	{}&{i_2}\ar[r]&{i_3}\ar[dr]&{}\\
	{i_1}\ar[drr]&{}&{}&{i_4}\ar[ull]\\
	{}&{i_6}\ar[ul]&{i_5}\ar[l]&{}\\
}
\quad\quad\quad
\xymatrix{
	{}&{i_2}\ar[r]&{i_3}\ar[dll]&{}\\
	{i_1}\ar[ur]&{}&{}&{i_4}\ar[dl]\\
	{}&{i_6}\ar[urr]&{i_5}\ar[l]&{}\\
}
\]
Таким образом общее правило выглядит так:
\begin{align*}
(i_1,\ldots,i_n)(i_1,i_k) &= (i_1,i_{k+1},\ldots,i_n)(i_2,\ldots,i_k)\\
(i_1,i_k)(i_1,\ldots,i_n) &= (i_1,\ldots,i_{k-1})(i_k,\ldots,i_n)
\end{align*}

\paragraph{Пара циклов и транспозиция}

Пусть $\sigma,\tau\in\Sym{n}$, причем $\sigma$ -- произведение двух независимых циклов, а $\tau$ -- транспозиция, переставляющая две вершины из разных циклов как на рисунке ниже.
\[
\xymatrix{
{}&{i_3}\ar[dr]^{\sigma}&{}&{}&{i_6}\ar[dr]^{\sigma}&{}\\
{i_2}\ar[ur]^{\sigma}&{}&{i_4}\ar[dl]^{\sigma}\ar@{-->}@/^10pt/[r]^{\tau}&{i_5}\ar[ur]^{\sigma}\ar@{-->}@/^10pt/[l]^{\tau}&{}&{i_7}\ar[dl]^{\sigma}\\
{}&{i_1}\ar[ul]^{\sigma}&{}&{}&{i_8}\ar[ul]^{\sigma}&{}\\
}
\]
Произведения $\sigma\tau$ и $\tau\sigma$ имеют вид
\[
\xymatrix{
	{}&{i_3}\ar[dr]&{}&{}&{i_6}\ar[dr]&{}\\
	{i_2}\ar[ur]&{}&{i_4}\ar[urr]&{i_5}\ar[dll]&{}&{i_7}\ar[dl]\\
	{}&{i_1}\ar[ul]&{}&{}&{i_8}\ar[ul]&{}\\
}
\quad\quad\quad
\xymatrix{
	{}&{i_3}\ar[drr]&{}&{}&{i_6}\ar[dr]&{}\\
	{i_2}\ar[ur]&{}&{i_4}\ar[dl]&{i_5}\ar[ur]&{}&{i_7}\ar[dl]\\
	{}&{i_1}\ar[ul]&{}&{}&{i_8}\ar[ull]&{}\\
}
\]
Таким образом общее правило выглядит так
\begin{align*}
(i_1,\ldots,i_k)(i_{k+1},\ldots,i_n)(i_k,i_{k+1}) &=(i_1,\ldots,i_k,i_{k+2},\ldots,i_n,i_{k+1})\\
(i_k,i_{k+1})(i_1,\ldots,i_k)(i_{k+1},\ldots,i_n) &= (i_k,i_1,\ldots,i_{k-1},i_{k+1},\ldots,i_n)
\end{align*}

\newpage

\section{Определитель}

\subsection{Философия}

Сейчас я хочу обсудить <<ориентированный объем>> на прямой, плоскости и в пространстве.

\paragraph{Прямая}

 На прямой мы можем выбрать <<положительное>> направление.
 Обычно на рисунке выбирают слева направо.
 Тогда длина вектора, который смотрит слева направо, считается положительной, а справа налево -- отрицательной.

\paragraph{Плоскость}

Здесь объем будет задаваться парой векторов, то есть некоторой квадратной матрицей размера $2$, где вектора -- это ее столбцы.
Основная идея такая: пусть мы хотим посчитать площадь между двумя векторами на плоскости, точнее площадь параллелограмма натянутого на вектора $e_1$ и $e_2$ как на первом рисунке ниже.
\[
\xymatrix{
	{}&{}&{}\ar@{--}[r]&{}\\
	{}\ar[urr]^{e_2}\ar[r]_{e_1}&{}\ar@{--}[urr]&{}&{}\\
	{}&{}&{}&{}\\
}\quad
\xymatrix{
	{}&{}&{}\\
	{}\ar[rr]^{e_2}\ar[r]_{e_1}&{}&{}\\
	{}&{}&{}\\
}\quad
\xymatrix{
	{}&{}&{}&{}\\
	{}\ar[drr]_{e_2}\ar[r]^{e_1}&{}\ar@{--}[drr]&{}&{}\\
	{}&{}&{}\ar@{--}[r]&{}\\
}
\]
Давайте двигать вектор $e_2$ к вектору $e_1$.
Тогда площадь будет уменьшаться и когда вектора совпадут, она будет равна нулю.
Однако, если мы продолжим двигать вектор $e_2$, то площадь между векторами опять начнет расти и картинка в конце концов станет симметрична исходной, а полученный параллелограмм равен изначальному.
Однако, эта ситуация отличается от предыдущей и вот как можно понять чем.
Предположим, что между векторами была натянута хорошо сжимаемая ткань, одна сторона которой красная, другая -- зеленая.
Тогда в самом начале на нас смотрит красная сторона этой ткани, но как только $e_2$ прошел через $e_1$ на нас уже смотрит зеленая сторона.
Мы бы хотели научиться отличать эти две ситуации с помощью знака, если на  нас смотрит красная сторона -- знак положительный, если зеленая -- отрицательный.

Еще один способ думать про эту ситуацию.
Представим, что плоскость -- это наш стол, а параллелограмм вырезан из бумаги.
Мы можем положить параллелограмм на стол двумя способами: лицевой стороной вверх или же вниз.
В первом случае мы считаем площадь положительной, а во втором -- отрицательной.
Возможность определить лицевую сторону связана с тем, что мы знаем, где у стола верх, а где низ.
Это возможно, потому что наша плоскость лежит в трехмерном пространстве и мы можем глядеть на нее извне.
Однако, если бы мы жили на плоскости и у нас не было бы возможности выглянуть за ее пределы, то единственный способ установить <<какой стороной вверх лежит параллелограмм>> был бы с помощью порядка векторов.

Еще одно важное замечание.
Если мы берем два одинаковых параллелограмма на нашем столе, которые лежат лицевой стороной вверх, то мы можем передвинуть один в другой, не отрывая его от стола.
А вот если один из параллелограммов имеет положительный объем, а другой отрицательный, то нельзя перевести один в другой, не отрывая от стола.
То есть, если вы живете на плоскости, то вам не получится переместить положительный параллелограмм в отрицательный, не сломав или не разобрав его.

\paragraph{Пространство}

В пространстве дело с ориентацией обстоит абсолютно аналогично.
Мы хотим уже считать объемы параллелепипедов натянутых на три вектора.
И мы так же хотим, чтобы эти объемы показывали <<с какой стороны>> мы смотрим на параллелепипед.
\[
\xymatrix{
	{}&{}\ar@{--}[rr]\ar@{--}[dl]&{}&{}\ar@{--}[dl]\\
	{}\ar@{--}[rr]&{}&{}&{}\\
	{}&{\cdot}\ar[uu]^(.3){e_3}\ar[rr]^(.3){e_2}\ar[dl]_{e_1}&{}&{}\ar@{--}[dl]\ar@{--}[uu]\\
	{}\ar@{--}[rr]\ar@{--}[uu]&{}&{}\ar@{--}[uu]&{}\\
}
\quad\quad\quad
\xymatrix{
	{}&{}\ar@{--}[rr]\ar@{--}[dl]&{}&{}\ar@{--}[dl]\\
	{}\ar@{--}[rr]&{}&{}&{}\\
	{}&{\cdot}\ar[uu]^(.3){e_2}\ar[rr]^(.3){e_3}\ar[dl]_{e_1}&{}&{}\ar@{--}[dl]\ar@{--}[uu]\\
	{}\ar@{--}[rr]\ar@{--}[uu]&{}&{}\ar@{--}[uu]&{}\\
}
\quad\quad\quad
\xymatrix{
	{}&{}\ar@{--}[rr]\ar@{--}[dl]&{}&{}\ar@{--}[dl]\\
	{}\ar@{--}[rr]&{}&{}&{}\\
	{}&{\cdot}\ar[uu]^(.3){e_1}\ar[rr]^(.3){e_3}\ar[dl]_{e_2}&{}&{}\ar@{--}[dl]\ar@{--}[uu]\\
	{}\ar@{--}[rr]\ar@{--}[uu]&{}&{}\ar@{--}[uu]&{}\\
}
\]
Здесь знак объема определяется по порядку векторов, как знак перестановки.
На рисунке объемы первого и третьего положительные, а у второго отрицательный.
Если вы сделаете модельки этих кубиков из подписанных спичек, то третий кубик -- это первый, но лежащий на другой грани.
А вот второй кубик получить из первого вращениями не получится.
Надо будет его разобрать и присобачить ребра по-другому.

Как и в случае с плоскостью, если бы мы могли выйти за пределы нашего трехмерного пространства, то у нас появилась бы лицевая и тыльная сторона, как у стола.
И тогда первый и третий кубики лежали бы лицевой стороной вверх, а второй -- вниз.
Мы, конечно же, так сделать не сможем и никогда в жизни не увидим подобное, но думать про такое положение вещей по аналогии с плоскостью можем и эта интуиция бывает полезна.

\paragraph{Пояснение планов}

В текущей лекции я не собираюсь обсуждать объемы, а всего лишь хочу коснуться некоторой техники, которая используется для работы с ориентированными объемами.
Чтобы начать честный рассказ про сами объемы (который обязательно будет, но позже), нам надо поговорить о том, что такое векторное пространство и как в абстрактном векторном пространстве мерить расстояния и углы.
Потому, пока мы не покроем эти темы, всерьез говорить про настоящие объемы мы не сможем.

\subsection{Три разных определения}

Начнем с классического определения в виде явной формулы.

\paragraph{Определитель (I)}

Рассмотрим отображение $\det \colon \Matrix{n}\to \mathbb R$, задаваемое следующей формулой: для любой матрицы $A\in\Matrix{n}$ положим
\[
\det A = \sum_{\sigma \in \Sym{n}} \sgn(\sigma) a_{1\sigma(1)}\ldots a_{n\sigma(n)}
\]
Данное отображение называется {\it определителем}, а его значение $\det A$ на матрице $A$ называется определителем матрицы $A$.

Давайте неформально обсудим, как считается выражение для определителя.
Как мы видим определитель состоит из суммы некоторых произведений.
Каждое произведение имеет вид $a_{1\sigma(1)}\ldots a_{n\sigma(n)}$ умноженное на $\sgn(\sigma)$.
Здесь из каждой строки матрицы $A$%
\footnote{Первый индекс -- индекс строки.}
выбирается по одному элементу так, что никакие два элемента не лежат в одном столбце (это гарантированно тем, что $\sigma$ -- перестановка и потому $\sigma(i)$ не повторяются).
Заметим, что слагаемых ровно столько, сколько перестановок -- $n!$ штук.
Из этих слагаемых половина идет со знаком плюс, а другая -- со знаком минус.

\paragraph{Нормированные полилинейные кососимметрические отображения (II)} 

Пусть $\phi\colon \Matrix{n}\to \mathbb R$ -- некоторое отображение и $A\in\Matrix{n}$.
Тогда про матрицу $A$ можно думать, как про набор из $n$ столбцов: $A = (A_1|\ldots|A_n)$.
Тогда функцию $\phi(A) = \phi(A_1,\ldots,A_n)$ можно рассматривать как функцию от $n$ столбцов.

В обозначениях выше рассмотрим отображения $\phi\colon \Matrix{n}\to \mathbb R$, удовлетворяющие следующим свойствам:
\begin{enumerate}
\item $\phi(A_1,\ldots, A_i + A_i', \ldots, A_n) = \phi(A_1,\ldots, A_i, \ldots, A_n) + \phi(A_1,\ldots,A_i', \ldots, A_n)$ для любого $i$.

\item $\phi(A_1,\ldots, \lambda A_i, \ldots, A_n) = \lambda \phi(A_1,\ldots, A_i, \ldots, A_n)$ для любого $i$ и любого $\lambda\in\mathbb R$.

\item $\phi(A_1,\ldots, A_i, \ldots, A_j, \ldots, A_n) = -\phi(A_1,\ldots, A_j, \ldots, A_i, \ldots, A_n)$ для любых различных $i$ и $j$.

\item $\phi(E) = 1$.
\end{enumerate}

Первые два свойства вместе называются {\it полилинейностью} $\phi$ по столбцам, т.е. это уважение суммы и умножения на скаляр.
Третье свойство называется {\it кососимметричностью} $\phi$ по столбцам.
Последнее условие -- это условие нормировки.
Данный набор свойств можно заменить эквивалентным с переформулированным третьим свойством:
\begin{enumerate}
\item $\phi(A_1,\ldots, A_i + A_i', \ldots, A_n) = \phi(A_1,\ldots, A_i, \ldots, A_n) + \phi(A_1,\ldots,A_i', \ldots, A_n)$ для любого $i$.

\item $\phi(A_1,\ldots, \lambda A_i, \ldots, A_n) = \lambda \phi(A_1,\ldots, A_i, \ldots, A_n)$ для любого $i$ и любого $\lambda\in\mathbb R$.

\item $\phi(A_1,\ldots, A', \ldots, A', \ldots, A_n) = 0$, т.е. если есть два одинаковых столбца, то значение $\phi$ равно нулю.

\item $\phi(E) = 1$.
\end{enumerate}

Действительно, обозначим $\Phi(a,b) = \phi(A_1,\ldots, a, \ldots, b, \ldots, A_n)$.
Тогда $\Phi$ полилинейная функция двух аргументов.%
\footnote{Такие отображения называются билинейными.}
И нам надо показать, что $\Phi(a,a) = 0$ для любого $a\in \mathbb R^n$ тогда и только тогда, когда $\Phi(a,b) = -\Phi(b,a)$ для любых $a,b\in\mathbb R^n$.
Для $\Rightarrow$ подставим $b = a$, получим $\Phi(a,a) = -\Phi(a,a)$.
Для обратного $\Leftarrow$ подставим $a+b$, получим $\Phi(a+b, a+b) = 0$.
Раскроем скобки: $\Phi(a,a) + \Phi(a,b) + \Phi(b,a) + \Phi(b,b) = 0$.
Откуда следует требуемое.

\paragraph{Пример}

На всякий случай поясню все свойства выше на примерах:
\begin{enumerate}
\item
$
\phi
\begin{pmatrix}
{1}&{3}\\
{2}&{7}
\end{pmatrix}
=
\phi
\left(\left.
\begin{pmatrix}
{1}\\{2}
\end{pmatrix}
\right|
\begin{pmatrix}
{3}\\{7}
\end{pmatrix}
\right)
=
\phi
\left(\left.
\begin{pmatrix}
{1}\\{2}
\end{pmatrix}
\right|
\begin{pmatrix}
{1}\\{3}
\end{pmatrix}
+
\begin{pmatrix}
{2}\\{4}
\end{pmatrix}
\right)
=
\phi
\begin{pmatrix}
{1}&{1}\\
{2}&{3}
\end{pmatrix}
+
\phi
\begin{pmatrix}
{1}&{2}\\
{2}&{4}
\end{pmatrix}
$

\item
$
\phi
\begin{pmatrix}
{1}&{3}\\
{2}&{9}
\end{pmatrix}
=
3\,
\phi
\begin{pmatrix}
{1}&{1}\\
{2}&{3}
\end{pmatrix}
$

\item
$
\phi
\begin{pmatrix}
{1}&{3}\\
{2}&{7}
\end{pmatrix}
=-
\phi
\begin{pmatrix}
{3}&{1}\\
{7}&{2}
\end{pmatrix}
$

\item[3']
$
\phi
\begin{pmatrix}
{1}&{1}\\
{2}&{2}
\end{pmatrix}
=0
$
\end{enumerate}

Везде далее будем упоминать отображения с такими свойствами, как отображения со свойством (II).

\paragraph{Нормированные полилинейные кососимметрические отображения (II')}

Аналогично (II) можно рассмотреть полилинейные кососимметрические отображения по строкам матрицы $A$ вместо столбцов.
Тогда можно рассматривать отображения $\phi'\colon \Matrix{n}\to \mathbb R$ с аналогами четырех свойств выше: полилинейность, кососимметричность, значение $1$ на единичной матрице.
Такие отображения мы будем называть, как отображения со свойствами (II').


\paragraph{Специальные мультипликативные отображения (III)} 

Рассмотрим множество отображений $\psi\colon \Matrix{n}\to \mathbb R$ удовлетворяющие следующими свойствам:
\begin{enumerate}
\item $\psi(AB) = \psi(A)\psi(B)$ для любых $A,B\in\Matrix{n}$.

\item 
$
\psi
\begin{pmatrix}
{1}&{}&{}&{}\\
{}&{\ddots}&{}&{}\\
{}&{}&{1}&{}\\
{}&{}&{}&{d}\\
\end{pmatrix}
= 
d
$ для любого ненулевого $d\in\mathbb R$.
\end{enumerate}
Всюду ниже будем упоминать отображения с такими свойствами, как отображения со свойством (I).%
\footnote{Обратите внимание, что существует много отображений со свойством (1), не удовлетворяющих свойству (2).
Действительно, если $\psi$ -- мультипликативное отображение, то есть удовлетворяет только свойству (1), то $\gamma_n(A) = \psi(A)^n$ -- тоже мультипликативное отображение для любого натурального $n\in \mathbb N$.
Кроме того, $\delta_\alpha(A) = |\psi(A)|^\alpha$ тоже является мультипликативным отображением для любого положительного $\alpha\in \mathbb R$.}


\paragraph{План дальнейших действий} 

Наша задача показать, что, во-первых, определитель обладает свойствами  (II), (II') и (III), а, во-вторых, что кроме определителя никакое другое отображение не удовлетворяет этим свойствам.
То есть все три определения между собой эквивалентны.
Самое сложное будет показать, что (III) влечет остальные два определения.
Это означает, что (III) легко проверять, но из него сложно выводить какие-либо свойства.
Самые полезные с вычислительной точки зрения -- определения (II) и (II').


\subsection{Явные формулы для определителя}

\paragraph{Подсчет в малых размерностях}

\begin{enumerate}
\item Если $A\in \Matrix{1} = \mathbb R$, то $\det A = A$.

\item Если $A \in \Matrix{2}$ имеет вид 
$
A = \left(\begin{smallmatrix}{a}&{b}\\{c}&{d}\end{smallmatrix}\right)
$
, то $\det A = ad - bc$.
Графически: главная диагональ минус побочная.

\item Если $A\in \Matrix{3}$ имеет вид
$
A = \left(\begin{smallmatrix}{a_{11}}&{a_{12}}&{a_{13}}\\{a_{21}}&{a_{22}}&{a_{23}}\\{a_{31}}&{a_{32}}&{a_{33}}\end{smallmatrix}\right)
$
, то определитель получается из $6$ слагаемых три из них с $+$ три с $-$.
Графически слагаемые можно изобразить так:
\[
\det A = 
+
\left(
\parbox{6pt}{
\xymatrix@R=6pt@C=6pt{
	{}\ar@{-}[ddrr]&{}&{}\\
	{}&{}&{}\\
	{}&{}&{}\\
}}
\right) +
\left(
\parbox{6pt}{
\xymatrix@R=6pt@C=6pt{
	{}&{}\ar@{-}[dr]&{}\\
	{}&{}&{}\ar@{-}[dll]\\
	{}\ar@{-}[uur]&{}&{}\\
}}
\right) +
\left(
\parbox{6pt}{
\xymatrix@R=6pt@C=6pt{
	{}&{}&{}\ar@{-}[ddl]\\
	{}\ar@{-}[urr]&{}&{}\\
	{}&{}\ar@{-}[ul]&{}\\
}}
\right) -
\left(
\parbox{6pt}{
\xymatrix@R=6pt@C=6pt{
	{}&{}&{}\ar@{-}[ddll]\\
	{}&{}&{}\\
	{}&{}&{}\\
}}
\right) -
\left(
\parbox{6pt}{
\xymatrix@R=6pt@C=6pt{
	{}&{}\ar@{-}[ddr]&{}\\
	{}\ar@{-}[ur]&{}&{}\\
	{}&{}&{}\ar@{-}[ull]\\
}}
\right) -
\left(
\parbox{6pt}{
\xymatrix@R=6pt@C=6pt{
	{}\ar@{-}[drr]&{}&{}\\
	{}&{}&{}\ar@{-}[dl]\\
	{}&{}\ar@{-}[uul]&{}\\
}}
\right) 
\]
Точная формула%
\footnote{Для больших размерностей чем $3$ на $3$ явная формула не пригодна из-за слишком большого числа слагаемых.
Даже с вычислительной точки зрения.}
\[
\det A = a_{11}a_{22}a_{33} + a_{12}a_{23}a_{31} + a_{13}a_{21}a_{32} - 
a_{13}a_{22}a_{31} - a_{12}a_{21}a_{33} - a_{11}a_{23}a_{32}
\]
\end{enumerate}

\paragraph{Треугольные матрицы}

\begin{claim}
\label{claim::DetUpperTr}
Для любых верхне и нижне треугольных матриц верны следующие формулы:
\[
\det
\begin{pmatrix}
{\lambda_1}&{\ldots}&{*}\\
{}&{\ddots}&{\vdots}\\
{}&{}&{\lambda_n}\\
\end{pmatrix}
 = 
\lambda_1 \ldots \lambda_n
\quad
\det
\begin{pmatrix}
{\lambda_1}&{}&{}\\
{\vdots}&{\ddots}&{}\\
{*}&{\ldots}&{\lambda_n}\\
\end{pmatrix}
 = 
 \lambda_1 \ldots \lambda_n
\]
В частности $\det E = 1$.
\end{claim}
\begin{proof}
Я докажу утверждение для верхнетреугольных матриц, нижнетреугольный случай делается аналогично.
Для доказательства надо посчитать определитель по определению и увидеть, что только одно слагаемое соответствующее тождественной перестановке является не нулем.
Действительно, рассмотрим выражение $a_{1\sigma(1)}\ldots a_{n\sigma(n)}$.
Посмотрим, когда это выражение не ноль.
Последний множитель $a_{n\sigma(n)}$ лежит в последней строке и должен быть не ноль.
Для этого должно выполняться $\sigma(n) = n$.
Теперь $a_{n-1\sigma(n-1)}$ должен быть не ноль.
Так как $\sigma(n) = n$, то $\sigma(n - 1)\neq n$.
А значит, чтобы $a_{n-1 \sigma(n-1)}$ был не ноль, остается только один случай $\sigma(n-1) = n-1$.
Продолжая аналогично, мы видим, что $\sigma(i) = i$ для всех строк $i$.
\end{proof}


\subsection{Свойства определителя}

\paragraph{Определитель и транспонирование}

Прежде чем перейти к доказательству следующего утверждения сделаем одно полезное наблюдение.
Если мы возьмем две произвольные перестановки $\sigma,\tau\in\Sym{n}$ и матрицу $A\in\Matrix{n}$, то выражения $a_{\tau(1) \sigma(\tau (1))}\ldots a_{\tau(n) \sigma(\tau(n))}$ совпадает с выражением $a_{1\sigma(1)}\ldots a_{n\sigma(n)}$ с точностью до перестановки сомножителей.
Это делается методом пристального взгляда: замечаем что каждый сомножитель одного выражения ровно один раз встречается в другом и наоборот.

\begin{claim}
\label{claim::DetTranspose}
Пусть $A\in \Matrix{n}$, тогда $\det A = \det A^t$.
\end{claim}
\begin{proof}
Посчитаем по определению $\det A^t$, получим
\[
\det A^t = \sum_{\sigma\in\Sym{n}}\sgn(\sigma)a_{\sigma(1)1} \ldots a_{\sigma(n)n} = 
\sum_{\sigma\in\Sym{n}}\sgn(\sigma)a_{\sigma(1)\sigma^{-1}(\sigma(1))} \ldots a_{\sigma(n)\sigma^{-1}(\sigma(n))}
\]
Теперь применим наше замечание перед доказательством:
\[
a_{\sigma(1)\sigma^{-1}(\sigma(1))} \ldots a_{\sigma(n)\sigma^{-1}(\sigma(n))}
=
a_{1 \sigma^{-1}(1)}\ldots a_{n\sigma^{-1}(n)}
\]
Значит
\[
\det A^t = \sum_{\sigma\in\Sym{n}}\sgn(\sigma) a_{1 \sigma^{-1}(1)}\ldots a_{n\sigma^{-1}(n)}
\]
Вспомним, что $\sgn(\sigma) = \sgn(\sigma^{-1})$.
Следовательно:
\[
\det A^t = \sum_{\sigma\in\Sym{n}}\sgn(\sigma^{-1}) a_{1 \sigma^{-1}(1)}\ldots a_{n\sigma^{-1}(n)}
\]
Теперь, если $\sigma$ пробегает все перестановки, то $\sigma^{-1}$ тоже пробегает все перестановки, так как отображение $\Sym{n}\to \Sym{n}$ по правилу $\sigma\mapsto \sigma^{-1}$ является биекцией.%
\footnote{Оно биекция, так как имеет обратное -- оно само.}
То есть мы можем сделать замену $\tau = \sigma^{-1}$ и приходим к выражению
\[
\det A^t = \sum_{\tau\in\Sym{n}}\sgn(\tau)a_{1\tau(1)}\ldots a_{n\tau(n)}
\]
Последнее в точности совпадает с определением $\det A$.
\end{proof}

Отметим, что если мы доказали какое-то свойство определителя для столбцов, то это утверждение автоматически гарантирует, что такое же свойство выполнено и для строк.
И наоборот, если что-то сделано для строк, то это автоматом следует для столбцов.
