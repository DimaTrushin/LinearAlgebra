\ProvidesFile{lecture19.tex}[Лекция 19]


\subsection{Идеальный спектр}

\begin{definition}
Пусть $\varphi \colon V\to V$ -- линейный оператор над произвольным полем $F$, тогда идеальный спектр $\varphi$ это следующее множество:
\[
\spec_F^I\varphi :=\{p\in F[t]\mid p\text{ -- неприводим со старшим коэффициентом } 1 \text{ и } p(\varphi)\text{ необратим}\}
\]
\end{definition}

\paragraph{Пример}

Если рассмотреть линейный оператор $A\colon \mathbb R^2\to \mathbb R^2$ заданный матрицей $A = \left(\begin{smallmatrix}{0}&{-1}\\{1}&{0}\end{smallmatrix}\right)$, то его минимальный многочлен будет $f = x^2 +1$.
Таким образом вещественный спектр пуст $\spec_\mathbb R A = \varnothing$, так как $f$ неприводим и не линеен, а значит не имеет корней.
Однако идеальный спектр будет непустым $\spec_\mathbb R^I A = \{x^2 + 1\}$.

\begin{claim}
Пусть $\varphi\colon V\to V$ -- некоторый оператор над произвольным полем $F$ и $f_\text{min}$ -- его минимальный зануляющий многочлен над $F$.
Тогда
\begin{enumerate}
\item Для любого зануляющего многочлена $f$ и любого $p\in \spec_F^I \varphi$ следует, что $p$ делит $f$.

\item $p\in \spec_F^I \varphi$ тогда и только тогда, когда $p$ делит $f_{\text{min}}$.
\end{enumerate}
\end{claim}
\begin{proof}
(1) Для этого достаточно показать, что если $p$ не делит $f$, то $p(\varphi)$ обратим.
Действительно, так как $p$ неприводим, это означает, что $f$ и $p$ взаимно просты.
Тогда по расширенному алгоритму Евклида мы знаем, что $1 = u(t) p(t) + v(t) f(t)$ для некоторых многочленов $u(t), v(t)\in F[t]$.
Тогда подставив в последнее равенство $\varphi$ мы видим 
\[
\Identity = u(\varphi) p(\varphi) + v(\varphi) f(\varphi) = u(\varphi) p(\varphi)
\]
То есть $u(\varphi)$ -- обратный к $p(\varphi)$, что и требовалось.

(2) Пусть теперь $f_\text{min}$ -- минимальный многочлен и пусть $p$ -- неприводимый делитель $f_\text{min}$, то есть $f_\text{min} = p h$.
Надо показать, что $p(\varphi)$ необратим.

Мы знаем, что $0 = f_\text{min}(\varphi) = p(\varphi) h(\varphi)$.
Предположим, что $p(\varphi)$ обратим.
Тогда в равенстве $p(\varphi)h(\varphi) = 0$ можно сократить на $p(\varphi)$.
Значит $h(\varphi) = 0$, что противоречит минимальности $\varphi$.
\end{proof}

\paragraph{Замечания}

\begin{itemize}
\item Пусть $f_{\text{min}}$ раскладывается на линейные множители
\[
f_{\text{min}}(t) = (t - \lambda_1)^{k_1}\ldots(t-\lambda_r)^{k_r}
\]
Тогда идеальный спектр $\varphi$ -- это в точности многочлены $\{t-\lambda_1,\ldots,t-\lambda_r\}$.
То есть каждый элемент идеального спектра однозначно соответствует элементу обычного спектра $\spec_F\varphi = \{\lambda_1,\ldots,\lambda_r\}$.
Потому идеальный спектр можно рассматривать как обобщение понятия спектра на случай, когда в минимальном многочлене есть нелинейные множители.

\item Последнее утверждение можно рассматривать как обобщение утверждений~\ref{claim::PolyAnnihilator} и~\ref{claim::MinPoly} о том, что спектр лежит среди корней зануляющего многочлена и в точности совпадает с корнями минимального.

\item Так как минимальный многочлен делит характеристический, то любой элемент идеального спектра является делителем характеристического многочлена.

На самом деле можно показать, что элементы идеального спектра -- это в точности делители характеристического многочлена.
Но я не буду вас мучить доказательством этого утверждения нашими методами.

Давайте я намекну про правильный способ.
Пусть $\varphi\colon V\to V$ -- некоторый оператор над полем $F$ и пусть $f_\text{min} = p_1^{k_1}\ldots p_r^{k_r}$.
Предположим, что многочлен $\chi_\varphi$ имеет неприводимый делитель $p$ отличный от всех $p_i$.
Пусть $\bar F$ -- алгебраическое замыкание $F$.
Тогда можно заменить оператор $\varphi$ на его версию над $\bar F$, а именно $\varphi_{\bar F}\colon V_{\bar F}\to V_{\bar F}$, у которого будет тот же минимальный и характеристический многочлен.
Например, это можно сделать, выбрав базис в $V$, оно превращается в $F^n$, потом взять $V_{\bar F} = \bar F^n$ и в нем задать $\varphi_{\bar F}$ той же матрицей, что и $\varphi$.
Характеристический многочлен не изменится, потому что матрица та же самая, но надо пояснить, почему минимальный многочлен не изменится.
В этом случае есть общая конструкция для $V_{\bar F}$, которая определяется так, что неизменность минимального многочлена будет очевидна (можно и руками показать, выбрав базис $\bar F$ как векторного пространства над $F$).
После чего мы видим, что $f_\text{min}$ и $\chi_\varphi$ имеют одни и те же корни в $\bar F$, то есть $p$ имеет общий корень с каким-то $p_i$.
Но это не возможно.
Действительно, в силу их взаимной простоты, мы имеем $1 = u(t) p(t) + v(t) p_i(t)$.
И если у них есть общий корень, то после подстановки его в равенсто, справа будет ноль, а слева -- единица.
На этом победа.%
\footnote{В курсе алгебры вам расскажут как для любого поля $F$ и любого многочлена $g\in F[t]$ построить большее поле $L\supseteq F$ такое, что в нем $g$ раскладывается на линейные множители.
Как мы видим из доказательства, этого нам достаточно.
Остается аккуратно объяснить, почему не изменится минимальный многочлен и вы будете готовы доказать этот факт.}
\end{itemize}

\begin{claim}
[БД]
Пусть $\varphi\colon V\to V$ -- некоторый оператор над произвольным полем $F$.
Тогда $p\in \spec_F^I \varphi$ тогда и только тогда, когда $p$ делит $\chi_\varphi$.
\end{claim}

\subsection{Обобщение собственных и корневых подпространств}

\begin{definition}
Пусть $\varphi\colon V\to V$ -- линейный оператор над произвольным полем и $p\in \spec_F^I \varphi$.
Тогда определим корневое подпространство как:
\[
V^p = \{v\in V\mid \exists k\colon p^k(\varphi) v = 0\} = \bigcup_{k\geqslant 0}\ker p^k(\varphi)
\]
и собственное подпространство
\[
V_p =\{v\in V \mid p(\varphi)v = 0\} = \ker p(\varphi)
\]
\end{definition}

\paragraph{Замечания}

\begin{itemize}
\item В силу леммы о стабилизации $V^p = \ker p^k(\varphi)$ для некоторого достаточно большого $k\leqslant \dim_F V$.

\item Это определение является обобщением корневого и собственного подпространства на случай идеального спектра.
Действительно, если $\lambda\in\spec_F \varphi$, то в идеальном спектре ему соответствует $p(t) = t - \lambda$.
Тогда определения превращаются в те же самые, что были даны в предыдущих разделах.
\end{itemize}

\subsection{Теоремы о разложении}

\begin{claim}
\label{claim::CoprimeKernels}
Пусть $\varphi\colon V\to V$ -- линейный оператор над полем $F$ и пусть $p,q\in F[t]$ -- два взаимно простых многочлена.
Тогда 
\begin{enumerate}
\item $\ker p(\varphi)\cap \ker q(\varphi) = 0$.

\item Оператор $p(\varphi)|_{\ker q(\varphi)}$ существует и обратим.
\end{enumerate}
\end{claim}
\begin{proof}
(1)  Из того, что многочлены $p$ и $q$ взаимнопросты, по расширенному алгоритму Евклида, найдутся многочлены $u,v\in F[t]$ такие, что $1 = u(t)p(t) + v(t)q(t)$.
Пусть теперь $w$ -- вектор из пересечения, тогда
\[
w = u(\varphi) p(\varphi) w + v(\varphi) q(\varphi) w = 0
\]

(2) Так как операторы $p(\varphi)$ и $q(\varphi)$ коммутируют, то ядро $q(\varphi)$ инвариантно относительно $p(\varphi)$ по утверждению~\ref{claim::KerImInvar}.
А значит существует оператор ограничения.
Так как для операторов обратимость равносильна инъективности (утверждение~\ref{claim::OperatorInvert}), то нам достаточно показать, что $\ker q(\varphi)|_{\ker p(\varphi)} = 0$.
Но $\ker q(\varphi)|_{\ker p(\varphi)} = \ker q(\varphi) \cap \ker p(\varphi) = 0$.
\end{proof}

\begin{claim}
\label{claim::IdealRootDec}
Пусть $\varphi\colon V\to V$ -- линейный оператор над полем $F$, $f\in F[t]$ зануляющий многочлен такой, что $f = p q \in F[t]$, где $(p, q) = 1$.
Тогда
\begin{enumerate}
\item $\ker p(\varphi) = \Im q(\varphi)$.

\item $\ker q(\varphi) = \Im p(\varphi)$.

\item $V = \ker p(\varphi) \oplus \ker q(\varphi)$.

\item Для любого $\varphi$ инвариантного подпространства $U\subseteq V$ найдутся два инвариантных подпространства $W \subseteq \ker p(\varphi)$ и $E\subseteq \ker q(\varphi)$ такие, что $U = W \oplus E$.
Более того подпространства $W$ и $E$ можно восстановить одним из двух способов:
\begin{enumerate}
\item $W = U \cap \ker p(\varphi)$ и $E = U\cap \ker q(\varphi)$.

\item $W = \pi_1(U)$, где $\pi_1\colon V\to V$ -- проектор на $\ker p(\varphi)$ вдоль $\ker q(\varphi)$.
И аналогично для $E = \pi_2(U)$, где $\pi_2\colon V\to V$ -- проектор на $\ker q(\varphi)$ вдоль $\ker p(\varphi)$.
\end{enumerate}

\item Если $f$ -- минимальный многочлен для $\varphi$, то  многочлен $p$ будет минимальным для оператора $\varphi|_{\ker p(\varphi)}$.

\end{enumerate}
\end{claim}
\begin{proof}
В начале сделаем некие общие подготовительные работы.
Из того, что многочлены $p$ и $q$ взаимнопросты, по расширенному алгоритму Евклида, найдутся многочлены $u,v\in F[t]$ такие, что $1 = a(t)p(t) + b(t)q(t)$.
Подставим в это равенство и в $f$ оператор $\varphi$, получим два равенства
\begin{gather*}
\Identity = a(\varphi) p(\varphi) + b(\varphi) q(\varphi)\\
0 = p(\varphi) q(\varphi)
\end{gather*}

(1) и (2).
Так как утверждения~(1) и~(2) симметричны, то достаточно доказать одно из них.

В начале покажем, что $\Im q(\varphi)\subseteq \ker p(\varphi)$.
Так как $p(\varphi)q(\varphi) = 0$, то для любого $v\in V$ верно, что $p(\varphi)q(\varphi)v = 0$, но это означает, что $q(\varphi)v \in\ker p(\varphi)$, то есть $\Im q(\varphi)\subseteq \ker p(\varphi)$.

Наоборот, возьмем $v\in \ker p(\varphi)$ и применим к нему первое операторное равенство, получим
\[
v = a(\varphi) p(\varphi) v + b(\varphi) q(\varphi) v = b(\varphi) q(\varphi) v =   q(\varphi) b(\varphi) v\in \Im q(\varphi)
\]

(3) Из взаимной простоты $p$ и $q$ следует, что $\ker p(\varphi) \cap \ker q(\varphi) = 0$ по утверждению~\ref{claim::CoprimeKernels}.
Значит сумма этих подпространств прямая, то есть $\ker p(\varphi) \oplus \ker q(\varphi) \subseteq V$.
Чтобы показать равенство, возьмем произвольный $v\in V$ и рассмотрим
\[
v = a(\varphi) p(\varphi) v + b(\varphi) q(\varphi) v = p(\varphi) a(\varphi) v + q(\varphi) b(\varphi) v\in \Im p(\varphi) +\Im q(\varphi) = \ker q(\varphi) + \ker p(\varphi)
\]
Здесь последнее равенство выполнено в силу предыдущих двух пунктов.

% Изменить доказательство на новое
(4) Пусть $U\subseteq V$ инвариантное подпространство.
Давайте определим подпространства $W$ и $E$ следующим образом: 
\[
W = \{q(\varphi) u \mid u\in U\} = q(\varphi) U\subseteq \Im q(\varphi) = \ker p(\varphi)
\quad \text{и}\quad
E = \{p(\varphi) u \mid u\in U\} = p(\varphi) U \subseteq \Im p(\varphi) = \ker q(\varphi)
\]
В частности $W\cap E = 0$.
Ясно, что построенные подпространства будут $\varphi$ инвариантны.
Теперь проверим, что $U = W + E$, для этого применим операторное равенство
\[
\Identity = a(\varphi) p(\varphi) + b(\varphi) q(\varphi)
\]
к вектору $u\in U$ и получим
\[
u = a(\varphi) p(\varphi)u + b(\varphi) q(\varphi)u
\]
Но тогда 
\[
a(\varphi) p(\varphi)u =  p(\varphi)(a(\varphi)u) \in p(\varphi)(U) = E
\]
Аналогично проверяется, что второе слагаемое лежит в $W$, что и доказывает, что $U = W\oplus E$.
Теперь покажем, почему слагаемые $W$ и $E$ восстанавливаются двумя способами.
Если $U = W \oplus E \subseteq \ker p(\varphi) \oplus \ker q(\varphi)$, то легко видеть, что 
\[
U \cap \ker p(\varphi) = W = \pi_1(U)
\quad\text{и}\quad
U \cap \ker q(\varphi) = E = \pi_2(U)
\]

(5) Теперь мы считаем, что $f = f_\text{min}$ для $\varphi$ на $V$.
Заметим, что
\[
p(\varphi|_{\ker p(\varphi)}) = p(\varphi)|_{\ker p(\varphi)} = 0
\]
Значит $p$ зануляет $\varphi|_{\ker p(\varphi)}$.
Так как минимальный многочлен обязательно делит $p$, то нам надо показать, что никакой делитель $p$ отличный от $p$ не зануляет $\varphi|_{\ker p(\varphi)}$.
Предположим противное, пусть $p_0 | p$ и $p_0(\varphi|_{\ker p(\varphi)}) = 0$.
Тогда рассмотрим многочлен $g = p_0 q$ и покажем, что $g(\varphi) = 0$.
Так как $V = \ker p(\varphi) \oplus \ker q(\varphi)$, то нам достаточно показать, что  $g(\varphi)$ действует нулем на любом векторе из $\ker p(\varphi)$ и на любом векторе из $\ker q(\varphi)$.
Но на $\ker q(\varphi)$ нулем действует $q(\varphi)$, а $g(\varphi) = p_0(\varphi)q(\varphi)$.
А на $\ker p(\varphi)$ оператор $p_0(\varphi)$ действует нулем по выбору $p_0$, а значит и $g(\varphi) = q(\varphi) p_0(\varphi)$ действует нулем.
То есть многочлен $g$ зануляет $\varphi$ и имеет степень меньше, чем $f_\text{min}$, противоречие.
\end{proof}

\begin{claim}
\label{claim::GenRootDec}
Пусть $\varphi\colon V\to V$ -- линейный оператор, $f_\text{min}$ -- его минимальный многочлен.
Пусть
\[
f_\text{min} = p_1^{k_1} \ldots p_r^{k_r}
\]
разложение минимального в неприводимые многочлены.
Тогда
\begin{enumerate}
\item  $V^{p_i} = \ker p_i^{k_i}(\varphi)$ причем $k_i$ -- минимальное такое $k$ для которого выполнено равенство $V^{p_i} = \ker p_i^{k}(\varphi)$.

\item $V = V^{p_1}\oplus \ldots \oplus V^{p_r}$.

\item Любое инвариантное подпространство $U\subseteq V$ имеет вид $U = U_1\oplus \ldots \oplus U_r$, где $U_i \subseteq V^{p_i}$ -- произвольные инвариантные подпространства.
\end{enumerate}
\end{claim}
\begin{proof}
1) Рассмотрим разложение многочлена $f_\text{min}$ на следующие множители
\[
f_\text{min} = \underbrace{p_1^{k_1}}_{p}\underbrace{p_2^{k_2}\ldots p_r^{k_r}}_{q}
\]
Тогда $V = \ker p(\varphi) \oplus \ker q(\varphi)$, то есть
$V = \ker p_1^{k_1}(\varphi) \oplus \ker q(\varphi)$.
По определению $\ker p_1^{k_1}(\varphi) \subseteq V^{p_1}$.
Если $\ker p_1^{k_1}(\varphi) \neq V^{p_1}$, то $V^{p_1}$ обязано пересекать $\ker q(\varphi)$ не по нулю.
По лемме о стабилизации (утверждение~\ref{claim::StabilityLemma}) найдется такое $N$, что $V^{p_1} = \ker p_1^{N}(\varphi)$.
Так как $p_1$ и $q$ взаимнопросты, то из пункта~(1) утверждения~\ref{claim::CoprimeKernels}, следует $\ker p_1^{N}\cap \ker q(\varphi) = 0$, противоречие.
Но теперь пункт~(5) предыдущего утверждения~\ref{claim::IdealRootDec} гласит, что так как $f_\text{min}$ был минимальным для $\varphi$ на всем пространстве $V$, то $p_1^{k_1}$ является минимальным многочленом для $\varphi$ ограниченным на $V^{p_1}$, то есть ни в какой меньшей степени $k$, чем $k_1$ мы не получим равенство $V^{p_1} = \ker p_1^k(\varphi)$.


2) Имея разложением
\[
f_\text{min} = \underbrace{p_1^{k_1}}_{p}\underbrace{p_2^{k_2}\ldots p_r^{k_r}}_{q}
\]
мы только что получили разложение
\[
V = \ker p_1^{k_1}(\varphi) \oplus \ker q(\varphi)
\]
Подпространство $\ker q(\varphi)$ является $\varphi$ инвариантным, так как $\varphi$ и $q(\varphi)$ коммутируют (утверждение~\ref{claim::KerImInvar}).
Пусть $\psi$ -- ограничение $\varphi$ на $\ker q(\varphi)$.
Так как мы рассматривали минимальный многочлен для $\varphi$ на всем пространстве $V$, то $q$ будет минимальным для $\psi$ на $\ker q(\varphi)$ (пункт~(5) утверждение~\ref{claim::IdealRootDec}).
Тогда мы можем индукцией по количеству неприводимых получить разложение
\[
\ker q(\varphi) = \ker p_2^{k_2}(\varphi) \oplus \ldots \oplus \ker p_r^{k_r}(\varphi)
\]
Объединяя это с результатом первого пункта мы получаем разложение $V = V^{p_1}\oplus \ldots \oplus V^{p_r}$.

3) Это непосредственно следует из утверждения~\ref{claim::IdealRootDec} пункт~(4) индукцией по количеству прямых слагаемых.
\end{proof}
