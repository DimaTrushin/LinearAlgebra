\ProvidesFile{lecture11.tex}[Лекция 11]


Теперь начнем реализовывать этот план доказательства.
Начнем со следующего.

\begin{claim}
\label{claim::PolyDisk}
Пусть $p\in \mathbb C[x]$ -- произвольный многочлен отличный от константы.
Тогда для любого $c > 0$ найдется $r > 0$, что $|p(z)| > c$ при $|z| > r$.
\end{claim}
\begin{proof}
Пусть $p(z) = a_0 + a_1 z + \ldots + a_n z^n$ и $a_n\neq 0$.
Тогда
\[
p(z) = a_n z^n \left(1 + \frac{a_{n-1}}{a_n z} + \ldots + \frac{a_0}{a_n z^n}\right) = a_n z^n (1 + \omega(z))
\]
Фиксируем произвольное положительное число $r > 1$ и рассмотрим $|z| > r$.
Тогда 
\[
|\omega(z)| \leqslant \left|\frac{a_{n-1}}{a_n z}\right| + \ldots + \left|\frac{a_0}{a_n z^n}\right|\leqslant \left|\frac{a_{n-1}}{a_n }\right|\frac{1}{r} + \ldots + \left|\frac{a_0}{a_n }\right|\frac{1}{r^n}\leqslant  \left(\left|\frac{a_{n-1}}{a_n }\right| + \ldots + \left|\frac{a_0}{a_n }\right|\right)\frac{1}{r} 
\]
Последнее выражение  обозначим за $\delta(r)$, оно идет к нулю при $r\to \infty$.
Давайте теперь оценим вне этого диска значение $|p(z)|$:
\[
|p(z)| = |a_n z^n(1+\omega(z))| =|a_n| |z|^n |1 + \omega(z)|\geqslant |a_n| |z|^n (1 - |\omega(z)|)\geqslant |a_n| |r|^n(1 - \delta(r))\to \infty,\text{ при } r\to \infty
\]
То есть мы сможем найти $r$ при котором вне диска $\{z\in \mathbb C\mid |z| \leqslant r\}$ будет выполняться $|p(z)| > c$.
\end{proof}

\begin{claim}
\label{claim::PolyMin}
Пусть $p\in \mathbb C[x]$ -- произвольный многочлен, тогда отображение $|p|\colon \mathbb C\to \mathbb R$ заданное по правилу $z\mapsto |p(z)|$ достигает минимума, то есть найдется такая точка $z_0\in\mathbb C$, что $|p(z_0)|\leqslant |p(z)|$ для любого $z\in \mathbb C$.
\end{claim}
\begin{proof}
Идея доказательства этого утверждения следующая.
Пусть $c = |p(0)|$.
Если это ноль, то мы нашли наш минимум.
Пусть $c\neq 0$, тогда давайте найдем диск $D_r(0)$ с центром в нуле и радиуса $r$ такой, что $|p(z)| > c$ для всех $z\notin D_r(0)$.
Тогда, по утверждению~\ref{claim::PolyDisk} мы можем найти диск $D_r(0)= \{z\in\mathbb C\mid |z|\leqslant r\}$, вне которого $|p(z)|>c$.
А значит, если мы найдем минимум для $|p(z)|$ на диске $D_r(0)$ он автоматически будет минимумом в $\mathbb C$.
Действительно, внутри диска в этой точке мы будем принимать наименьшее значение, в частности значение будет не больше $c$.
Но вне диска мы не можем принять значение меньше, так как там мы строго больше $c$.

Теперь надо найти минимум внутри диска $D_r(0)$.
Давайте я приведу два доказательства: идейное и доказательство в лоб.
Идейное доказательство такое.
Функция $\phi\colon \mathbb C\to \mathbb R$ по правилу $z\mapsto |p(z)|$ есть композиция двух отображений: полиномиального $p\colon \mathbb C\to \mathbb C$ и модуля $|{-}|\colon \mathbb C\to\mathbb R$ по правилу $z\mapsto |z|$.
Оба эти отображения непрерывны, а значит и отображение $z\mapsto |p(z)|$ тоже непрерывно.
Кроме того, диск $D_r(0)$ является компактом, а любое непрерывное отображение на компакте достигает минимума.%
\footnote{Если вы сейчас пребываете в шоке, то это нормально.
Сейчас я исправлюсь и напишу простое доказательство, но он будет несколько длиннее.}
\end{proof}

\paragraph{Минимум на диске (по простому?)}

Чтобы найти минимум на диске, мне придется пользоваться фактами из математического анализа, ведь свойства комплексных чисел должны зависеть от особенностей их природы, которая не чисто алгебраическая.
Нам понадобится следующий факт.%
\footnote{Искренне надеюсь, что с ним вы знакомы.
Его можно считать одной из аксиом вещественных чисел.}
\begin{claim*}
[БД]
У любой последовательности на отрезке найдется сходящаяся подпоследовательность.
То есть для любой $a_n\in [a, b]$ найдется подпоследовательность $a_{n_k}$ такая, что существует
\[
\lim_{k\to \infty}a_{n_k}\in [a, b]
\]
\end{claim*}

Рассмотрим функцию $f = |p|\colon D_r(0)\to \mathbb R$ по правилу $f(z) = |p(z)|$.
Так как эта функция ограничена снизу нулем, то существует нижняя грань
\[
a = \inf_{z\in D_r(0)} f(z)
\]
По определению нижней грани, мы можем выбрать последовательность $z_n\in D_r(0)$ такую, что $f(z_n) \to a$.
Такая последовательность обязательно имеет вид $z_n = a_n + i b_n$, где $a_n, b_n\in [-r, r]$ -- последовательности вещественных чисел на отрезке.
Из них мы по очереди можем выбрать сходящиеся подпоследовательности $a_{n_k}$ и $b_{n_k}$, так что последовательность $z_{n_k}$ сходится в $D_r(0)$ к какой-то точке $z_0$.
А значит
\[
a = \lim_{k\to \infty} f(z_{n_k}) = \lim_{k\to \infty} |p(z_{n_k})| = \left|\lim_{k\to \infty} p(z_{n_k})\right| = \left|p\left(\lim_{k\to \infty} z_{n_k}\right)\right| = |p(z_0)|
\]
Давайте объясним все переходы.
Первый -- это определение нашей последовательности, мы по ней подбираемся к инфимуму.
Второй -- это непрерывность модуля, то есть для комплексного числа верно $\lim_{n\to \infty}|z_n| = |\lim_{n\to \infty} z_n|$.
Действительно, ведь $|a + bi| = \sqrt{a^2 + b^2}$ и функция корня от суммы квадратов непрерывна, то есть предел в точке равен ее значению в точке.
Третий переход следует из непрерывности многочлена $p(z)$.
Действительно, такой многочлен -- это сумма произведений мнимых и вещественных частей с коэффициентами, а в таких функциях мы тоже умеем переходить к пределу.
Последнее равенство -- это возможность взять предел у выбранной подпоследовательности.
Таким образом в точке $z_0$ достигается нижняя грань на диске $D_r(0)$, а значит это точка минимума.

\paragraph{Минимум обязан быть нулем}

Теперь осталось доказать, что минимум для функции $|p(z)|$ (если $p$ не константа) обязательно должен быть нулем.
Точнее мы покажем, что если $|p(z_0)|\neq 0$, то обязательно найдется точка с еще меньшим модулем, то есть найдется $z_1\in \mathbb C$, что $|p(z_1)|< |p(z_0)|$.
Но в начале нам понадобится следующий пример.

\begin{example}
[V.I.P. пример]
Рассмотрим $p(z) = z^d$ и посмотрим на функцию $p\colon \mathbb C\to \mathbb C$ по правилу $z \mapsto p(z) = z^d$.
Давайте рассмотрим окружность $z(t) = r e^{2\pi i t}$ для $t\in [0, 1]$.
Когда $t$ пробегает от $0$ до $1$, то $z(t) = r e^{2\pi i t}$ пробегает по окружности радиуса $r$ один оборот против часовой стрелки.
Давайте посмотрим на образ этой окружности под действием $p$, получим $p(z(t)) = r^d e^{2\pi i d t}$.
То есть теперь, когда $t$ пробегает от $0$ до $1$ мы пробегаем окружность радиуса $r^d$ но уже $d$ раз против часовой стрелки (делаем $d$ оборотов вместо одного).
% TO DO
% Надо нарисовать картинку!!
\end{example}


\begin{claim}
\label{claim::PolyMinIsZero}
Пусть $p\in\mathbb C[x]$ -- произвольный не константный многочлен и $z_0\in \mathbb C$ такая точка, что $p(z_0)\neq 0$.
Тогда найдется точка $z_1\in \mathbb C$ такая, что $|p(z_1)|<|p(z_0)|$.
\end{claim}
\begin{proof}
Если точка $z_0$ не является точкой минимума для $|p|$, то нужная точка $z_1$ найдется по определению.
То есть мы можем предположить, что $z_0$ -- точка минимума.
Давайте определим многочлен $g(z) = p(z_0 + z)$.
Тогда у многочлена $g(z)$ то же множество значений, что и у $p(z)$, но у него точка $0$ является точкой минимума.
В свою очередь $g$ представляется в виде
\[
g(z) = a_0 + a_r z^r + a_{r+1}z^{r+1} + \ldots + a_n z^n
\]
Здесь выше $a_0$ -- это значение многочлена $g$ в нуле, которое по условию не ноль.
Число $a_r$ -- это первый ненулевой коэффициент после $a_0$ в многочлене $g$.
Учтите, что $r$ может быть $1$, а может быть больше $1$.
Давайте перепишем многочлен $g$ следующим образом:
\[
g(z) = a_0 + a_r z^r\left(1 + \frac{a_{r+1}}{a_r}z + \ldots + \frac{a_n}{a_r} z^{n-r}\right) = a_0 + a_r z^r\left(1 + \omega(z) \right) = a_0 + g_0(z) + \omega(z) g_0(z)
\]
где
\[
\omega(z) =  \frac{a_{r+1}}{a_r}z + \ldots + \frac{a_n}{a_r} z^{n-r}\quad \text{и} \quad g_0(z) = a_rz^r
\]

Давайте в начале посмотрим на многочлен $h(z) = a_0 + a_r z^r = a_0 + g_0(z)$ и покажем что для  него найдется точка $z_1$ такая, что $|h(z_1)| < |h(0)|$.
Если $|z| = \delta$, то $z^r$ описывает окружность радиуса $\delta^r$ вокруг нуля и делает $r$ оборотов.
Выражение $a_r z^r$ описывает окружность радиуса $R = |a_r|\delta^r$ вокруг нуля и делает $r$ оборотов.%
\footnote{Начальная точка для окружности имеет аргумент равный аргументу $a_r$.}
А значит $h(z) = a_0 + a_r z^r$ описывает окружность радиуса $R$ вокруг точки $a_0$ и делает $r$ оборотов.
При малых $\delta$ эта окружность пересекается с радиус вектором $a_0$ в некоторой точке $m$.
Тогда решая уравнение $m = a_0 + a_r z^r$,%
\footnote{Это мы можем сделать, так как тут задача про извлечение корня из числа $(m - a_0) / a_r$.}
мы найдем точку $z_1$ такую, что $h(z_1) = m$.
Но по построению точка $m$ ближе к нулю, чем $a_0$.
Действительно, а это и значит, что 
\[
|h(z_1)| = |a_0| - R < |a_0| = |h(0)|
\]

Теперь я хочу показать, что $|g(z_1)| < |g(0)|$ при малых $\delta$.
Главная идея заключается в том, что $g$ и $h$ имеют близкие значения если $\delta$ достаточно мало.
Давайте для начала оценим $\omega(z)$, когда $|z| = \delta < 1$:
\[
|\omega(z)| \leqslant \left| \frac{a_{r+1}}{a_r}\right| \delta + \ldots + \left|\frac{a_n}{a_r}\right| 
\delta^{n-r}\leqslant \left(\left| \frac{a_{r+1}}{a_r}\right|  + \ldots + \left|\frac{a_n}{a_r}\right|\right) 
\delta = C \delta
\]
Последнее неравенство следует из того, что $\delta < 1$.
А через $C$ мы обозначили полученную константу.
Значит при $\delta < \varepsilon / C$ мы можем считать, что $|\omega(z)| < \varepsilon$.

Теперь посмотрим на значение $|g(z_1)|$:
% Нужна картинка
\[
|g(z_1)| \leqslant |h(z_1)| + |\omega(z_1)| |a_rz^r| = |a_0| - R + |\omega(z_1)| R = |a_0| - (1 - |\omega(z_1)|)R < |a_0| - (1 - \varepsilon) R = |g(0)| - (1 - \varepsilon) R
\]
Таким образом мы видим, что при малых $\delta$ значение $|g(z_1)|$ строго меньше $|g(0)|$.
Что и требовалось.
\end{proof}

\begin{proof}
[Доказательство основной теоремы алгебры]
Давайте я проговорю строго доказательство утверждения~\ref{claim::CAlgClosed} от начала и до конца пользуясь доказанными выше результатами.

Пусть $f \in \mathbb C[z]\setminus\mathbb C$ -- произвольный неконстантный многочлен.
Тогда по утверждению~\ref{claim::PolyMin}, отображение $|f|\colon \mathbb C\to \mathbb R_+$ достигает своего минимума.
Пусть точка минимума $z_0$.
Тогда $|f(z)| \geqslant |f(z_0)| \geqslant 0$ для всех $z\in \mathbb C$.
Если $f(z_0) = 0$, то мы нашли корень у многочлена $f$.
Потому можно считать, что $|f(z_0)| > 0$.
Тогда по утверждению~\ref{claim::PolyMinIsZero} найдется точка $z_1\in\mathbb C$ такая, что $|f(z_1)| < |f(z_0)|$, что противоречит тому, что $z_0$ была точкой минимума.
А значит такого быть не может, что $|f(z_0)| \neq 0$ и теорема доказана.
\end{proof}

\subsection{Многочлены}

В этом разделе я хочу сказать пару слов про многочлены.
Пусть $F$ -- некоторое поле.
Тогда многочлен над полем $F$ -- это картинка вида $f = a_0 + a_1 x + \ldots + a_n x^n$, где $a_i\in F$.
Формально, такая картинка -- это конечная последовательность чисел $(a_0,\ldots,a_n)$, но психологически лучше и правильнее думать именно про картинки.
Еще можно для краткости писать $f = \sum_{k\geqslant 0}a_k x^k$, подразумевая, что в этой сумме только конечное число ненулевых коэффициентов.
Это удобное соображение позволяет удобно записать правила для сложения и умножения многочленов, которые определяются следующим образом
\[
\Bigl( \sum_{k\geqslant 0}a_k x^k\Bigl) + \Bigl( \sum_{k\geqslant 0}b_kx^k\Bigl) =  \sum_{k\geqslant 0}(a_k+b_k) x^k\quad\text{ и }\quad 
\Bigl( \sum_{k\geqslant 0}a_k x^k\Bigl)\Bigl( \sum_{k\geqslant 0}b_k x^k\Bigl) =  \sum_{k\geqslant 0}\Bigl(\sum_{m+n = k} a_m b_n\Bigl) x^k
\]
Таким образом многочлен -- это не функция, а картинка.
Однако, каждый многочлен $f\in F[x]$ задает функцию $F\to F$ по правилу $x\mapsto f(x)$.
Но в случае конечных полей (то есть полей из конечного числа элементов) разные многочлены могут давать одни и те же функции.
Напомню, что степень многочлена $f$ -- это наибольший номер $n$, что коэффициент $a_n\neq 0$.%
\footnote{По-хорошему надо еще аккуратно определить степень нулевого многочлена.
Но ее обычно определяют по ситуации так, как удобнее.
Например можно положить $-1$ или $-\infty$ и есть еще пара способов.
Но об этом можно особенно не запариваться.}

\paragraph{Примеры}

\begin{itemize}
\item Пример конечного поля.

Пусть $p\in \mathbb Z$ -- некоторое простое число.
Обозначим через $\mathbb Z_p$ множество остатков по этому числу, то есть $\mathbb Z_p = \{0, 1, \ldots, p-1\}$.
Введем на этом множестве операции сложения и умножения по модулю простого числа $p$, то есть
\[
a + b = a + b\pmod p\quad \text{и}\quad a b = ab \pmod p
\]
Тогда можно проверить, что $Z_p$ является полем, где числа $0$ и $1$ являются нулем и единицей поля.
Единственная аксиома, которая требует усилий -- показать, что любой ненулевой элемент $Z_p$ обратим.
Давайте возьмем произвольный ненулевой элемент $a\in \mathbb Z_p$.
Так как $ a < p$ и $p$ -- простое число, то $(a, p) =1$.
По расширенному алгоритму евклида найдутся целые числа $u, v\in\mathbb Z$ такие, что $1 = u a + vp$.
Рассмотрим это равенство по модулю простого числа $p$ и получим, что
$u a = 1 \pmod p$, а это и означает, что $u$ является обратным к $a$ по умножению.

\item Пример, когда разные многочлены дают одну и ту же функцию.

Рассмотрим многочлены $\mathbb Z_2[x]$.
Тогда $\mathbb Z_2$ состоит только из $0$ и $1$.
В этом случае все многочлены $x^n$ задают одну и ту же функцию.
\end{itemize}

\begin{definition}
 Пусть $F$ -- произвольное поле и $f\in F[x]$ -- некоторый многочлен.
 Если число $a\in F$ является его корнем, то $f$ делится на $x - a$, а значит представляется в виде $f(x) = (x - a) g(x)$ для некоторого $g\in F[x]$.
 Аналогично, если $a$ является корнем $g$, то можно выделить $(x - a)$ и в $g$ и так далее.
 В итоге можно найти разложение $f(x) = (x - a)^k g(x)$, где $g(a) \neq 0$.
 В этом случае говорят, что $k$ -- это кратность корня $a$ в многочлене $f$.
 Корень кратности $1$ называется простым.
\end{definition}


\begin{definition}
Пусть $f\in F[x]$ имеет вид $f = a_0 + a_1 x + \ldots + a_n x^n$.
Определим формальную производную следующим образом $f' = a_1 + 2a_2 x+\ldots + na_n x^{n-1}$ или по-другому $f'=\sum_{k\geqslant 1} k a_k x^{k-1}$.
\end{definition}

Несложно убедиться, что, определив таким образом производную, она удовлетворяет всем естественным свойствам, к которым мы привыкли в анализе.
В качестве упражнения предлагается проверить следующее.

\begin{claim}
Пусть $F$ -- произвольное поле.
Для формальной производной выполнены следующие свойства:
\begin{enumerate}
\item $(f + g)' = f' + g'$ для любых $f,g\in F[x]$.

\item $(\lambda f )' = \lambda f'$ для любых $\lambda\in F$ и $f\in F[x]$.

\item $(fg)' = f' g + fg'$ для любых $f,g\in F[x]$.

\item $f(g(x))' = f'(g(x)) g'(x)$ для любых $f,g\in F[x]$.
\end{enumerate}
\end{claim}


С помощью формальной производной можно проверить кратность корня в произвольном многочлене.
Для начала нам нужно следующее вспомогательное утверждение.

\begin{claim}
Пусть $F$ -- произвольное поле, $f\in F[x]$ -- некоторый многочлен и $a\in F$ -- его корень кратности $k$.
Тогда 
\begin{enumerate}
\item Число $a$ является корнем кратности хотя бы $k-1$ в многочлене $f'$.

\item Если число $k 1 \neq 0$ в $F$, то $a$ является корнем кратности в точности $k - 1$ в многочлене $f'$.
\end{enumerate}
\end{claim}
\begin{proof}
1) По определению имеем $f = (x - a)^k g(x)$ причем $g(a) \neq 0$.
Возьмем производную от $f$, получим
\[
f' = k(x-a)^{k-1}g(x) + (x-a)^k g'(x) = (x-a)^{k-1}(kg(x) + (x-a)g'(x))
\]
и мы видим, что у производной $a$ имеет кратность хотя бы $k-1$.


2) Давайте поймем, когда кратность может вырасти.
Только если множитель $(kg(x) + (x-a)g'(x))$ зануляется в $a$.
Если подставить $a$, то получим $kg(a)$.
Число $g(a)\neq 0$ по выбору, но если $k 1\neq 0$, то и их произведение не ноль в поле $F$, а это будет означать, что кратность корня в точности $k - 1$.
\end{proof}

\paragraph{Примеры и замечания}

\begin{enumerate}
\item
Давайте продемонстрируем ситуацию, когда кратность корня может возрасти.
Например, выберем $F = \mathbb Z_p$ и в качестве многочлена $h$ рассмотрим $x^p - 1$.
Тогда $h' = 0$.
Теперь положим $f = xh(x) = x^{p+1}- x$.
Тогда $f' = h(x) = x^p - 1$.
С другой стороны $x^p - 1 = (x-1)^p$, а значит $1$ имеет кратность $p$ в многочлене $f$.
Но и в многочлене $f'$ $1$ имеет кратность $p$.

\item Если для любого натурального числа $k\in \mathbb N$ в поле $F$ выполнено, $k 1 \neq 0$, то можно следующим образом проверить корень многочлена $f\in F[x]$ на кратность.
Если $a\in F$ -- некоторый корень.
Надо посмотреть на $f'(a)$.
Если это число ноль, то $a$ корень кратности больше $1$, а если не ноль, то кратности в точности $1$.

\item Если $F$ произвольное поле, то общий алгоритм проверки корня на простоту следующий.
Надо взять многочлен $f\in F[x]$, для которого $a$ является корнем.
Посчитать производную $f'$, потом посчитать нод $d(x) = (f, f')$.
Если $d(a) = 0$, то $a$ кратный корень, если $d(a) \neq 0$, то это корень кратности $1$.%
\footnote{Я не буду останавливаться на доказательствах этих фактов.
Все они вам встретятся в курсе алгебры.}
\end{enumerate}


\newpage
\section{Векторные пространства}

\subsection{Идея и определение}

\paragraph{Идея}

Мы с вами до этого изучали много разных объектов, которые не сильно похожи друг на друга.
Например, вектор-столбцы $F^n$, матрицы $\operatorname{M}_{m\,n}(F)$, функции $f\colon X\to F$, многочлены $F[x]$.
Все эти товарищи нам постоянно встречаются и каждый раз приходится для каждого из них все доказывать заново и во время доказательств мы видим, что наши рассуждения повторяются.
Это означает, что на самом деле у всех этих объектов есть некий общий интерфейс, через который мы на самом деле с ним работаем.
Самое главное в этом интерфейсе то, что мы можем брать элементы из этих объектов, умножать эти элементы на числа и складывать между собой.
Абстрактное векторное пространство как раз и формализует идею такого общего интерфейса, через который в множестве можно складывать элементы и умножать на числа.


У такого подхода есть несколько плюсов.
Во-первых, формальное удобство: как только вы что-то сделали для абстрактного векторного пространства и увидели, что что-то конкретное является таковым, то все ваши достижения автоматом применимы в этой конкретной ситуации.
Общий алгоритм для векторного пространства будет одинаково хорошо работать и для столбцов, и для матриц, и для функций и т.д.
Во-вторых, есть менее очевидный бонус.
Когда мы доказываем что-то про абстрактное векторное пространство, то про него надо думать как про $F^n$.
Это поможет вам не потеряться в формализме и догадаться, что откуда берется.
Неформально это означает, что если вы что-то умеете делать для $F^n$, то это автоматически верно для любого векторного пространства!
Формально это не совсем правда, но в классе хороших пространств это так.%
\footnote{Под хорошими тут подразумеваются конечно мерные.}
Тем не менее, даже в классе всех пространств, интуиция из $F^n$ очень полезна.

\paragraph{Определение}

Следующее определение -- это пример определения с контекстом.
Это означает, что прежде, чем его дать, вы должны зафиксировать некоторую информацию, которая необходима для вашего определения и без этой информации оно -- бессмысленный мусор.
У определения векторного пространства в качестве такого контекста выступает некоторое поле $F$.
Это значит, что пока вы не зафиксировали какое-то поле, вы не можете говорить о векторных пространствах над полем $F$, а <<просто векторных пространств>> без указания какого-либо поля не существует.

\begin{definition}
\label{def::VectorSpace}
Пусть $F$ -- некоторое фиксированное поле.
Тогда векторное пространство над полем $F$ -- это следующий набор данных $(V, +, \cdot)$, где
\begin{itemize}
\item $V$ -- множество.
Элементы этого множества будут называться векторами.

\item $+\colon V \times V \to V$ -- бинарная операция, то есть правило, действующее так: $(v,u)\mapsto v + u$, где $u,v \in V$.

\item $\cdot \colon F \times V \to V$ -- бинарная операция, то есть правило, действующее так: $(\alpha, v)\mapsto \alpha v$, где $\alpha \in F$ и $v\in V$.
\end{itemize}
При этом эти данные удовлетворяют следующим $8$ аксиомам:
\begin{enumerate}
\item {\bf Ассоциативность сложения} Для любых векторов $u,v,w\in V$ верно $(u+v) + w = u + (v+w)$.

\item {\bf Существование нулевого вектора} Существует такой вектор $0\in V$, что для любого $v\in V$ выполнено $0 + v = v + 0 = v$.

\item {\bf Существование противоположного вектора} Для любого вектора $v\in V$ существует вектор $-v\in V$ такой, что $v + (-v) = (-v) + v = 0$.

\item {\bf Коммутативность сложения} Для любых векторов $u,v \in V$ верно $u + v = v + u$.

\item {\bf Согласованность умножения со сложением векторов} Для любого числа $\alpha \in F$ и любых векторов $u,v \in V$ верно $\alpha(v + u) = \alpha v + \alpha u$.

\item {\bf Согласованность умножения со сложением чисел} Для любых чисел $\alpha, \beta\in F$ и любого вектора $v\in V$ верно $(\alpha + \beta)v = \alpha v + \beta v$.

\item {\bf Согласованность умножения с умножением чисел} Для любых чисел $\alpha,\beta\in F$ и любого вектора $v\in V$ верно $(\alpha\beta)v = \alpha(\beta v)$.

\item {\bf Нетривиальность} Для любого $v\in V$ верно $1 v = v$.%
\footnote{Здесь $1\in F$.}
\end{enumerate}
\end{definition}

\paragraph{Примеры}

\begin{enumerate}
\item Поле $F$ (или кто больше привык к вещественным числам $\mathbb R$) является векторным пространством над $F$ (соответственно над $\mathbb R$).

\item Более обще, множество вектор-столбцов $F^n$ является векторным пространством над $F$.

\item Множество матриц $\operatorname{M}_{m\,n}(F)$ является векторным пространством над $F$.

\item Пусть $X$ -- произвольное множество, тогда множество функций $F(X, F) = \{f\colon X\to F\}$ является векторным пространством над $F$.
Надо лишь объяснить как складывать функции и умножать на элементы $F$.
Операции поточечные, пусть $f,g\colon X\to F$, тогда функция $(f+g)\colon X\to F$ действует по правилу $(f+g)(x) = f(x) + g(x)$.
Если $\alpha \in F$, то функция $(\alpha f)\colon X\to F$ действует по правилу $(\alpha f)(x) = \alpha f(x)$.

\item Множество многочленов $F[x] = \{a_0+a_1x + \ldots + a_n x^n\mid a_i \in F,\,n\in \mathbb Z_{\geqslant 0}\}$.
Тут надо обратить внимание, что мы подразумеваем под многочленом.
Для нас многочлен -- это НЕ функция, многочлен -- это картинка вида $a_0 + a_1 x + \ldots + a_n x^n$.%
\footnote{Для любителей формализма, можете считать, что многочлен -- это конечная последовательность элементов $F$ вида $(a_0,\ldots,a_n)$, но длина последовательности может быть любой, включая нулевую.}
Складываются и умножаются эти картинки по одинаковым правилам.
Важно, что две такие картинки равны тогда и только тогда, когда у них равные коэффициенты.
Множество всех многочленов $F[x]$ является векторным пространством над $F$.
\end{enumerate}

\paragraph{Замечание}

Стоит отметить, что в обычных векторных пространствах мы привыкли к некоторым свойствам, которые бы хотелось иметь и в общем случае.
Например, в $F^n$ есть единственный нулевой вектор, а аксиомы в общем случае говорят, что нулевой вектор лишь существует.
Однако, можно показать, что нулевой вектор автоматически единственный.
Давайте перечислим некоторые непосредственные следствия из аксиом:
\begin{enumerate}
\item Нулевой вектор единственный.
Действительно, если у нас два нуля $0_1$ и $0_2$, то рассмотрим $0_1 + 0_2$.
Так как $0_1$ является нулем, то по определению $0_1 + 0_2 = 0_2$.
С другой стороны, так как $0_2$ является нулем, то $0_1 + 0_2 = 0_1$.
А это и означает, что оба нуля совпадают.

\item Для любого $v\in V$ существует единственный $-v$.
Действительно, пусть $u, w\in V$ два вектора обратных к $v$.
Это значит, что выполнены равенства
\[
v + u = u + v = 0 \quad\text{и}\quad v + w = w + v = 0
\]
Но тогда
\[
w = 0 + w = (u + v) + w = u + (v + w) = u + 0 = u
\]

\item Для любого вектора $v\in V$ имеем $0 v = 0$.%
\footnote{Здесь слева $0$ -- это нулевой элемент поля, а справа $0$ -- это нулевой вектор.}
Начнем с равенства $0 + 0 = 0$ для нуля из $F$.
Раз два числа равны, то после умножения одного и того же вектора на них, результаты останутся одинаковыми.
Значит для произвольного $v\in V$ имеем $(0 + 0) v = 0 v$.
Раскроем скобки и прибавим к обеим сторонам равенства элемент обратный к $0 v$, получим
\[
0 v + 0 v = 0v \quad\Rightarrow\quad 0 v + 0 v + (-(0v)) = 0v + (-(0v))\quad\Rightarrow\quad 0v + 0 = 0 \quad\Rightarrow\quad 0 v = 0
\]

\item Для любого числа $\alpha \in F$ верно $\alpha 0 = 0$.%
\footnote{Здесь $0$ -- это нулевой вектор в обеих частях равенства.}
В этом случае доказательство аналогично предыдущему, надо лишь стартовать с равенства векторов, а не чисел $0 + 0 = 0$ в $V$.
Умножим эти векторы на одно и то же число $\alpha$, получим $\alpha(0 + 0) = \alpha 0$.
Опять раскроем скобки и прибавим $- (\alpha 0)$, получим:
\[
\alpha 0 + \alpha 0 = \alpha 0 \quad \Rightarrow\quad
\alpha 0 + \alpha 0 + (-(\alpha 0)) = \alpha 0 + (-(\alpha 0))\quad \Rightarrow\quad
\alpha 0 + 0 = 0 \quad \Rightarrow\quad \alpha 0 = 0
\]

\item Для любого вектора $v\in V$ верно $-v = (-1)v$.
Для этого рассмотрим равенство $1 + (-1) = 0$.
Умножим эти одинаковые числа на один и тот же вектор $v\in V$ и получим $(1 + (-1))v = 0$.
Теперь надо раскрыть скобки и прибавить $-v$
\[
1 v + (-1) v = 0\quad \Rightarrow \quad v + (-1) v = 0 \quad \Rightarrow \quad (-1) v = -v
\]
\end{enumerate}
