\ProvidesFile{lecture17.tex}[Лекция 17]


\subsection{Характеристики линейных операторов}

В этом разделе я перечислю основные характеристики, которые можно определить для любого линейного оператора.

\paragraph{След}

Перед определением докажем техническое утверждение.

\begin{claim*}
Пусть $V$ -- векторное пространство над полем $F$ и пусть $\varphi\colon V\to V$ -- некоторый линейный оператор.
Тогда число $\tr(A_\varphi)$ не зависит от базиса, в котором посчитана матрица $A_\varphi$.
\end{claim*}
\begin{proof}
Действительно, пусть у нас есть два базиса $e' = (e_1',\ldots,e_n')$ и $e = (e_1,\ldots,e_n)$ связанные матрицей перехода $e' = eC$.
Пусть $\varphi e = eA_\varphi$ и $\varphi e' = e' A_\varphi'$.
Тогда как мы видели выше $A_\varphi' = C^{-1} A_\varphi C$.
Тогда
\[
\tr(A_\varphi') = \tr (C^{-1}A_\varphi C) = \tr(A_\varphi C C^{-1}) = \tr(A_\varphi)
\]
\end{proof}

Положим по определению $\tr\varphi = \tr A_\varphi$ и будем называть это число следом оператора $\varphi$.
Это определение корректно, так как данное число не зависит от базиса, в котором считается матрица оператора.%
\footnote{Тут нужно сделать важное замечание.
Как мы видим след оператора определяется через его матрицу, но не зависит от матрицы, а зависит только от самого линейного оператора.
Потому есть соблазн дать эквивалентное определение совсем не используя матрицу оператора.
К сожалению так сделать невозможно.
Одной из причин является отсутствие следа в бесконечно мерных векторных пространствах.
Любые попытки дать <<без координатное>> определение следа на самом деле является лишь тщательной маскировкой его координатной природы.}

\paragraph{Определитель}

Перед определением докажем техническое утверждение.

\begin{claim*}
Пусть $V$ -- векторное пространство над полем $F$ и пусть $\varphi\colon V\to V$ -- некоторый линейный оператор.
Тогда число $\det(A_\varphi)$ не зависит от базиса, в котором посчитана матрица $A_\varphi$.
\end{claim*}
\begin{proof}
Действительно, пусть у нас есть два базиса $e' = (e_1',\ldots,e_n')$ и $e = (e_1,\ldots,e_n)$ связанные матрицей перехода $e' = eC$.
Пусть $\varphi e = eA_\varphi$ и $\varphi e' = e' A_\varphi'$.
Тогда как мы видели выше $A_\varphi' = C^{-1} A_\varphi C$.
Тогда
\[
\det(A_\varphi') = \det (C^{-1}A_\varphi C) = \det(C^{-1})\det(A_\varphi)\det(C) =\det(A_\varphi)
\]
\end{proof}

Положим по определению $\det\varphi = \det A_\varphi$ и будем называть это число определителем оператора $\varphi$.
Это определение корректно, так как данное число не зависит от базиса, в котором считается матрица оператора.%
\footnote{Здесь верно то же самое замечание, что и для следа.
Определитель оператора нельзя определить без матрицы оператора, но в то же время он не зависит от матрицы, а зависит лишь от самого оператора.}

\paragraph{Характеристический многочлен}

Пусть опять $\varphi\colon V\to V$ -- произвольный линейный оператор, тогда для любого $\lambda \in F$, $\lambda \Identity - \varphi\colon V\to V$ -- тоже линейный оператор.
Тогда по предыдущему определению корректно определен определитель такого оператора, который мы обозначим так: $\chi_\varphi (\lambda) = \det(\lambda \Identity - \varphi)$ и будем называть характеристическим многочленом оператора $\varphi$.

Пусть теперь в некотором базисе $\varphi$ имеет матрицу $A_\varphi$.
Тождественный оператор $\Identity$ в любом базисе задается единичной матрицей.
Тогда по предыдущему определению $\det(\lambda \Identity - 
\varphi)$ совпадает с $\det (\lambda E - A_\varphi)$.
То есть характеристический многочлен оператора -- это характеристический многочлен любой из его матриц в каком-нибудь базисе (в силу корректности определения определителя оператора, все эти многочлены будут одинаковыми).

Есть другой способ смотреть на характеристический многочлен.
Можно просто сказать, что для оператора $\varphi$ его характеристический многочлен -- это характеристический многочлен его матрицы $A_\varphi$ и надо лишь показать, что он не зависит от базиса.
Это делается следующей проверкой
\[
\det(\lambda E - C A_\varphi C^{-1}) = \det(\lambda C C^{-1} - C A_\varphi C^{-1}) = \det(C(\lambda E - A_\varphi)C^{-1}) = \det(\lambda E - A_\varphi)
\]

\paragraph{Спектр}

Как и выше $\varphi\colon V\to V$ -- линейный оператор на векторном пространстве, тогда положим 
\[
\spec_F(\varphi) = \{\lambda\in F\mid \varphi - \lambda \Identity \text{ не обратим}\}
\]
И будем называть это множество спектром линейного оператора $\varphi$.%
\footnote{Заметим, что определение спектра дается без помощи матрицы линейного оператора.}
Пусть теперь $A_\varphi$ -- матрица линейного оператора в каком-нибудь базисе.
Оператор обратим тогда и только тогда, когда обратима его матрица (потому что все операции над операторами превращаются в операции над матрицами).
Потому условие $\varphi - \lambda \Identity$ не обратим превращается в условие $A_\varphi - \lambda E$ не обратима.
То есть спектр линейного оператора совпадает со спектром любой из его матриц в каком-нибудь базисе.

\paragraph{Минимальный многочлен}

Если у нас есть многочлен $f\in F[t]$ вида $f = a_0 + a_1 t + \ldots + a_nt^n$ и задан линейный оператор $\varphi \colon V\to V$, то можно определить оператор $f(\varphi)$ по правилу
\[
f(\varphi) = a_0 \Identity + a_1 \varphi + \ldots + a_n \varphi^n
\]
Здесь степень $\varphi^k$ -- это композиция оператора $\varphi$ с самим собой $k$ раз, а сумма и умножение на коэффициенты из поля берутся поточечно.%
\footnote{Смотри определения для линейных отображений в разделе~\ref{section::HomOperations}.}

Если в результате подстановки оператора в многочлен мы получили нулевой оператор (тот который на всех векторах действует нулем), то мы говорим, что многочлен зануляет $\varphi$ и пишем $f(\varphi) = 0$.
Если в каком-то базисе $e_1,\ldots,e_m\in V$ оператор $\varphi$ имеет матрицу $A\in \operatorname{M}_{m}(F)$, то $f(\varphi) = 0$ тогда и только тогда, когда $f(A) = 0$.
Действительно, при переходе к базису $f(\varphi)$ имеет матрицу $f(A)$, а нулевой оператор соответствует нулевой матрице.
Теперь мы можем определить минимальный многочлен оператора, как такой ненулевой многочлен $f_{\text{min}\,\varphi}\in F[t]$, что
\begin{enumerate}
\item $f_{\text{min}\,\varphi}(\varphi) = 0$.

\item $f_{\text{min}\,\varphi}$ имеет наименьшую степень среди всех ненулевых многочленов зануляющих $\varphi$.

\item Старший коэффициент $f_{\text{min}\,\varphi}$ равен единице.
\end{enumerate}
В силу того, что для многочлена занулять оператор это тоже самое, что занулять его матрицу, то минимальный многочлен для линейного оператора совпадает с минимальным многочленом для его матрицы в любом базисе.

\paragraph{Ранг}

Мы знаем, что линейный оператор -- это просто линейное отображение, но на одном пространстве.
Для любого линейного отображения мы видели, что ранг его матрицы не меняется при смене базиса (см.~утверждение~\ref{claim::HomClassification}).
В частности ранг матрицы линейного оператора не меняется при смене базиса.

\subsection{Обратимость оператора}

\begin{claim}
\label{claim::OperatorInvert}
Пусть $V$ -- векторное пространство над некоторым полем $F$ и $\varphi\colon V\to V$ -- некоторый линейный оператор.
Тогда следующие свойства эквивалентны:
\begin{enumerate}
\item $\ker \varphi = 0$.

\item $\Im \varphi = V$.

\item $\varphi$ обратим.

\item $\det \varphi \neq 0$.
\end{enumerate}
\end{claim}
\begin{proof}
Это утверждение является преформулировкой утверждения~\ref{claim::InvertibleDiscription} на языке оператора.
С другой стороны его можно получить из комбинации пунктов утверждения~\ref{claim::ImKer}.
\end{proof}

\subsection{Инвариантные подпространства}

Пусть $U\subseteq V$ -- подпространство в некотором векторном пространстве $V$ над полем $F$ и пусть $\varphi\colon V\to V$ -- некоторый линейный оператор.
Будем говорить, что векторное подпространство $U$ является инвариантным относительно $\varphi$ (или просто $\varphi$-инвариантным), если $\varphi(U)\subseteq U$.

\paragraph{Пример}

\begin{itemize}
\item Рассмотрим пример поворота трехмерного пространства вокруг некоторой оси, а именно, пусть $A\colon \mathbb R^3 \to \mathbb R^3$, $x\mapsto Ax$, где $A\in \Matrix{3}$ задана так
\[
A = 
\begin{pmatrix}
{1}&{0}&{0}\\
{0}&{\cos \alpha}&{-\sin \alpha}\\
{0}&{\sin \alpha}&{\cos \alpha}\\
\end{pmatrix}
\]
В данном случае мы поворачиваем вокруг оси $\langle e_1 \rangle$.
Заметим, что подпространство $\langle e_1\rangle$ является инвариантным, любой вектор из этого подпространства остается неподвижным.
Кроме того, подпространство $\langle e_2, e_3\rangle$ -- плоскость поворота, тоже является инвариантным относительно $\varphi$, любой вектор в ней поворачивается на угол $\alpha$.

\item Для любого оператора $\varphi \colon V\to V$ его ядро и образ являются инвариантными подпространствами.
\end{itemize}

\paragraph{Ограничение оператора} 

\begin{definition}
Пусть $V$ -- векторное пространство над полем $F$.
Если $\varphi \colon V\to V$ -- линейный оператор и $U\subseteq V$ -- некоторое инвариантное подпространство, то тогда можно определить оператор $\varphi|_U\colon U\to U$, действующий по правилу $u\mapsto \varphi(u)$.
Такой оператор называется ограничением $\varphi$ на $U$.
\end{definition}

\paragraph{Инвариантность в терминах матрицы}

Пусть $V = U \oplus W$ -- прямая сумма подпространств.
Выберем в $U$ базис $e=(e_1,\ldots,e_n)$, а в $W$ базис $f = (f_1,\ldots,f_m)$.
Тогда $e \cup f$ является базисом $V$.
Если $\varphi \colon V \to V$ -- некоторый линейный оператор, то его можно записать в этом базисе в следующем блочном виде
\[
\varphi (e, f) = (e, f)
\begin{pmatrix}
{A}&{B}\\
{C}&{D}
\end{pmatrix}
\]
Заметим, что при этом подпространство $U$ будет $\varphi$-инвариантным тогда и только тогда, когда $C = 0$.
Действительно, если $U$ инвариантно, то $\varphi(U)\subseteq U$.
С другой стороны $U = \langle e \rangle$.
То есть $U$ инвариантно тогда и только тогда, когда $\varphi(e) \subseteq U$.
С другой стороны, по определению матрицы оператора $\varphi (e) = eA + fC$.
Но $eA + fC$ лежит в $\langle e\rangle$ тогда и только тогда, когда $C = 0$.
В этом случае определен оператор $\varphi|_U$ и матрица $A$ будет матрицей этого оператора в базисе $e$.

Аналогично, подпространство $W$ инвариантно тогда и только тогда, когда $B = 0$.
Если же оба пространства инвариантны, то матрица $\varphi$ является блочно диагональной.
То есть отсюда мы видим геометрический смысл блочно верхнетреугольных и блочно диагональных матриц.
Блочно верхнетреугольная означает наличие инвариантных подпространств натянутых на первый кусок базисных векторов.
Блочно диагональный вид означает разложение пространства в прямую сумму инвариантных подпространств.
Подобное разбиение в прямую сумму инвариантных позволяет сводить задачу про один оператор к задачам про оператор на пространстве меньшего размера.
Это бывает полезно, если надо вести рассуждение индукцией по размерности подпространств.

\begin{claim}
\label{claim::KerImInvar}
Пусть $\varphi,\psi\colon V\to V$ два коммутирующих линейных оператора.
Тогда $\ker \varphi$ и $\Im\varphi$ являются $\psi$-инвариантными.
\end{claim}
\begin{proof}
Случай $\ker \varphi$.
Мы должны показать, что $\psi(\ker\varphi) \subseteq \ker \varphi$.
Возьмем произвольный вектор $v\in \ker\varphi$, нам надо показать, что $\psi(v) \in \ker \varphi$.
То есть мы должны показать, что $\varphi(\psi(v)) = 0$.
Но $\varphi \psi v = \psi \varphi v = \psi 0 = 0$.

Случай $\Im \varphi$.
Мы должны показать, что $\psi(\Im \varphi) \subseteq \Im \varphi$.
Возьмем произвольный вектор $v\in \Im\varphi$, нам надо показать, что $\psi(v) \in \Im\varphi$.
Но условие $v\in \Im \varphi$ означает, что $v = \varphi (u)$ для некоторого $u\in V$.
Но тогда $\psi v = \psi \varphi u = \varphi (\psi(u))$, что и требовалось.
\end{proof}


\subsection{Собственные векторы и значения}

\begin{definition}
Пусть $V$ -- некоторое векторное пространство над полем $F$ и $\varphi \colon V\to V$ -- линейный оператор.
Вектор $v\in V$ называется собственным для $\varphi$, если найдется такое $\lambda \in F$, что $\varphi v = \lambda v$.
\end{definition}

\paragraph{Замечания}

\begin{itemize}
\item Вектор $v\in V$ является собственным тогда и только тогда, когда $\langle v \rangle$ является $\varphi$-инвариантным подпространством.
Таким образом изучать собственные векторы -- это то же самое, что изучать не более чем одномерные инвариантные подпространства.

\item Вектор $0\in V$ всегда является собственным для любого линейного оператора.
\end{itemize}

\begin{definition}
Пусть $V$ -- некоторое векторное пространство над полем $F$ и $\varphi \colon V\to V$ -- линейный оператор.
Число $\lambda \in F$ называется собственным значением $\varphi$, если найдется ненулевой $v\in V$ такой, что $\varphi v = \lambda v$.
\end{definition}

\paragraph{Замечания}

\begin{itemize}
\item Важно отметить, что в определении требуется, чтобы $v\neq 0$.
Это связано с тем, что вектор $0\in V$ является собственным для любого $\lambda$, то есть всегда верно $\varphi 0 = \lambda 0$.
И если не потребовать этого условия, то любое число удовлетворяет этому определению и в нем теряется смысл.Будь те внимательны.

\item Популярная ошибка -- считать, что $0$ не может быть собственным значением.
На самом деле, число $0$ как может являться собственным значением, так и может не являться им.
А именно, число $0$ является собственным значением тогда и только тогда, когда $\ker \varphi \neq 0$.
Потому что собственные векторы для значения $0$ -- это векторы $v\in V$ такие, что $\varphi(v) = 0 v = 0$.
Потому наличие ненулевого такого вектора означает, наличие ненулевого вектора в ядре, а это равносильно неинъективности, а значит и необратимости оператора (в силу утверждения~\ref{claim::OperatorInvert}).
\end{itemize}

\paragraph{Собственные и корневые подпространства}

\begin{definition}
Пусть $V$ -- некоторое векторное пространство над полем $F$ и $\varphi \colon V\to V$ -- линейный оператор.
Для любого числа $\lambda \in F$ определим собственное подпространство
\[
V_\lambda = \{v\in V\mid \varphi v = \lambda v\}
\]
\end{definition}

Заметим, что такое подмножество обязательно является подпространством, например, потому что совпадает с $\ker (\varphi - \lambda \Identity)$.
Действительно, $\varphi v = \lambda v$ тогда и только тогда, когда $\varphi v - \lambda v = 0$.
Что равносильно тому, что $(\varphi - \lambda \Identity)v = 0$, что значит $v\in \ker(\varphi - \lambda \Identity)$.

\paragraph{Замечание}

Обратите внимание, что оператор $\varphi$ на собственном подпространстве $V_\lambda$ действует как скалярный оператор $\lambda \Identity$, то есть все умножает на $\lambda$ просто по определению $\varphi v = \lambda v$ для любого $v\in V_\lambda$.
Тут Капитан Очевидность передает привет.
Однако, не спешите, его помощник по имени Нетривиальное Следствие сейчас расскажет пару слов.

Давайте рассмотрим произвольный многочлен $f\in F[x]$, тогда определен оператор $f(\varphi)\colon V\to V$.
Так вот, оператор $f(\varphi)$ на собственном подпространстве $V_\lambda$ действует умножением на $f(\lambda)$, то есть $f(\varphi) v = f(\lambda) v$.
Это очень простое наблюдение жутко полезно.

\begin{claim}
\label{claim::EigenSpec}
Пусть $V$ -- некоторое векторное пространство над полем $F$ и $\varphi \colon V\to V$ -- линейный оператор.
Тогда следующие условия равносильны:
\begin{enumerate}
\item $V_\lambda \neq 0$.

\item $\lambda \in \spec_F\varphi$.

\item $\lambda$ -- корень $\chi_\varphi(t)$.

\item $\lambda$ -- корень минимального многочлена для $\varphi$.
\end{enumerate}
\end{claim}
\begin{proof}
Эквивалентность последних трех условий была доказана в утверждениях~\ref{claim::MinPoly} и~\ref{claim::CharSpec}.
Здесь эти условия приводятся, чтобы создать общую картину у читающего.
Давайте проверим эквивалентность первого условия с оставшимися.

(1)$\Rightarrow$ Пусть $V_\lambda \neq 0$, тогда $\varphi v = \lambda v$ для некоторого ненулевого вектора.
Значит $(\varphi - \lambda \Identity)v = 0$.
А значит оператор $\varphi - \lambda \Identity$ не обратим.

$\Rightarrow$(1) Пусть $\varphi -\lambda \Identity$ не обратим.
Тогда по одному из эквивалентных свойств обратимости оператора (утверждение~\ref{claim::OperatorInvert}), это означает, что $\varphi - \lambda \Identity$ имеет не нулевое ядро.
То есть есть ненулевой вектор $v\in V$ такой, что $(\varphi - \lambda \Identity)v = 0$.
А это и значит, что $\varphi v = \lambda v$ после раскрытия скобок и переноса второго слагаемого в право.
\end{proof}


\begin{definition}
Пусть $V$ -- векторное пространство над полем $F$, $\varphi\colon V\to V$ -- линейный оператор и $\lambda$ -- его собственное значение.
Тогда кратность $\lambda$ в характеристическом многочлене $\chi_\varphi$ называется кратностью собственного значения.
\end{definition}

Почему это определение имеет смысл, вы увидите чуть позже, когда мы будем говорить про диагонализацию (утверждение~\ref{claim::DiagCrit}).

\begin{claim}
Пусть $F$ -- алгебраически замкнутое поле, $V$ -- векторное пространство над полем $F$ и $\varphi\colon V\to V$ -- линейный оператор.
Тогда обязательно существует ненулевой собственный вектор $v\in V$ для некоторого $\lambda\in F$.
\end{claim}
\begin{proof}
Действительно, наличие такого вектора означает, что для некоторого $\lambda\in F$ пространство $V_\lambda$ не нулевое.
А это по предыдущему утверждению равносильно тому, что $\lambda$ -- корень характеристического многочлена для $\varphi$.
Так как этот многочлен не константный (его степень равна размерности пространства),%
\footnote{Мы скромно закроем глаза на случай $V = 0$, то есть пространство нульмерно.
В этом случае большой вопрос, что считать спектром.
Правильно полагать его пустым.
Верность утверждения тогда зависит от аккуратности формулировки.
Но не надо забивать себе этим голову, просто имейте в виду, что иногда этот случай нужен.}
а $F$ -- алгебраически замкнуто, то у нас обязательно существует корень $\lambda\in F$.
А значит $V_\lambda \neq 0$ (по утверждению~\ref{claim::EigenSpec}).
\end{proof}

\begin{definition}
Пусть $V$ -- некоторое векторное пространство над полем $F$ и $\varphi \colon V\to V$ -- линейный оператор.
Для любого числа $\lambda \in F$ определим корневое подпространство
\[
V^\lambda = \{v\in V\mid \exists n:\,(\varphi - \lambda\Identity)^n v = 0\}
\]
\end{definition}

\paragraph{Замечания}

\begin{itemize}
\item Заметим, что  $V^\lambda=\bigcup_{n\geqslant 0}\ker (\varphi - \lambda \Identity)^n$.
Каждое из ядер является подпространством.
Однако в общем случае объединение подпространств не является подпространством.
Но в данном случае $\ker(\varphi - \lambda \Identity)^k\subseteq \ker(\varphi-\lambda \Identity)^{k+1}$, то есть наши подпространства возрастают.
Я оставлю в качестве упражнения убедиться, что при таком условии объединение обязательно будет подпространством.
 
\item Кроме того, по определению $V_\lambda \subseteq V^\lambda$.
При этом равенства в этом включении может не быть.
Пусть 
\[
A = 
\begin{pmatrix}
{0}&{1}\\
{0}&{0}
\end{pmatrix}
\]
Тогда $A\colon F^2\to F^2$ -- линейный оператор.
При этом $A^2 = 0$.
То есть $f(x) = x^2$ -- зануляющий многочлен.
Заметим, что он обязательно минимальный.
А значит $\spec(A) = \{0\}$.
Тогда $V_0 = \ker A$ и оно порождено вектором $e_1$.
С другой стороны $F^2 = \ker A^2$, а потому $V^0 = F^2$.

\item Обратите внимание, что $V_\lambda \neq 0$ тогда и только тогда, когда $V^\lambda\neq 0$.
В одну сторону -- это следует из вложения $V_\lambda\subseteq V^\lambda$.
В другую сторону, если $v\in V^\lambda$ и $v\neq 0$, то найдем такое $k$, что $w =(\varphi - \lambda\Identity)^k v \neq 0 $, а $(\varphi - \lambda \Identity)w = (\varphi - \lambda\Identity)^{k+1}v =0$.
Тогда $w\in V_\lambda$ и не нулевой.

\item Подпространства $V_\lambda$ и $V^\lambda$ являются $\varphi$ инвариантными для любого $\lambda$.

\end{itemize}

\subsection{Лемма о стабилизации}

\begin{claim}
\label{claim::StabilityLemma}
Пусть $V$ -- векторное пространство над полем $F$ и $\varphi \colon V\to V$ -- линейный оператор.
Тогда
\begin{enumerate}
\item Найдется такое число $0\leqslant k\leqslant \dim_F V$, что
\[
0\subsetneq \ker \varphi \subsetneq \ker \varphi^2\subsetneq \ldots \subsetneq \ker\varphi^k = \ker \varphi^{k+1} = \ldots
\]

\item
Найдется такое число $0\leqslant k\leqslant \dim_F V$, что
\[
 V \supsetneq \Im \varphi \supsetneq \Im \varphi^2 \supsetneq \ldots \supsetneq \Im\varphi^k = \Im \varphi^{k+1} = \ldots
\]
\end{enumerate}
\end{claim}

Давайте поясним, что утверждается.
Мы говорим, что ядра оператора сначала строго растут, а начиная с какого-то момента обязательно становятся одинаковыми для всех последующих шагов.
Аналогичное происходит с образами, только они сначала строго уменьшаются, а потом становятся одинаковыми.
Стоит обратить внимание, что $k$ может быть равным $0$, это означает, что нет строгих включений и равенства начинаются с самого начала.

\begin{proof}
(1) Нам достаточно показать, что если в какой-то момент $\ker \varphi^m = \ker\varphi^{m+1}$, то $\ker \varphi^{m+1} = \ker\varphi^{m+2}$.
Включение $\ker \varphi^{m+1}\subseteq \ker \varphi^{m+2}$ понятно из определения (если что-то зануляется $\varphi^{m+1}$, то оно зануляется и большей степенью $\varphi^{m+2}$).
Надо показать обратное.
Пусть $v\in \ker \varphi^{m+2}$, тогда $\varphi^{m+2}v = 0$.
То есть $\varphi^{m+1}(\varphi v) = 0$.
Это значит $\varphi v \in \ker \varphi^{m+1} = \ker \varphi^m$.
Последнее означает, что $\varphi^m(\varphi v) = 0$, то есть $\varphi^{m+1} v = 0$.
Значит $v\in \ker \varphi^{m+1}$, что и требовалось.

Теперь надо понять, что $k$ не превосходит размерность $V$.
Но это следует из того факта, что в цепочке
\[
0\subsetneq \ker \varphi \subsetneq\ker\varphi^2\subsetneq \ldots
\]
размерность подпространств каждый шаг растет хотя бы на единицу.
Значит больше, чем $\dim_F V$ шагов у нас быть не может.

(2) Доказательство этого факта проходит аналогично.
Либо можно воспользоваться соотношением между размерностями ядра и образа (утверждение~\ref{claim::ImKer} пункт~(3)) и увидеть, что стабилизация у образов начинается на том же значении $k$, что и у ядер.
\end{proof}

\paragraph{Замечание}

В силу этого утверждения мы получаем, что $V^\lambda = \ker (\varphi - \lambda \Identity)^m$ для некоторого достаточно большого $m$.
Понятно, что на самом деле, достаточно взять $m = \dim_F V$.%
\footnote{Если уж бы до конца честным, то можно еще сильнее уменьшить $m$.
Тут достаточно взять кратность собственного значения в минимальном многочлене, я докажу это позже.}
