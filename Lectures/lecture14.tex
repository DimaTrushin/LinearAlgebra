\ProvidesFile{lecture14.tex}[Лекция 14]


\subsection{Подпространства в $F^n$}
\label{section::Subspaces}

Давайте посмотрим как можно задавать подпространства в $F^n$.
Существует два способа
\vspace{3pt}

\begin{tabular}{c|c}
{\bf Явный}&{\bf Неявный}\\
\hline
{Если $v_1,\ldots,v_k\in V$, тогда $U = \langle v_1,\ldots, v_k \rangle$}&{Если $A\in \operatorname{M}_{m\,n}(F)$, тогда $U = \{y\in F^n \mid Ay = 0\}$}\\
\end{tabular}
\vspace{3pt}

По-хорошему, хочется научиться пересчитывать векторное пространство заданное в одной из этих форм в другую.
Мы разберем пока только одну из этих задач.
А именно, пусть подпространство задано неявно в виде системы, то как найти его базис?

Если подпространство $U\subseteq F^n$ задано в виде $U = \{y\in F^n\mid Ay = 0\}$ для некоторой матрицы $A\in\operatorname{M}_{m\,n}(F)$, то любой базис пространства $U$ будем называть фундаментальной системой решений (ФСР).
Ниже мы разберем задачу построения какого-нибудь ФСР для однородной системы линейных уравнений.

\paragraph{Нахождение ФСР однородной СЛУ}

В начале мы приведем алгоритм находящий ФСР, а потом объясним почему он работает.

\paragraph{Дано}

Система однородных линейных уравнений $Ax = 0$, где $A\in \operatorname{M}_{m\,n}(F)$ и $x\in F^n$.

\paragraph{Задача}

Найти ФСР системы $Ax = 0$.

\paragraph{Алгоритм}

\begin{enumerate}
\item Привести матрицу $A$ элементарными преобразованиями строк к улучшенному ступенчатому виду.
Например
\[
A' = 
\begin{pmatrix}
{1}&{0}&{a_{31}}&{0}&{a_{51}}\\
{0}&{1}&{a_{32}}&{0}&{a_{52}}\\
{0}&{0}&{0}&{1}&{a_{53}}\\
\end{pmatrix}
\]

\item Пусть $k_1,\ldots,k_r$ -- позиции свободных переменных.
Если положить одну из этих переменных равной $1$, а все остальные нулями, то существует единственное решение, которое мы обозначим через $u_i$ (всего $r$ штук).
Например, для матрицы $A'$ выше свободные переменные имеют номера $3$ и $5$.
Тогда вектора (записанные в строку)
\[
u_1 = 
\begin{pmatrix}
{-a_{31}}&{-a_{32}}&{1}&{0}&{0}
\end{pmatrix},\,
u_2 = 
\begin{pmatrix}
{-a_{51}}&{-a_{52}}&{0}&{-a_{53}}&{1}
\end{pmatrix}
\]
являются ФСР.
\end{enumerate}

\begin{proof}
[Доказательство корректности алгоритма поиска ФСР]

Пусть в общем виде, ступенчатый вид матрицы $A$ выглядит так
\[
\begin{pmatrix}
{1}&{*}&{0}&{*}&{0}&{*}&{*}&{0}&{*}\\
{}&{}&{1}&{*}&{0}&{*}&{*}&{0}&{*}\\
{}&{}&{}&{}&{1}&{*}&{*}&{0}&{*}\\
{}&{}&{}&{}&{}&{}&{}&{1}&{*}\\
\end{pmatrix}
\]
Тогда построенные вектора имеют вид
\[
\begin{matrix}
{}&{}&{k_1}&{}&{k_2}&{}&{\ldots}&{\ldots}&{}&{k_r}\\
{u_1 }&{(*}&{1}&{0}&{0}&{0}&{0}&{0}&{0}&{0)}\\
{u_2}&{(*}&{0}&{*}&{1}&{0}&{0}&{0}&{0}&{0)}\\
{\vdots}&{(*}&{0}&{*}&{0}&{*}&{1}&{0}&{0}&{0)}\\
{\vdots}&{(*}&{0}&{*}&{0}&{*}&{0}&{1}&{0}&{0)}\\
{u_r}&{(*}&{0}&{*}&{0}&{*}&{0}&{0}&{*}&{1)}\\
\end{matrix}
\]
В начале проверим, что $u_i$ линейно независимы.
Действительно, тогда линейная комбинация $\alpha_1 u_1 +\ldots + \alpha_r u_r$ имеет вид
\[
\begin{pmatrix}
{*}&{\alpha_1}&{*}&{\alpha_2}&{*}&{\ldots}&{\ldots}&{*}&{\alpha_r}\\
\end{pmatrix}
\]
Если эта линейная комбинация равна нулю, то значит и все $\alpha_i$ равны нулю.

Теперь пусть $v$ -- произвольное решение системы $Ax = 0$.
Посмотрим на его координаты в свободных позициях
\[
\begin{pmatrix}
{*}&{v_1}&{*}&{v_2}&{*}&{\ldots}&{\ldots}&{*}&{v_r}\\
\end{pmatrix}
\]
Теперь рассмотрим вектор $w = v - v_1 u_1 - \ldots - v_r u_r$.
С одной стороны это решение системы $Ax = 0$.
С другой стороны у этого решения все свободные переменные равны нулю.
А значит автоматически и все главные переменные равны нулю, что означает, что $w  = 0$.
То есть $v = v_1 u_1 + \ldots + v_r u_r$, что и требовалось.
\end{proof}

\paragraph{Замечание}

\begin{itemize}
\item Обратите внимание, что ФСР -- это любой базис в пространстве $\{y\in F^n \mid Ay = 0\}$, а не только тот, который построен по алгоритму.

\item В алгоритме выше, мы могли бы вместо $1$ и $0$ расставить любой набор из $r$ линейно независимых векторов длины $r$ в позиции со свободными переменными.
Это тоже дало бы базис.
Однако, у построенного ФСР именно по алгоритму выше есть одно важное преимущество: в нем легко считать координаты.
Действительно, для любого вектора из пространства решений его свободные переменные -- это и есть координаты в построенном базисе.
\end{itemize}

\newpage
\section{Линейные отображения}

\subsection{Идея и определение}

Пусть $F$ -- некоторое поле и нам даны два векторных пространства $V$ и $U$.
Самый первый вопрос, который встает глядя на них: а как их сравнить?
Одно и то же ли это пространство?
Для того, чтобы отвечать на подобные вопросы, нам и нужны линейные отображения.
Давайте начнем с более простого вопроса: а как сравнивать множества?
Множества сравниваются с помощью отображений.
Всевозможные теоремы об эквивалентности разных подходов с помощью инъективных, сюръективных и биективных отображений в конце концов говорят, что два множества надо считать одинаковыми, если между ними есть биекция, то есть в них одинаковый запас элементов.
Но теперь векторное пространство -- это не просто множество, на нем еще есть и операции.
Если на биекцию смотреть, как на способ переименовать элементы, то теперь мы хотим, чтобы при этом переименовании одни операции превращались в другие, то есть мы хотим, чтобы наше отображение было согласовано неким способом с операциями.
Ну и как в случае множеств, мы не будем ограничивать себя только биекциями, ибо все остальные отображения тоже оказываются очень полезными для сравнения разных векторных пространств.

\begin{definition}
Пусть $V$ и $U$ -- векторные пространства над некоторым полем $F$.
Отображение $\varphi\colon V\to U$ называется линейным, если оно удовлетворяет двум свойствам:
\begin{enumerate}
\item $\varphi(v_1 + v_2) = \varphi(v_1) + \varphi(v_2)$ для всех $v_1,v_2\in V$.

\item $\varphi(\lambda v) = \lambda \varphi(v)$ для всех $\lambda \in F$ и всех $v\in V$.
\end{enumerate}
\end{definition}

\paragraph{Примеры}

\begin{enumerate}
\item $F^n \to F^m$ задано по правилу $x\mapsto Ax$, где $A\in \operatorname{M}_{m\,n}(F)$.

\item $V\to U$ задано по правилу $v\mapsto 0$ для любого $v\in V$.
Такое отображение называется нулевым и обозначается $0$.

\item $V\to V$ задано по правилу $v\mapsto v$.
Такое отображение называется тождественным и обозначается $\Identity$.

\item Пусть $D[0,1]$ -- множество дифференцируемых функций на отрезке $[0,1]$, $F[0,1]$ -- множество всех функций на отрезке $[0,1]$.
Тогда зададим линейное отображение $D[0,1]\to F[0,1]$ по правилу $f\mapsto f'$, то есть взятие производной является линейным отображением.

\item Пусть $L[0,1]$ -- множество функций $f$ на отрезке $[0,1]$ таких, что существует интеграл $\int\limits_0^1 |f(x)|\,dx$%
\footnote{Для тех кто беспокоится о том, какой же тут берется интеграл, сообщим по честному, что на самом деле тут у нас функции интегрируемые по Лебегу на отрезке $[0,1]$.}
и пусть $C[0,1]$ -- множество непрерывных функций на отрезке $[0,1]$.
Тогда рассмотрим следующее отображение $f\mapsto \int\limits_0^t f(x)\,dx$.

\item Пусть $V$ -- некоторое векторное пространство с базисом $e_1,\ldots, e_n$.
Тогда отображение $F^n \to V$ по правилу $x \mapsto (e_1,\ldots,e_n)x$ является линейным.
\end{enumerate}

\paragraph{Замечание}

Давайте посмотрим на примеры (1), (4) и (5).
Они говорят, что в терминах линейных отображений можно говорить о системах линейных уравнений (1), о дифференциальных уравнениях (4) или об интегральных уравнениях (5).
Пример (6) нам встречался, когда мы объясняли, что любое векторное пространство превращается в пространство столбцов, если в нем выбрать базис.


\begin{definition}
Пусть $V$ и $U$ -- векторные пространства над некоторым полем $F$, тогда отображение $\varphi\colon V\to U$ называется изоморфизмом, если 
\begin{enumerate}
\item $\varphi$ линейно.

\item $\varphi$ биективно.
\end{enumerate}
\end{definition}

Заметим, что если отображение $\varphi\colon V\to U$ изоморфзим, то существует обратное отображение (потому что $\varphi$ биекция) и обратное так же является линейным.%
\footnote{Предлагаю дотошным читателям проверить это заявление, а остальным пустить скупую слезу и, поверив мне на слово, двигаться дальше в мир неизведанного.}
Если между двумя векторными пространствами $V$ и $U$ существует изоморфизм, то говорят, что они изоморфны и пишут $V \simeq U$.

Множество всех линейных отображений из пространства $V$ в пространство $U$ обозначается через $\Hom_F(V,U)$.
Формально
\[
\Hom_F(V,U) = \{f\colon V\to U \mid f \text{ -- линейное}\}
\]
Множество линейных отображений из $V$ в поле $F$ называется двойственным пространством к $V$ и обозначается через $V^*$.
Формально
\[
V^* = \{\xi \colon V\to F\mid \xi \text{ -- линейное}\} = \Hom_F(V,F)
\]
Если нет путаницы, то индекс, обозначающий поле $F$ опускают и пишут просто $\Hom(V,U)$.%
\footnote{Для интересующихся таким странным обозначением поясню, $\Hom$ происходит от слова homomorphism -- гомоморфизм, которое является более общим определением и в случае векторных пространств дает в точности понятие линейного отображения.}

\subsection{Операции на линейных отображениях}
\label{section::HomOperations}

Давайте порадуем себя еще немного абстрактным формализмом и введем операции на линейных отображениях.
Причем мы постараемся так, что множество всех линейных отображений между двумя векторными пространствами $\Hom_F(V,U)$ внезапно превратится в векторное пространство.
А это не так уж и плохо, это значит, что постоянно оставаясь в рамках векторных пространств, мы сможем к новым конструкциям применять все те же методы, что мы изначально разрабатывали для произвольных векторных пространств.

\paragraph{Сумма}

Пусть $\varphi\colon V\to U$ и $\psi \colon V\to U$ два линейных отображения между векторными пространствами $V$ и $U$.
Тогда, чтобы определить отображение $(\varphi+\psi)\colon V\to U$, надо задать его действие на всех векторах из $V$ и проверить, что полученное правило линейно.
Зададим его так: для любого $v\in V$ положим $(\varphi + \psi)(v) = \varphi(v) + \psi(v)$.

\paragraph{Умножение на скаляр}

Пусть $\varphi\colon V\to U$ -- линейное отображение между векторными пространствами $V$ и $U$ и пусть $\lambda \in F$ -- произвольный элемент поля.
Тогда определим $(\lambda \varphi)\colon V\to U$ как отображение задаваемое правилом $(\lambda\varphi)(v) = \lambda \varphi(v)$.

\paragraph{Композиция}

Пусть $\varphi\colon V\to U$ и $\psi \colon U \to W$ -- два линейных отображения, а $V$, $U$ и $W$ -- векторные пространства.
Тогда теоретико множественная композиция отображений $\psi\circ \varphi\colon V\to W$ это отображение задаваемое по правилу $(\psi\circ \varphi)(v) = \psi(\varphi(v))$.
Методом пристального взгляда на определение композиции и линейного отображения проверяется, что композиция тоже является линейным отображением.
Теоретико множественная композиция тогда обозначается просто $\psi\varphi$ и называется композицией линейных отображений.
Здесь полезно иметь перед глазами картинку:
\[
\xymatrix{
  {V}\ar[r]^{\varphi}\ar@/_10pt/[rr]_{\psi\varphi}&{U}\ar[r]^{\psi}&{W}
}
\]

\paragraph{Замечание}

Можно проверить методом пристального взгляда, что множество $\Hom_F(V,U)$ с операциями сложения и умножения на скаляр превращается в векторное пространство над полем $F$, а значит и двойственное пространство $V^*$ тоже является векторным пространством над $F$.%
\footnote{Глупо было бы ожидать другого результата.}

\subsection{Построение линейных отображений}

Мы хорошенько поиграли в абстракции, а теперь внимание главный вопрос дня: а как задавать линейные отображения между двумя векторыми пространствами $V$ и $U$?
И как следствие другой вопрос: а вообще существуют хоть какие-нибудь линейные отображения между $V$ и $U$?
К счастью на второй вопрос мы уже знаем ответ с помощью примера (2) выше.
Нулевое отображение у нас есть всегда, но это совсем не интересно.
На самом деле вопрос задания объектов нового вида -- это один из самых важных вопросов.
Когда мы проходили перестановки, нам нужен был способ с ними работать, работая с действительными числами, мы даже не задумываемся над способом их задания, так как эти способы нам знакомы с детства, задавать функции многих из нас тоже научили в школе.
Теперь пришло время научиться работать с линейными отображениями.

\begin{claim}
\label{claim::LinMapExist}
Пусть $V$ и $U$ -- векторные пространства и пусть $e_1,\ldots,e_n$ -- базис пространства $V$.
Тогда для любого набора $u_1,\ldots,u_n\in U$%
\footnote{Векторам $u_i$ разрешено повторяться.}
существует единственное линейное отображение $\varphi\colon V\to U$ такое, что $\varphi(e_i) = u_i$.
\end{claim}
\begin{proof}

Давайте проверим единственность в начале.
Пусть $\varphi,\psi\colon V\to U$ -- два таких отображения, тогда для любого вектора $v\in V$ имеем $v = x_1 e_1 + \ldots + x_n e_n$.
А значит
\[
\varphi(v) = \varphi(x_1 e_1 + \ldots + x_n e_n) = x_1 \varphi(e_1) + \ldots + x_n \varphi(e_n) = x_1 u_1 + \ldots + x_n u_n
\]
Аналогично $\psi(v)$ равно тому же самому.
Потому $\varphi = \psi$.%
\footnote{Чтобы проверить, что два отображения равны, как раз и надо проверить, что они совпадают на любом элементе, а мы ровно это и сделали.}

Теперь займемся вопросом существования.
Для этого надо всего лишь проверить, что правило $\varphi(v) =  x_1 u_1 + \ldots + x_n u_n$ задает линейное отображение.
Для этого надо проверить, что $\varphi(v_1 + v_2) = \varphi(v_1) + \varphi(v_2)$ и $\varphi(\lambda v) = \lambda \varphi(v)$.
Я молча выпишу ниже две необходимые для проверки строчки и понадеюсь на вашу сознательность и сообразительность в завершении доказательства.
Пусть $e = (e_1,\ldots,e_n)$ и $u = (u_1,\ldots, u_n)$, тогда для $v\in V$ найдется единственный $x\in F^n$, что $v = ex$.
А значит выражение $ux\in U$ зависит только от $v$, а не от выбора координат.
Потому можно положить $\varphi(v) = ux$.
Теперь линейность следует из равенств
\begin{gather*}
\varphi(v_1 + v_2) = \varphi(ex_1 + ex_2) = \varphi(e(x_1 + x_2)) = u(x_1 + x_2) = u x_1 + ux_2 = \varphi(v_1) + \varphi(v_2)\\
\varphi(\lambda v) = \varphi(\lambda ex) = \varphi(e(\lambda x)) = u (\lambda x) = \lambda u x = \lambda \varphi(v)
\end{gather*}
\end{proof}

Таким образом, если у вас зафиксирован базис в пространстве, то чтобы построить линейное отображение достаточно отправить базисные векторы куда угодно и у вас автоматом будет только одно линейное отображение, действующее на базисе таким образом.

\paragraph{Классификация векторных пространств с точностью до изоморфизма}

Теперь мы можем дать критерий, когда два векторных пространства изоморфны.

\begin{claim}
\label{claim::VectorClassific}
Пусть $V$ и $U$ -- векторные пространства над полем $F$.
Они изоморфны тогда и только тогда, когда $\dim_F V = \dim_F U$.
\end{claim}
\begin{proof}
Если векторные пространства $V$  и $U$ изоморфны, это значит, что существует изоморфизм $\varphi\colon V\stackrel{\sim}{\longrightarrow} U$.
Остается заметить, что при изоморфизме линейно независимое и порождающее множество переходит в линейно независимое и порождающее.
А значит, базис переходит в базис.
Следовательно пространства имеют одинаковую размерность.

Теперь пусть у пространств одинаковая размерность.
Тогда пусть $e_1,\ldots,e_n$ -- базис $V$ и $f_1,\ldots,f_n$ -- базис $U$.
Обратим внимание, что их количество одинаковое, так как это и есть их размерности, которые совпадают между собой по условию.
Тогда по предыдущему утверждению существует единственное отображение $\varphi\colon V\to U$ такое, что $e_i\mapsto f_i$.
Аналогично существует обратное отображение $\psi\colon U\to V$ переводящее $f_i\mapsto e_i$.
А следовательно композиции $\psi\varphi$ и $\varphi\psi$ оставляют на месте базисы пространств $V$ и $U$ соответственно.
Последнее означает, что  $\psi\varphi = \Identity_V$ и $\varphi\psi = \Identity_U$.
То есть это обратимые отображения, то есть биекции.
\end{proof}

\subsection{Линейные отображения и базис}

\paragraph{Матрица линейного отображения}

Пусть в векторном пространстве $V$ зафиксирован базис $e_1,\ldots,e_n$, а в пространстве $U$ базис $f_1,\ldots, f_m$.
Тогда любое линейное отображение $\varphi\colon V\to U$ задается набором $\varphi(e_1,\ldots,e_n) = (\varphi(e_1),\ldots, \varphi(e_n))$.%
\footnote{Это равенство можно понимать, как произведение матрицы $1$ на $1$ с элементом из $\Hom_F(V,U)$, с вектором размера $1$ на $n$ с элементами из пространства $V$.}
Любой вектор $\varphi(e_i)$ можно разложить по базису $f_1,\ldots,f_m$, то есть есть равенство $\varphi(e_i) = (f_1,\ldots,f_m)A_i$, где $A_i\in F^m$ -- столбец коэффициентов образа $e_i$ по $f_1,\ldots,f_m$.
Тогда все эти равенства вместе можно записать так:
\[
\varphi(e_1,\ldots,e_n) = (f_1,\ldots,f_m)A, \text{ где } A_\varphi = (A_1|\ldots|A_n)\in \operatorname{M}_{m\,n}(F)
\]
Как только фиксированы базисы $e_i$ и $f_i$, матрица $A_\varphi$ определена однозначно.
Эта матрица и называется матрицей линейного отображения $\varphi$ в базисах $e_i$ и $f_i$.%
\footnote{Как обычно я злоупотребляю своей любовью к опусканию чрезмерных деталей в обозначениях.
Надо помнить, что вообще говоря матрица $A_\varphi$ не имеет значения, если не сказано в каких именно базисах она посчитана.
Потому хорошо бы еще писать, что это $A_{\varphi\,e_i\,f_i}$, но на практике не удобно использовать такое громоздкое обозначение, да и не особенно много проку от него.}
Еще раз повторим в кратких обозначениях, если $e=(e_1,\ldots,e_n)$ и $f = (f_1,\ldots,f_m)$ то, чтобы показать, что какая-то матрица $A$ является матрицей линейного отображения $\varphi\colon V\to U$ в базисах $e$ и $f$, то надо показать равенство $\varphi e = f A$.

\paragraph{Действие в координатах}

Пусть теперь $v\in V$.
Тогда вектор $v$ можно разложить по базису $e$ в виде $v = ex$, для некоторого $x\in F^n$.
Применим $\varphi$ к $v$, получим
\[
\varphi(v) = \varphi(ex) = (\varphi e)x = f A_\varphi x
\]
То есть $\varphi(v)$ имеет координаты $A_\varphi x$.
То есть, если отождествить пространство $V$ с $F^n$ посредством $v = ex \mapsto x$ и пространство $U$ с $F^m$ посредством $u = fy \mapsto y$, то наше линейное отображение $\varphi$ превращается в отображение $F^n \to F^m$ по правилу $x \mapsto A_\varphi x$.
Таким образом изучать линейное отображение между двумя конечномерными векторными пространствами -- это все равно, что изучать отображение между пространствами столбцов, заданное умножением слева на матрицу.

\paragraph{Смена базиса}

Пусть теперь в пространстве $V$ задано два базиса $e = (e_1,\ldots, e_n)$ и $e' = (e_1',\ldots,e_n')$.
Аналогично, в пространстве $U$ -- два базиса $f = (f_1,\ldots,f_m)$ и $f' = (f_1',\ldots,f_m')$.
Кроме того выполнены равенства $e' = eC$ для некоторой $C\in \operatorname{M}_n(F)$  и $f' = f D$ для некоторой $D\in \operatorname{M}_m(F)$, то есть $D$ и $C$ -- матрицы перехода от нештрихованных базисов к штрихованным.
Тогда в базисах $e$ и $f$ линейное отображение $\varphi\colon V\to U$ выглядит $\varphi e = f A_\varphi$, а в базисах $e'$ и $f'$ -- $\varphi e' = f' A_\varphi'$.
Давайте поймем какая связь между матрицами $A_\varphi$ и $A_\varphi'$ в терминах матриц перехода $C$ и $D$.
\[
\varphi e' = f' A_\varphi' \,\Rightarrow\, \varphi e C = f D A_\varphi' \,\Rightarrow \, \varphi e = f D A_\varphi' C^{-1}
\]
В силу единственности матрицы линейного отображения
\[
A_\varphi = D A_\varphi' C^{-1} \,\Rightarrow\, A_\varphi' = D^{-1} A_\varphi C
\]
Последнее равенство показывает как меняется матрица линейного отображения, когда мы меняем базисы пространств.
Запоминать можно так: когда мы хотим заменить старую матрицу $A_\varphi$ на новую $A_\varphi'$ надо справа домножить на матрицу перехода от $e$ к $e'$, а слева на обратную к матрице перехода от $f$ к $f'$.
Еще полезно перед глазами держать следующую картинку:
\[
\xymatrix{
  {\text{старые}}&{F^n}\ar[r]^{\varphi}&{F^m}\ar[d]&{x = Cx'}\ar@{|->}[r]&{y = A_\varphi Cx'}\ar@{|->}[d]\\
  {\text{новые}}&{F^n}\ar[u]\ar[r]^{\varphi}&{F^m}&{x'}\ar@{|->}[u]\ar@{|->}[r]&{y' = D^{-1}A_\varphi Cx'}
}
\]
