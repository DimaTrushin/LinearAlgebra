\ProvidesFile{lecture12.tex}[Лекция 12]


\subsection{Подпространства}

Пусть $V$ -- векторное пространство над $F$.
Тогда непустое подмножество $U\subseteq V$ называется подпространством, если на него можно ограничить операции $+$ и $\cdot$ и относительно них оно является векторным пространством.
Давайте определим подпространство формально.
\begin{definition}
Пусть $V$ -- векторное пространство над полем $F$.
Тогда подмножество $U\subseteq V$ называется подпространством, если
\begin{enumerate}
\item $U$ не пусто.%
\footnote{При наличии свойства~(3) это свойство эквивалентно тому, что нулевой вектор $V$ попадает в $U$.
Действительно, если он попадает, то $U$ не пусто.
Наоборот, если $u\in U$ -- какой-то вектор, то $0 = 0 u\in U$ по третьему свойству.}

\item Для любых векторов $u,u'\in U$ верно, что $u+u'\in U$.

\item Для любого скаляра $\alpha\in F$ и вектора $u\in U$ верно, что $\alpha u \in U$.
\end{enumerate}
\end{definition}
Если $U$ -- подпространство в $V$, то на $U$ можно корректно ограничить операции сложения и умножения на скаляр из исходного пространства $V$.
Таким образом у нас получается набор данных $(U, +, \cdot)$ и теперь надо, чтобы выполнялись все аксиомы векторного пространства для них.
Оказывается, что все аксиомы будут выполняться автоматически!
Например, почему у нас будет $0\in U$.
Потому что если мы возьмем любой вектор $u\in U$, то $0 = 0 u \in U$.
Остальное я оставлю в качестве упражнения.

\paragraph{Примеры}

\begin{enumerate}
\item Для любого векторного пространства $V$ подмножества $0$ и $V$ всегда являются подпространствами.

\item Множество $\{y\in F^n \mid Ay = 0\}\subseteq F^n$, где $A\in\operatorname{M}_{m\,n}(F)$, является векторным подпространством в $F^n$.

\item Множество многочленов $\mathbb R[x]$ является подпространством в пространстве $F(\mathbb R, \mathbb R)$ -- всех функций на прямой.
\end{enumerate}

Обратите внимание, что подпространство из второго примера кажется устроено сложнее, чем векторное пространство $F^n$, в котором оно лежит.
Однако, окажется, что взаимодействие с ним через абстрактный интерфейс векторного пространства происходит точно так же.
То есть на самом деле подпространство устроено не сложнее, чем исходное пространство.
Об этом речь пойдет после того, как мы узнаем, что такое базисы и что значит, что какие-то векторные пространства одинаковые.

\subsection{Линейные комбинации}

\paragraph{Мотивация}

Пусть у нас есть векторное пространство $V$ над полем $F$.
Давайте поймем, а что вообще с ним можно делать?
Во-первых, $V$ -- это множество.
Значит из него можно брать элементы.
Во-вторых, там есть операция умножения на числа, то есть любой вектор можно умножить на какое-то число.
В-третьих, вектора можно складывать.
Все это означает, что все что можно делать с векторным пространством, это набрать каких-то векторов из него $v_1,\ldots,v_n$ и написать выражение вида $\alpha_1 v_1 + \ldots + \alpha_n v_n$, для произвольных $\alpha_i\in F$.
Это выражение будет задавать нам какой-то вектор из $V$.
Как мы видим, особенно не разбежишься с разнообразием действий.
Однако, важно, что с помощью подобных выражений можно вытащить абсолютно всю информацию из векторных пространств, которую только возможно.
Именно поэтому все наше внимание будет посвящено выражениям такого вида, так как из них получится узнать все, что только можно про векторные пространства.

\paragraph{Линейные комбинации}

\begin{definition}
Пусть $V$ -- некоторое векторное пространство над полем $F$ и пусть $v_1,\ldots,v_n\in V$ -- некоторый набор векторов.
Тогда выражение вида $\alpha_1 v_1 +\ldots + \alpha_n v_n$, где $\alpha_i\in F$, называется линейной комбинацией $v_1,\ldots,v_n$.
Линейная комбинация называется тривиальной, если все $\alpha_i = 0$.
В противном случае она называется нетривиальной.
\end{definition}

\begin{definition}
Вектора $v_1,\ldots,v_n\in V$ называются линейно зависимыми, если существует их нетривиальная линейная комбинация равная нулю, то есть для каких-то $\alpha_i\in F$ (так что хотя бы один не равен нулю) выражение $\alpha_1 v_1+\ldots + \alpha_n v_n = 0$.
Подчеркнем, что вектора линейно независимы, если из равенства $\alpha_1 v_1 + \ldots + \alpha_n v_n = 0$ следует, что все $\alpha_i = 0$.
\end{definition}

\paragraph{Примеры}

\begin{enumerate}
\item Вектор $0$ всегда линейно зависим.

\item Вектор $v\in V$ линейно зависим тогда и только тогда, когда он равен нулю.

\item Вектора $v_1, v_2 \in V$ линейно зависимы тогда и только тогда, когда они пропорциональны (то есть один из них равен другому умноженному на элемент поля).
\end{enumerate}

Заметим, что если множество векторов $v_1,\ldots, v_k$ линейно независимо, то и любое его подмножество тоже линейно независимо.
Потому интересно не уменьшать, а увеличивать линейно независимые подмножества векторов.
Линейно независимое множество векторов $v_1,\ldots, v_k$ называется максимальным, если при добавлении к нему любого вектора оно становится линейно зависимым.

\paragraph{Линейная оболочка}

\begin{definition}
Пусть $E\subseteq V$ -- некоторое подмножество в векторном пространстве $V$ над полем $F$.
Тогда обозначим через $\langle E \rangle$ множество всех линейных комбинаций векторов из $E$, то есть
\[
\langle E \rangle = \{\alpha_1 v_1 + \ldots + \alpha_n v_n \mid \alpha_i\in F,\, v_i \in E,\, n\in\mathbb N\}
\]
\end{definition}

Сделаем важное замечание, если $E = \varnothing$ (пусто), то $\langle \varnothing \rangle$ полагаем равным нулевому подпространству (подпространству состоящему только из нуля).
Это полезное и удобное соглашение можно понимать так: если берется линейная комбинация с нулевым числом слагаемых, то она равна нулю.

Заметим, что $\langle E \rangle$ является наименьшим векторным подпространством содержащим $E$.
Потому, для любого подпространства $U\subseteq V$ верно $\langle U \rangle  = U$.

\paragraph{Пример}

Полезно держать перед глазами следующий пример.
Пусть $V = \mathbb R^3$ -- пространство, $e_1 =(1,0,0)$, $e_2 = (0,1,0)$, $e_3 = (0,0,1)$ -- три вектора вдоль координатных осей.
Тогда $\langle e_1\rangle$, $\langle e_2\rangle$ и $\langle e_3\rangle$ -- это в точности координатные оси.
Подпространства $\langle e_1, e_2\rangle$, $\langle e_1, e_3\rangle$ и $\langle e_2, e_3\rangle$ -- это плоскости содержащие пары координатных осей, $\langle e_1, e_2, e_3\rangle$ будет совпадать со всем пространством $\mathbb R^3$.

\paragraph{Порождающее подмножество}

\begin{definition}
Пусть $V$ -- векторное пространство над полем $K$, тогда подмножество $E\subseteq V$ называется порождающим, если $\langle E \rangle  = V$.
\end{definition}

Другими словами, $E$ является порождающим если любой вектор из $V$ является линейной комбинацией векторов из $E$.
Отметим, что $V$ целиком всегда является порождающим.
Если $E\subseteq E'\subseteq V$ и подмножество $E$ является порождающим, то и $E'$ тоже порождающее.
Потому порождающее семейство всегда можно увеличить и это не интересно, интереснее попытаться его уменьшить и сделать более экономным.
Порождающее множество $E$ называется минимальным, если любое строго меньшее подмножество $E$ уже не порождающее.
Для этого достаточно проверить, что для любого $v\in E$ множество $E\setminus \{v\}$ уже не порождающее.


\subsection{Базис}

Подмножество $E\subseteq V$ называется линейно независимым, если любое конечное подмножество векторов $E$ линейно независимо.
Если $E'\subseteq E\subseteq V$ и $E$ является линейно независимым, то $E'$ тоже будет линейно независимым.
Потому линейно независимое подмножество можно всегда уменьшать%
\footnote{Вопрос линейной независимости пустого множества оставим на совести строгой аксиоматической теории множеств и не будем его касаться, чтобы не обжечься о всякий формальный геморрой.}
и это не интересно, интереснее попытаться его увеличить и сделать наиболее большим.
Линейно независимое подмножество $E$ называется максимальным, если любое строго содержащее его подмножество является линейно зависимым.
Для этого достаточно проверить, что для любого $v\in V\setminus E$ множество $E\cup \{v\}$ является линейно зависимым.

\begin{claim}
\label{claim::Basis}
Пусть $V$ -- некоторое векторное пространство над некоторым полем $F$.
Тогда следующие условия на подмножество $E\subseteq V$ эквивалентны:
\begin{enumerate}
\item $E$ -- минимальное порождающее подмножество.

\item $E$ -- максимальное линейно независимое подмножество.

\item $E$ -- одновременно порождающее и линейно независимое подмножество.
\end{enumerate}
\end{claim}
\begin{proof}
Будем доказывать по схеме (1)$\Leftrightarrow$(3)$\Leftrightarrow$(2).

(1)$\Rightarrow$(3).
Пусть $E$ -- минимальное порождающее, нам надо показать, что оно будет линейно независимым.
Предположим противное, пусть найдется вектора $v_1,\ldots,v_n\in E$ и числа $\alpha_1,\ldots,\alpha_n\in F$, так что не все из них равны нулю, что выполнено $\alpha_1 v_1 +\ldots + \alpha_n v_n = 0$.
Мы можем предположить, что $\alpha_1 \neq 0$.
Тогда $v_1 = \beta_2 v_2 +\ldots + \beta_n v_n$ для некоторых $\beta_i\in F$.
Давайте покажем, что тогда $E\setminus\{v_1\}$ тоже является порождающим, что будет противоречить минимальности $E$.
Действительно, пусть $v\in V$ -- произвольный вектор.
Так как $E$ -- порождающее, то $v$ выражается через вектора из $E$, $v = \sum_i \alpha_i' v_i'$.
Если среди $v_i'$ нет вектора $v_1$ то мы выразили $v$ через $E\setminus\{v_1\}$, если есть то подставим вместо него выражение $\beta_2 v_2 + \ldots + \beta_n v_n$ и получим выражение $v$ только через вектора из $E\setminus \{v_1\}$, что и требовалось.

(3)$\Rightarrow$(1).
Пусть $E$ одновременно порождающее и линейно независимое, нам надо показать, что оно минимальное порождающее.
Достаточно проверить, что для любого $v\in E$ вектор $v$ не лежит в $\langle E\setminus\{v\}\rangle$.
Действительно, пусть лежит, тогда найдутся вектора $v_1,\ldots, v_n\in E\setminus\{v\}$ и числа $\alpha_i\in F$ такие, что $v = \alpha_1 v_1 + \ldots + \alpha_n v_n$, то тогда $(-1)v + \alpha_1 v_1 + \ldots + \alpha_n v_n = 0$ -- нетривиальная линейная комбинация разных элементов $E$, что противоречит линейной независимости $E$.

(2)$\Rightarrow$(3).
Пусть $E$ -- максимальное линейно независимое, нам надо показать, что оно будет порождающим.
Нам надо показать, что $\langle E \rangle  = V$.
Пусть это не так, возьмем $v\in V\setminus \langle E \rangle$, тогда множество $E\cup \{v\}$ строго больше, а значит линейно зависимо.
То есть для каких-то $v_1,\ldots,v_n \in E\cup \{v\}$ и чисел $\alpha_i \in F$ (так что не все из них нули) выполнено $\alpha_1 v_1 + \ldots + \alpha_n v_n = 0$.
Выкинув все нулевые слагаемые, можем считать, что на самом деле все $\alpha_i$ не равны нулю.
Если среди $v_i$ нет $v$, то значит все они из $E$.
Тогда это означает, что $E$ линейно зависимо, что неправда.
Значит один из $v_i$ -- это $v$.
Будем считать, что $v_1 = v$.
Так как по нашему предположению все коэффициенты не нулевые, то $v = v_1$ выражается через остальные $v_2,\ldots,v_n$.
Но это означает, что $v\in \langle E\rangle$, а это противоречит с выбором $v$.

(3)$\Rightarrow$(2).
Пусть $E$ одновременно порождающее и линейно независимое, нам надо показать, что оно максимальное линейно независимое.
Для этого возьмем любой вектор $v\in V\setminus E$ и покажем, что $E \cup \{v\}$ линейно зависимо.
Действительно, мы знаем, что $E$ порождающее, значит $v$ представляется в виде линейной комбинации векторов из $E$, то есть $v = \alpha_1 v_1 + \ldots + \alpha_n v_n$ для некоторых $v_i\in E$ и $\alpha_i\in F$.
Но тогда $(-1)v + \alpha_1 v_1 + \ldots + \alpha_n v_n  0$ -- нетривиальная линейная комбинация векторов из $E\cup \{v\}$, то есть последнее множество линейно зависимо, что и требовалось.
\end{proof}

Пусть $V$ -- векторное пространство над некоторым полем $F$, тогда подмножество $E\subseteq V$ удовлетворяющее одному из трех эквивалентных условий предыдущего утверждения называется базисом $V$.

\paragraph{Примеры}

\begin{enumerate}
\item Пусть $V = F^n$, тогда вектора
\[
v_1 = 
\begin{pmatrix}
{1}\\{0}\\{\vdots}\\{0}\\
\end{pmatrix},
v_2 = 
\begin{pmatrix}
{0}\\{1}\\{\vdots}\\{0}\\
\end{pmatrix},
\ldots,
v_n = 
\begin{pmatrix}
{0}\\{0}\\{\vdots}\\{1}
\end{pmatrix}
\]
являются базисом.
Очевидно, что эти вектора линейно независимы и любой вектор через них выражается.

\item Пусть $V = F[x]$ -- множество многочленов, тогда в качестве базиса можно взять $E = \{1,x,x^2, \ldots , x^n,\ldots\}$ -- множество всех степеней $x$.
Заметим, что в данном случае базис получается бесконечным.

\item Пусть $X$ -- произвольное множество и $V = \{f\colon X\to F\}$ -- множество всех функций на $X$ со значениями в $F$.
Тогда это векторное пространство над $F$ с очень любопытным свойством.

Ситуация с базисами тут устроена так.
Чтобы работать с бесконечными множествами нам нужно использовать аккуратно определенную теорию множеств.
Я не буду вдаваться в подробности, что там да как строится, но важно понимать, что в теории множеств вообще говоря не всякое утверждение является доказуемым или опровергаемым.
Подобные утверждения можно включить в качестве дополнительных аксиом, а можно их отрицания использовать в качестве таких же законных аксиом и будут получаться совершенно разные теории множеств.
Есть такая популярная аксиома <<аксиома выбора>>, которую очень любят включать в список стандартных.

Если вы используете аксиому выбора, то можно доказать, что всякое векторное пространство имеет базис.
Если же вы не используете аксиому выбора, то нельзя ни доказать, ни опровергнуть существования базиса уже в пространстве $V$ из этого примера.
Оказывается, что факт существования базиса является более слабым утверждением, чем аксиома выбора.
Кроме того, если базис существует по аксиоме выбора, то это значит, что не существует никакой процедуры, которая бы помогла вам описать этот базис, потому что существование подобной процедуры дало бы вам доказательство существования базиса без аксиомы выбора.
\end{enumerate}

\paragraph{Замечания}

\begin{itemize}
\item Пусть $V$ -- некоторое векторное пространство и $E'\subseteq V$ -- произвольное линейно независимое подмножество.
Тогда его всегда можно дополнить до базиса $E \supseteq E'$, потому что базис -- это максимальное линейно независимое подмножество.
В случае, если существует конечный базис, это просто.
А если конечного не существует, то тут придется обращаться к аккуратной формулировке аксиоматики теории множеств.

\item Пусть $V$ -- некоторое векторное пространство и $E''\subseteq V$ -- произвольное порождающее множество, тогда из него всегда можно выбрать базис $E\subseteq E''$.
Как и в предыдущем случае, если существует конечный базис, то это просто.
А если нет конечного базиса, то это требует аккуратной аксиоматики теории множеств.
\end{itemize}



\subsection{Удобный формализм}

Пусть $V$ -- некоторое векторное пространство над некоторым полем $F$.
Возьмем некоторые вектора $v_1,\ldots, v_n\in V$ и набор чисел $x_1,\ldots,x_n\in F$.
Тогда можно составить строку из векторов $v_i$ и столбец из чисел $x_i$ и перемножить в следующем порядке
\[
\begin{pmatrix}
{v_1}&{v_2}&{\ldots}&{v_n}
\end{pmatrix}
\begin{pmatrix}
{x_1}\\{x_2}\\{\vdots}\\{x_n}
\end{pmatrix}
=
x_1 v_1 + \ldots + x_n v_n
\]
Таким образом мы можем записывать линейные комбинации с помощью матричных объектов, когда матрицы состоят не только из чисел, но и из векторов.
Если при этом ввести обозначения
\[
v = 
\begin{pmatrix}
{v_1}&{v_2}&{\ldots}&{v_n}
\end{pmatrix}
\quad\text{и}\quad
x = 
\begin{pmatrix}
{x_1}\\{x_2}\\{\vdots}\\{x_n}
\end{pmatrix}
\]
то линейную комбинацию можно записать как $v x$.
Если $w\in V$ -- некоторый вектор, то тот факт, что он линейно выражается через $v_i$ тогда записывается так $w = vx$ для некоторого $x\in F^n$.
Пусть теперь у нас есть несколько векторов $w_1,\ldots, w_m\in V$ и каждый из них выражается через вектора $v_1,\ldots, v_n$, тогда
\[
w_1 = 
\begin{pmatrix}
{v_1}&{v_2}&{\ldots}&{v_n}
\end{pmatrix}
A_1,
\ldots,
w_m = 
\begin{pmatrix}
{v_1}&{v_2}&{\ldots}&{v_n}
\end{pmatrix}
A_m
\]
где $A_i\in F^n$.
Тогда составим из $A_i$ матрицу $A\in \operatorname{M}_{n\,m}(F)$ и получим запись
\[
\begin{pmatrix}
{w_1}&{w_2}&{\ldots}&{w_m}
\end{pmatrix}
=
\begin{pmatrix}
{v_1}&{v_2}&{\ldots}&{v_n}
\end{pmatrix}
A
\]

\subsection{Размерность}

Наша задача сейчас показать, что в векторном пространстве любые два базиса имеют одинаковое количество элементов.
Однако, обсуждать как сравнивать бесконечные множества между собой я не очень хочу, потому мы с этого момента ограничимся случаями конечных базисов.
Для начала нам надо показать, что если векторное пространство имеет хотя бы один конечный базис, то все его базисы конечны и имеют одинаковое количество элементов.

\begin{claim}
Пусть $V$ -- некоторое векторное пространство над полем $F$ и пусть $\{e_1,\ldots,e_n\}\subseteq V$ -- базис $V$.
Тогда если $E\subseteq V$ -- некоторый базис $V$, то $|E| = n$.
\end{claim}
\begin{proof}
Нам достаточно показать, что $|E|\leqslant n$.
Тогда базис $E$ становится конечным и мы можем поменять местами два базиса и применить это же утверждение для доказательства обратного неравенства.

Предположим, что это не верно, тогда в $E$ есть хотя бы $n+1$ элемент $v_1,\ldots, v_{n+1}$.
Так как $e_1,\ldots, e_n$ -- базис, то каждый $v_i$ линейно выражается через этот базис.
Значит можно найти матрицу $A\in \operatorname{M}_{n\,n+1}(F)$ такую, что
\[
\begin{pmatrix}
{v_1}&{v_2}&{\ldots}&{v_{n+1}}
\end{pmatrix}
=
\begin{pmatrix}
{e_1}&{e_2}&{\ldots}&{e_n}
\end{pmatrix}
A
\]
Рассмотрим систему $Ax = 0$, где $x\in F^{n+1}$.
В этой системе количество столбцов больше, чем количество строк.
Значит обязательно существует ненулевое решение $x\in F^{n+1}$.
Тогда умножим на него предыдущее равенство справа, получим
\[
\begin{pmatrix}
{v_1}&{v_2}&{\ldots}&{v_{n+1}}
\end{pmatrix}x
=
\begin{pmatrix}
{e_1}&{e_2}&{\ldots}&{e_n}
\end{pmatrix}
Ax=
0
\]
То есть мы нашли нетривиальную линейную комбинацию векторов $v_1,\ldots, v_{n+1}$.
Но по определению $E$ в нем не должно быть линейно зависимых векторов, противоречие.
\end{proof}

Пусть $V$ -- векторное пространство над полем $F$, тогда размерностью $V$ называется число элементов в любом из его базисов.%
\footnote{Корректность этого определения следует из предыдущей леммы в случае существования хотя бы одного конечного базиса.
Однако, если бы мы были знакомы с теорией мощности для произвольных множеств, то мы бы показали, что количество элементов в базисе не зависит от базиса всегда.
Потому можно говорить о размерности даже для бесконечно мерных пространств.
Например, размерность многочленов $F[x]$ счетная, а для бесконечного множества $X$ размерность пространства $F^X$ совпадает с $|F^X|$, то есть она зависит от мощности поля.}
Размерность $V$ будем обозначать через $\dim V$ или $\dim_F V$, если надо подчеркнуть, какое поле $F$ имеется в виду.

\begin{claim}
Пусть $U\subseteq V$ -- подпространство в векторном пространстве над полем $F$.
Тогда
\begin{enumerate}
\item $\dim U \leqslant \dim V$.

\item $\dim U = \dim V$ тогда и только тогда, когда $U = V$.
\end{enumerate}
\end{claim}
\begin{proof}
(1) В начале сделаем замечание.
Пусть $E\subseteq V$ -- какое-то линейно независимое подмножество $V$.
Так как базис -- это максимальное линейно независимое подмножество, то $|E|\leqslant \dim V$.

Пусть $E\subseteq U$ -- базис $U$.
Тогда $E$ -- линейно независимое подмножество $U$, а значит и $V$.
Но тогда из замечания выше $|E|\leqslant \dim V$.
А по определению $\dim U = |E|$.

 (2) Теперь сделаем еще одно замечание.
 Пусть $E\subseteq V$ -- некоторое линейно независимое подмножество $V$.
 Как понять, что оно максимальное?
Достаточно, проверить, что в нем $\dim V$ элементов.
Действительно, если бы при этом оно было не максимальным, то в максимальном было бы больше $\dim V$ элементов, что противоречит определению размерности.
 
Теперь пусть $E\subseteq U$ -- базис $U$ и пусть $\dim U = \dim V$.
Мы хотим показать, что $ U = V$.
Тогда $E$ -- это линейно независимое подмножество в $V$ и в нем $|E| = \dim U = \dim V$ элементов.
Но тогда по замечанию выше оно является базисом в $V$.
Так как $E$ -- базис в $U$, то $U = \langle E \rangle$, а так как $E$ -- базис в $V$, то $V = \langle E \rangle$, то есть $U= V$.
Утверждение в обратную сторону очевидно.
\end{proof}

\subsection{Конкретные векторные пространства}
\label{subsection::FnSpace}

Пусть $V$ -- векторное пространство над некоторым полем $F$.
Вообще говоря, в этом случае элементы $V$ могут быть чем угодно (функции, вектор-столбцы, матрицы, отображения и т.д.), но если в нем можно выбрать конечный базис, то оно автоматически превратится в пространство $F^n$.
Сейчас я хочу обсудить все этапы этого магического превращения.

Пусть $e_1,\ldots,e_n\in V$ -- некоторый базис пространства $V$.
Тогда можно рассмотреть отображение 
\begin{align*}
F^n &\to V\\
x &\mapsto ex
\end{align*}
где $e=(e_1,\ldots,e_n)$, $x\in F^n$.
Так как $e$ порождает $V$, то это отображение сюръективно.
С другой стороны из линейной независимости следует инъективность: если $ex = ey$ для $x,y\in F^n$, то $e (x - y) = 0$, а значит $ x - y = 0$.
Таким образом мы получаем, что каждый вектор-столбец длины $n$ однозначно соответствует некоторому вектору из $V$.
Кроме того, если присмотреться внимательно, то мы увидим, что сложение столбцов соответствует сложению векторов и то же самое верно для умножения на скаляр.
Таким образом, мы видим, что между этими пространствами нет никакой разницы.
Изучать одно из них -- это все равно, что изучать другое.
По-другому, можно думать еще так: если вам дали произвольное конечномерное пространство, то всегда можно считать, что это $F^n$ (для этого нужно всего лишь выбрать базис).

\paragraph{Координаты}

Если вектор $v\in V$ разложен по некоторому базису $e = (e_1,\ldots,e_n)$ пространства $V$, то есть представлен в виде $v = ex$, где $x\in F^n$, то столбец $x = (x_1,\ldots,x_n)$ называют координатами вектора $v$ в базисе $e_1,\ldots, e_n$.
