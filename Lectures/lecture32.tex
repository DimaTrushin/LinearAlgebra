\ProvidesFile{lecture32.tex}[Лекция 32]


\subsection{Сопряженное линейное отображение}

В случае произвольного линейного оператора $\phi$ на пространстве $V$, его сопряженный или двойственный $\phi^*$ живет на $V^*$ и это неудобно.
В случае евклидова или эрмитова пространства можно определить сопряженный оператор уже на самом пространстве $V$.
Я в начале расскажу, как строится подобный оператор на пространстве $V$ с помощью скалярного произведения, а потом уже поясню, как он связан с нашим старым знакомым на двойственном пространстве.

\begin{claim}
Пусть $V$ и $U$ -- евклидовы или эрмитовы пространства и $\phi\colon V\to U$ линейное отображение.
Тогда существует единственное линейное отображение $\phi^*\colon U\to V$ такое, что $(\phi v, u) = (v, \phi^* u)$ для любых $v\in V$ и $u\in U$.
\end{claim}
\begin{proof}
Для доказательства перейдем в базисы и сделаем все на матричном языке.
Давайте я для определенности рассмотрю эрмитов случай.
Пусть $e_1,\ldots,e_n$ ортонормированный базис пространства $V$, а $f_1,\ldots,f_m$ -- ортонормированный базис пространства $U$.
Тогда $V$ превращается в $\mathbb C^n$, $U$ превращается в $\mathbb C^m$, скалярное произведение в обоих пространствах становится стандартным, а отображение $\phi$ превращается в отображение $\mathbb C^n \to \mathbb C^m$ по правилу $x \mapsto Ax$ для некоторой $A\in\operatorname{M}_{m\,n}(\mathbb C)$.
Давайте будем искать наше отображение $\phi^*$  в виде $\phi^*(x) = Bx$, где $B\in \operatorname{M}_{n\,m}(\mathbb C)$ и покажем, что существует ровно одно линейное отображение удовлетворяющее нужным условиям.

Действительно, условие $(\phi v, u) = (v, \phi^* u)$ переписывается так
\[
(\overline{Ax})^t y = \bar x^t By\iff \bar x^t \bar A^t y = \bar x^t B y
\]
И это должно выполняться для любых $x\in \mathbb C^n$ и $y\in \mathbb C^m$.
А это возможно тогда и только тогда, когда $B = \bar A^t$.
Что и означает, что нужное отображение найдется и единственное.
\end{proof}

\begin{definition}
Если $V$ и $U$ -- евклидовы или эрмитовы пространства и $\phi\colon V\to U$ -- линейное отображение.
Тогда отображение $\phi^*\colon U\to V$ называется сопряженным к $\phi$.
В частности, если $\varphi\colon V\to V$ является линейным оператором, то $\varphi^*$ называется сопряженным оператором.%
\footnote{Иногда говорят евклидово сопряженный или эрмитово сопряженный, чтобы подчеркнуть, что мы в евклидовом или эрмитовом случае.}
\end{definition}

\paragraph{Замечания}

\begin{itemize}
\item В силу симметричности скалярного произведения определение сопряженного отображения можно дать в виде $(\phi^* u, v) = (u, \phi v)$.
Таким образом по простому, сопряженное линейное отображение  -- это такое линейное отображение, которое получается при перекидывании внутри скалярного произведения.
Это означает, что если вы хотите что-то доказать для сопряженного линейного отображения, то надо желаемый факт выразить в терминах скалярного произведения, а после этого перекинуть $\phi^*$ и превратить его в $\phi$ и воспользоваться свойствами $\phi$ или наоборот.

\item Обратите внимание, что если у нас оператор $\phi\colon V\to V$ в ортонормированном базисе задан матрицей $A$, то оператор $\phi^*$ в этом же базисе будет задан матрицей $A^t$ в вещественном случае и $A^* = \bar A^t$ в комплексном случае.

\item Пусть теперь $\phi\colon V\to U$ -- произвольное линейное отображение.
Пусть $e_1,\ldots,e_n$ -- некоторый базис в $V$ и $f_1,\ldots, f_m$ -- некоторый базис в $U$.
Пусть скалярное произведение $V$ в указанном базисе задано матрицей $B$, а в пространстве $U$ в указанном базисе задано матрицей $G$.
Тогда если $\phi$ задан матрицей $A$, то матрица сопряженного линейного отображения будет $B^{-1} A^t G$ в вещественном случае и $B^{-1}\bar A^t G$ в комплексном.

\item Если $\phi\colon V\to V$ движение, то это значит, что $(\phi v, \phi u) = (v, u)$.
Если обозначить $\phi u $ за $w$, то получим, что $\phi$ движение, тогда и только тогда, когда $(\phi v, w) = (v, \phi^{-1}w)$ для любых $v,w\in V$.
То есть $\phi$ движение, тогда и только тогда, когда $\phi^* = \phi^{-1}$.
Таким образом движения можно выразить в терминах сопряженного оператора.

\item В общем случае даже не пытайтесь понять геометрический смысл сопряженного оператора.
Это совсем не очевидная штука.
\end{itemize}

\paragraph{Связь с двойственным пространством}

Здесь мне придется разобрать отдельно евклидов и эрмитов случай.
Пусть $V$ и $U$ -- евклидовы пространства и $\phi\colon V\to U$ -- линейное отображение.
Тогда существует двойственное линейное отображение $\phi^*\colon U^*\to V^*$ по правилу $\xi \mapsto \xi \phi$.
Кроме этого, скалярное произведение индуцирует изоморфизм $V\to V^*$ по правилу $v\mapsto ({-},v)$ и  аналогично для $U$.
В итоге получаем следующую диаграмму
\[
\xymatrix{
	{V^*}&{U^*}\ar[l]_{\phi^*}\\
	{V}\ar@{~>}[u]&{U}\ar@{~>}[u]\ar@{-->}[l]_{\psi}
}
\quad
\xymatrix{
	{({-},\psi u) = (\phi({-}), u)}&{({-}, u)}\ar@{|->}[l]\\
	{\psi u}\ar@{|->}[u]&{u}\ar@{|->}[u]\ar@{|->}[l]
}
\]
Вертикальные стрелки -- это изоморфизмы с помощью скалярного произведения, а нижняя пунктирная стрелка -- это композиция: сначала изоморфизм $U\to U^*$, потом $\phi^*$, потом обратный изоморфизм $V^* \to V$.
Давайте посмотрим, что получается в качестве отображения $\psi$.
Для этого возьмем произвольный вектор $u\in U$ и пройдем двумя путями в $V^*$.
Справа показан расчет результатов.
Получаем, что функции $({-},\psi u)$ и $(\phi({-}), u)$ на $V$ совпадают при любых $u\in U$, то есть для любого $v\in V$ и любого $u\in U$ имеем равенство $(v, \psi u) = (\phi v, u)$.
Теперь мы видим, что $\psi$ совпадает с определением сопряженного линейного отображения $\phi^*\colon U\to V$ .


Теперь давайте разберемся с комплексным случаем.
Ситуация здесь похожая, но появляется одна тонкость.
Но обо всем по порядку.
Пусть $V$ и $U$ -- эрмитовы пространства и $\phi\colon V\to U$ -- линейное отображение.
Однако, в этом случае отображение $V\to V^*$ по правилу $v\mapsto (v, {-})$ не является $\mathbb C$-линейным, а отображение $v\mapsto ({-}, v)$ не корректно, потому что результат не является $\mathbb C$-линейной функцией на $V$.
Это правится переходом к пространствам $\bar V$ и $\bar U$.
А именно, отображение $\phi\colon V\to U$ определяет отображение $\bar \phi\colon \bar V\to \bar U$ по правилу $v\mapsto \phi(v)$, то есть мы действуем так же, как и исходное отображение $\phi$.
Кроме этого отображение $V\to \bar V^*$ по правилу $({-}, v)$ является $\mathbb C$-линейным изоморфизмом.
Теперь имеем следующую диаграмму в комплексном случае
\[
\xymatrix{
	{\bar V^*}&{\bar U^*}\ar[l]_{\bar \phi^*}\\
	{V}\ar@{~>}[u]&{U}\ar@{~>}[u]\ar@{-->}[l]_{\psi}
}
\quad
\xymatrix{
	{({-},\psi u) = (\phi({-}), u)}&{({-}, u)}\ar@{|->}[l]\\
	{\psi u}\ar@{|->}[u]&{u}\ar@{|->}[u]\ar@{|->}[l]
}
\]
И как и в вещественном случае раскручивание определений приводит нас к равенству $(v, \psi u) = (\phi v, u)$ для всех $v\in V$ и $u\in U$, что означает, что $\psi$ совпадает с определением сопряженного оператора.

\subsection{Самосопряженные операторы}

Мы уже видели пример операторов согласованных со скалярными произведениями -- движения.
Сейчас я познакомлю вас с еще одним классом операторов.

\begin{definition}
Пусть $V$ евклидово или эрмитово пространство.
Тогда оператор $\phi$ называется самосопряженным, если $\phi = \phi^*$.
В евклидовом пространстве самосопряженные операторы обычно называются симметричными.
\end{definition}

\paragraph{Замечания}

\begin{itemize}
\item Если $e_1,\ldots,e_n$ -- ортонормированный базис в пространстве $V$ и оператор $\phi$ в этом базисе задан матрицей $A$, то $\phi^*$ задан матрицей $A^t$ в евклидовом случае и $\bar A^t$ в комплексном.
Потому оператор $\phi$ самосопряжен тогда и только тогда, когда $A = A^t$ в евклидовом случае и $A = \bar A^t$ в комплексном случае.

\item Если $\phi\colon V\to V$ -- некоторый оператор в евклидовом пространстве и $\phi_\mathbb C\colon V_\mathbb C\to V_\mathbb C$ -- его комплексификация.
Тогда оператор $\phi$ самосопряжен тогда и только тогда, когда $\phi_\mathbb C$ самосопряжен.
Легче всего это проверить так: выберем ортонормированный базис $e_1,\ldots,e_n$ в пространстве $V$, тогда он будет ортонормированным в пространстве $V_\mathbb C$.
Если оператор $\phi$ в этом базисе задан матрицей $A\in \Matrix{n}$, то оператор $\phi_\mathbb C$ задан той же матрицей в этом же базисе.
Но тогда условие $A = A^t$ равносильно $A = \bar A^t$.
Первое из них означает самосопряженность $\phi$, а второе самосопряженность $\phi_\mathbb C$.

\item Давайте я тут нагоню немного мистики.
Оказывается, что самосопряженные операторы являются моделью измеряемых величин в квантовой физике.
Их спектры отвечают возможным значениям от измерений.
Кроме того, с помощью скалярного произведения можно ввести вероятностную меру на спектре, что превращает наши измерения в случайные.
Тот факт, что два оператора коммутируют означает, что эти измерения можно проделать одновременно.
А если операторы не коммутируют, то значит эти измерения одновременно не производятся.
Главный пример из квантовой механики -- координата и импульс.
Измеряя все точнее координату, мы теряем точность измерения импульса и наоборот.
Все эти странные вещи можно объяснить с помощью самосопряженных операторов на некотором гильбертовом пространстве (это полные бесконечномерные евклидовы пространства).
Чтобы в полной мере изучать такие вещи, приходится изучать правильную бесконечномерную линейную алгебру.
Имя ей -- функциональный анализ.
Хотите заниматься квантовой криптографией или квантовыми вычислениями?
Знайте, такое безобразие тоже существует и эти вещи ближе, чем они кажутся.
Главное -- выучить линейную алгебру как следует.
\end{itemize}

\begin{claim}
\label{claim::SelfAdjBasicProp}
Пусть $V$ -- евклидово или эрмитово пространство и $\phi\colon V\to V$ -- самосопряженный оператор, тогда
\begin{enumerate}
\item В вещественном случае: $\spec_\mathbb R \phi\neq \varnothing$.
В комплексном случае: $\spec_\mathbb C\phi \subseteq \mathbb R$.

\item Если $\lambda\neq\mu$ -- два разных собственных значения для $\phi$, то подпространства $V_\lambda$ и $V_\mu$ ортогональные.

\item Если $U\subseteq V$ $\phi$-инвариантное подпространство, то $U^\bot$ тоже $\phi$-инвариантное подпространство.
\end{enumerate}
\end{claim}
\begin{proof}
1) Давайте в начале рассмотрим комплексный случай.
Пусть $\lambda\in \spec_\mathbb C\phi$.
Тогда для найдется ненулевой вектор $v\in V$ такой, что $\phi v = \lambda v$.
Посчитаем выражение $(\phi v, v)$ двумя способами.
С одной стороны
\[
(\phi v, v) = (\lambda v, v) = \bar \lambda (v, v)
\]
С другой стороны
\[
(\phi v, v) = (v, \phi^* v) = (v,\phi v) = (v,\lambda v) = \lambda (v, v)
\]
То есть $\bar\lambda (v, v) = \lambda (v,v)$.
Так как $v$ ненулевой, то $(v,v)\neq0$.
А это значит, что $\bar\lambda = \lambda$, то есть $\lambda \in \mathbb R$.

Теперь рассмотрим вещественный случай.
Пусть $\phi\colon V\to V$ наш самосопряженный оператор.
Рассмотрим его комплексификацию $\phi_\mathbb C\colon V\to V$.
Это самосопряженный оператор в эрмитовом пространстве.
По уже доказанному его спектр вещественный.
Но $\phi$ и $\phi_\mathbb C$ имеют одинаковые матрицы, значит их спектры совпадают.
В частности $\spec_\mathbb R \phi = \spec_\mathbb C\phi_\mathbb C\neq \varnothing$.

2) Пусть теперь $v\in V_\lambda$ и $u\in V_\mu$.
Тогда
\[
(\phi v, u) = (\lambda v, u) = \bar \lambda (v,u) = \lambda (v, u)
\]
Здесь мы воспользовались тем, что все собственные значения вещественные.
Теперь посчитаем по-другому
\[
(\phi v, u) = (v,\phi^* u) = (v, \phi u) = (v, \mu u) = \mu (v, u)
\]
То есть $\lambda (v, u) = \mu (v,u)$.
Но при этом $\lambda \neq \mu$.
Значит $(v, u) = 0$.

3) Нам надо показать, что если $w\bot U$, то $\phi(w)\bot U$.
Имеем $(\phi w, u) = (w, \phi^* u) = (w, \phi u)$ для любого $u\in U$ и $w\in U^\bot$.
В силу $\phi(U) \subseteq U$ получаем, что $\phi u \in U$.
То есть $(\phi w, u) = (w, \phi u) = 0$ для любого $u\in U$, что и требовалось.
\end{proof}


\subsection{Классификация самосопряженных операторов}

\begin{claim}
\label{claim::SelfAdjHDiag}
Пусть $V$ -- эрмитово пространство и $\phi\colon V\to V$ -- некоторый оператор.
Тогда оператор $\phi$ самосопряжен тогда и только тогда, когда
\begin{enumerate}
\item $\phi$ диагонализуем в ортонормированном базисе.

\item $\spec_\mathbb C\phi\subseteq \mathbb R$.
\end{enumerate}
\end{claim}
\begin{proof}
($\Leftarrow$).
Если в некотором ортонормированном базисе $\phi$ задан диагональной матрицей $A$ с вещественными числами на диагонали, то $\bar A^t = A^t = A$.
А значит $\phi$ самосопряжен.

($\Rightarrow$).
Утверждение~\ref{claim::SelfAdjBasicProp} пункт~(1) уже влечет, что спектр $\phi$ целиком состоит из вещественных чисел.
Нам лишь надо показать, что оператор диагонализуется.
Так как спектр не пуст, то существует собственный вектор $v\in V$ с вещественным собственным значением $\lambda$.
Тогда $V = \langle v\rangle \oplus \langle v\rangle^\bot$.
Так как $\langle v \rangle$ является $\phi$ инвариантным, то $U = \langle v\rangle^\bot$ является $\phi$ инвариантным (утверждение~\ref{claim::SelfAdjBasicProp} пункт~(3)).
А значит индукцией по размерности пространства, мы находим ортонормированный базис $e_2,\ldots,e_n$ в пространстве $U$, в котором диагонализуется оператор $\phi|_U$.
Если мы выберем $e_1 = v / |v|$, то получим искомый базис $e_1,e_2,\ldots,e_n$.
\end{proof}

\begin{claim}
\label{claim::SelfAdjHExists}
Пусть $V$ -- комплексное векторное пространство и $\phi\colon V\to V$ -- некоторый оператор.
Тогда существует скалярное произведение на $V$ такое, что $\phi$ становится самосопряженным оператором тогда и только тогда, когда
\begin{enumerate}
\item $\phi$ диагонализуем.

\item $\spec_\mathbb C\phi\subseteq \mathbb R$.
\end{enumerate}
\end{claim}
\begin{proof}
($\Rightarrow$).
Если существует такое скалярное произведение, то все следует из предыдущего утверждения~\ref{claim::SelfAdjHDiag}.

($\Leftarrow$).
Пусть теперь $e_1,\ldots,e_n$ -- базис в котором $\phi$ диагонализуем.
Тогда зададим скалярное произведение так, чтобы этот базис был ортонормированным.
В этом случае выполнена пара условий из утверждения~\ref{claim::SelfAdjHDiag} и мы получаем, что $\phi$ самосопряжен относительно построенного скалярного произведения.
\end{proof}


\begin{claim}
\label{claim::SelfAdjEDiag}
Пусть $V$ -- евклидово пространство и $\phi\colon V\to V$ -- некоторый оператор.
Тогда $\phi$ самосопряжен тогда и только тогда, когда $\phi$ диагонализуем в ортонормированном базисе.
\end{claim}
\begin{proof}
Доказательство этого утверждения один в один повторяет слова доказательства в комплексном случае (утверждение~\ref{claim::SelfAdjHDiag}).
Единственное обратите внимание, что тут мы просто пользуемся тем, что по утверждению~\ref{claim::SelfAdjBasicProp} пункт~(1), спектр самосопряженного оператора не пуст.
\end{proof}

\begin{claim}
Пусть $V$ -- некоторое вещественное пространство и $\phi\colon V\to V$ -- некоторый оператор.
Тогда существует скалярное произведение такое, что $\phi$ самосопряжен, тогда и только тогда, когда $\phi$ диагонализуем.
\end{claim}
\begin{proof}
Доказательство слово в слово повторяет комплексный случай (утверждение~\ref{claim::SelfAdjHExists}).
\end{proof}


\subsection{Билинейные формы и операторы}
\label{section::BilinOper}

В евклидовом пространстве есть тесная связь между линейными операторами и билинейными формами.
Аналогичная связь есть в эрмитовых пространствах, но уже между операторами и полуторалинейными формами.
Этот механизм позволяет связать характеристики оператора с характеристиками билинейной (или полуторалинейной) формы.
Я собираюсь обсудить эту связь.


Пусть $V$ -- евклидово или эрмитово пространство и пусть задан линейный оператор $\phi\colon V\to V$.
Тогда он определяет билинейную (полуторалинейную) форму на $V$ по правилу $\beta_\phi(v, u) = (v, \phi u)$.
Таким образом получается отображение $\Hom_\mathbb R(V, V)\to \Bil(V)$ в евклидовом случае и $\Hom_\mathbb C(V, V)\to \Bil_{1\frac{1}{2}}(V)$ в эрмитовом.
Давайте теперь посмотрим как это отображение выглядит в некотором базисе.

Я для определенности разберу комплексный случай, вещественный делается аналогично.
Пусть $e_1,\ldots,e_n$ -- ортонормированный базис $V$, тогда $V$ превращается в $\mathbb C^n$, скалярное произведение превращается в стандартное $(x, y) = \bar x^t y$ и оператор задается $\phi x = Ax$ для некоторой $A\in \operatorname{M}_n(\mathbb C)$.
В этом случае билинейная форма $\beta_\phi$ имеет вид $\beta_\phi(x,y) = \bar x^t A y$, то есть задается той же самой матрицей.
Таким образом мы видим, что в случае выбора ортонормированного базиса, отображение из операторов в билинейные (полуторалинейные) формы превращается в тождественное отображение на матрицах.%
\footnote{Важно иметь в виду, что при выборе не ортонормированного базиса, отображение будет иметь вид $A\mapsto G A$, где $G$ -- матрица скалярного произведения в выбранном базисе.}
В частности, построенное отображение из операторов в билинейные (полуторалинейные) формы является биекцией и даже изоморфизмом векторных пространств.
Что по простому означает следующее: для любой билинейной (полуторалинейной) формы $\beta$ существует единственный линейный оператор $\phi\colon V\to V$ такой, что $\beta(v,u) = (v, \phi u)$.

Обратите внимание, что в вещественном случае оператор $\phi\colon V\to V$ самосопряжен тогда и только тогда, когда $\beta_\phi$ симметрична.
В комплексном случае $\phi$ самосопряжен тогда и только тогда, когда $\beta_\phi$ эрмитова.
Действительно, давайте проверим это для комплексного случая.
Если выбрать ортонормированный базис и оператор $\phi$ превращается в $\phi x = A x$, то условие самосопряженности оператора -- это условие $\bar A^t = A$.
При этом полуторалинейная функция будет иметь вид $\beta_\phi(x, y) = \bar x^t A y$ и условие ее эрмитовости -- это опять же условие $\bar A^t = A$.
Вот и все.%
\footnote{Это условие можно было бы проверять без базиса.
По определению $\beta_\phi(v, u) = (v, \phi u)$, кроме того $\overline{\beta_\phi(u, v)} = \overline{(u, \phi v)} = (\phi v, u) = (v, \phi^* u)$.
Таким образом равенство этих двух выражений совпадает с определением эрмитовости $\beta_\phi$ и самосопряженности $\phi$.}
У этого наблюдения про формы и операторы есть два полезных следствия.

\begin{claim}
\label{claim::BilinOrthoDiag}
Пусть $V$ -- евклидово или эрмитово пространство и $\beta$ -- симметричная билинейная (или эрмитова) форма.
Тогда существует ортонормированный базис $e_1,\ldots,e_n$, в котором $\beta$ задается диагональной матрицей.
\end{claim}
\begin{proof}
Пусть $\phi\colon V\to V$ такой оператор, что $\beta(v, u) = (v, \phi u)$.
Так как $\beta$ симметрична (эрмитова) оператор $\phi$ является самосопряженным.
Пусть $e_1,\ldots,e_n$ -- ортонормированный базис, в котором $\phi$ диагонализуется.
Тогда он задан в этом базисе диагональной матрицей $A = \diag(\lambda_1,\ldots,\lambda_n)$.
Так как базис $e_1,\ldots,e_n$ ортонормированный, то $\beta$ задана той же самой матрицей $A$.
\end{proof}

На это утверждение можно смотреть следующим образом.
Давайте обсудим его в вещественном случае.
Пусть у вас на пространстве $V$ задано две симметричные билинейные формы, причем одна из них положительно определена.
Тогда существует базис, в котором обе формы задаются диагональными матрицами.
В такой форме это утверждение возникает в дифференциальной геометрии для первой и второй квадратичной формы.

\begin{claim}
Пусть 
\begin{enumerate}
\item
$V$ -- вещественное пространство, $\beta\colon V\times V\to \mathbb R$ -- симметричная билинейная форма и пусть в базисе $e_1,\ldots,e_n$ форма имеет вид $\beta(x, y) = x^t B y$.

\item
$V$ -- комплексное пространство, $\beta\colon V\times V\to \mathbb C$ -- эрмитова форма и пусть в базисе $e_1,\ldots,e_n$ форма имеет вид $\beta(x, y) = \bar x^t B y$.

\end{enumerate}
Тогда характеристический многочлен $B$ раскладывается на линейные множители с вещественными коэффициентами и сигнатура $B$ определяется по знакам собственных значений матрицы $B$, а именно: $\#1$ совпадает с количеством положительных корней $\chi_B(t)$, $\#-1$ совпадает с количеством отрицательных корней $\chi_B(t)$, а $\#0$ совпадает с кратностью нуля в $\chi_B(t)$.
\end{claim}
\begin{proof}
Давайте введем в пространстве $V$ скалярное произведение такое, что $e_1,\ldots,e_n$ становится ортонормированным базисом.
Тогда матрица $B$ задает линейный оператор $\phi\colon V \to V$ такой, что $\beta(v, u) = (v, \phi u)$.
Так как $\beta$ симметричная (эрмитова), то $\phi^*$ самосопряжен.
Мы можем найти новый ортонормированный базис $f_1,\ldots,f_n$, в котором $\phi$ диагонализуется $\diag(\lambda_1,\ldots,\lambda_n)$.
Тогда числа на диагонали -- это спектр матрицы $B$.
С другой стороны, $\beta$ в базисе $f_1,\ldots,f_n$ тоже задается матрицей $\diag(\lambda_1,\ldots,\lambda_n)$, а значит ее сигнатура определяется по количеству положительных, отрицательных и нулевых числе среди корней характеристического многочлена с учетом кратности.
\end{proof}

\paragraph{Замечание}

Пусть нам задан оператор $\phi\colon \mathbb R^n\to \mathbb R^n$ с помощью симметричной матрицы $A\in \operatorname{M}_n(\mathbb R)$.
Тогда $\phi$ самосопряжен относительно стандартного скалярного произведения и по нему строится симметричная билинейная форма $\beta_\phi(x, y) = (x, \phi (y))$.
По предыдущему утверждению мы знаем, что сигнатура $\beta_\phi$ совпадает со знаками собственных значений $\phi$.
Обратите внимание, что сигнатуру находить проще, чем знаки собственных значений, для ее поиска достаточно алгоритма Гаусса (симметричной версии).
Таким образом у нас появляется алгоритм для матрицы $A$ по определению количества положительных и отрицательных собственных значений -- для этого надо у матрицы $A$ определить сигнатуру как у билинейной формы.
Теперь заметим, что если мы применим этот метод к матрице $A - \lambda E$, то мы можем определить количество собственных значений больше и меньше $\lambda$.
А значит у нас есть метод по определению количества собственных значений на любом отрезке.
