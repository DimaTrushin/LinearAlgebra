\ProvidesFile{lecture20.tex}[Лекция 20]


\subsection{Геометрический смысл кратности корней минимального и характеристического многочлена}

\begin{claim}
\label{claim::RootMultGeom}
Пусть $\varphi\colon V \to V$ -- линейный оператор и $\lambda \in \spec_F \varphi$.
Пусть число $k$ выбрано так, что
\[
\ker (\varphi-\lambda\Identity)^{k-1} \neq \ker (\varphi-\lambda\Identity)^k = \ker (\varphi-\lambda\Identity)^{k+1}
\]
Тогда
\begin{enumerate}
\item Число $k$ -- это кратность $\lambda$ в минимальном многочлене оператора $\varphi$.

\item Число $\dim \ker(\varphi - \lambda \Identity)^k = \dim V^\lambda$ -- это кратность корня $\lambda$ в характеристическом многочлене оператора $\varphi$.
\end{enumerate}
\end{claim}
\begin{proof}
(1) Это частный случай утверждения~\ref{claim::GenRootDec} пункт~(1) для $p_1(x) = x-\lambda$.

(2) Пусть $f_\text{min} = (x-\lambda)^k g(x)$ -- минимальный многочлен для $\varphi$ на $V$.
Тогда пользуясь утверждением~\ref{claim::IdealRootDec} пункт~(3) и предыдущим пунктом этого утверждения, мы видим, что $V = V^\lambda \oplus \ker g(\varphi)$.
Теперь давайте посчитаем характеристический многочлен для $\varphi$.
По утверждению~\ref{claim::RestrictionChar} $\chi_\varphi(t)$ есть произведение $\chi_{\varphi|_{V^\lambda}}$ и $\chi_{\varphi|_{\ker g(\varphi)}}$.
Утверждение~\ref{claim::CharPolyOnRootSpace} гласит, что $\chi_{\varphi|_{V^\lambda}}(t) = (t - \lambda)^{\dim V^\lambda}$.
А утверждение~\ref{claim::CoprimeKernels} пункт~(2), что $\varphi - \lambda\Identity$ обратим на $\ker g(\varphi)$.
Значит, оператор $\varphi|_{\ker g(\varphi)}$ не содержит $\lambda$ в своем спектре, а значит $\lambda$ не корень $\chi_{\varphi|_{\ker g(\varphi)}}$.
Это завершает доказательство.
\end{proof}

\subsection{Минимальные инвариантные}

Пусть $\varphi\colon V\to V$ -- некоторый оператор и $p\in \spec^I_F\varphi$.
В этом случае подпространство $V_p = \ker p(\varphi)$ не нулевое.
Более того, по определению оператор $p(\varphi)$ равен нулю на этом подпространстве, а значит $p$ зануляет $\varphi|_{V_p}$.
В частности минимальный многочлен $\varphi|_{V_p}$ должен делить $p$.
Но так как $p$ неприводим, то единственный вариант -- минимальный многочлен совпадает с $p$.
Аналогично, если $U\subseteq V_p$ произвольное ненулевое инвариантное подпространство, то минимальный многочлен $\varphi_U$ будет $p$ по тем же самым соображениям.
Мы знаем, что в случае обычного спектра собственное подпространство $V_\lambda$ состоит из инвариантных прямых, на которых $\varphi$ действует растяжением в $\lambda$ раз.
В случае $V_p$ все подпространство состоит из одинаковых инвариантных кусочков, которые уже не являются прямыми.
Давайте опишем как эти инвариантные подпространства выглядят.

\begin{claim}
Пусть $\varphi\colon V\to V$ -- некоторый оператор и $p\in \spec^I_F\varphi$ и при этом $m = \deg p$.
И пусть $v\in V_p$ -- произвольный ненулевой вектор.
Тогда $v,\varphi v,\ldots, \varphi^{m-1}v$ линейно независимы и их линейна оболочка будет минимальным инвариантным подпространством содержащем $v$.
\end{claim}
\begin{proof}
Давайте рассмотрим линейную оболочку
\[
U = \langle v, \varphi v, \varphi^2 v, \ldots, \varphi^k v, \ldots\rangle
\]
По построению ясно, что это инвариантное подпространство содержащее $v$.
Кроме того, если какое-то инвариантное подпространство $W$ содержит $v$, то оно обязано содержать $\varphi v$.
А значит обязано содержать $\varphi^2 v$ и так далее.
То есть оно содержит $U$.
Таким образом $U$ является наименьшим инвариантным подпространством содержащим $v$.

Теперь рассмотрим элемент $p = t^m + a_{m-1}t^{m-1} + \ldots + a_1 t + a_0$.
Подставим в него $\varphi$ и получим ноль, то есть
\[
0 = p(\varphi) = \varphi^m + a_{m-1} \varphi^{m-1} + \ldots + a_1 \varphi + a_0 \Identity
\]
Теперь применим это равенство к вектору $v$, получим
\[
\varphi^mv + a_{m-1} \varphi^{m-1}v + \ldots + a_1 \varphi v + a_0 v = 0
\]
А значит $\varphi^m v$ выражается через $\varphi^k v$, где $k < m$.
Умножая это равенство на $\varphi^N$ мы видим, что при $N \geqslant m$, $\varphi^N v$ выражается через $\varphi^k v$, где $k < N$.
Следовательно подпространство $U$ порождается векторами $v, \varphi v, \ldots, \varphi^{m-1} v$.

Теперь покажем, что эти векторы линейно независимы.
Предположим противное, то есть найдется система коэффициентов $\alpha_0,\ldots, \alpha_{m-1}$ такая, что
\[
\alpha_0 v + \alpha_1 \varphi v + \ldots + \alpha_{m-1} \varphi^{m-1} v = 0
\]
Если мы положим $q(t) = \alpha_0 + \alpha_1 t + \ldots + \alpha_{m-1} t^{m-1}$, то это означает, что $q(\varphi)v = 0$.
Так как $q(\varphi)$ и $\varphi^k$ коммутируют для любого $k$, то $q(\varphi) \varphi^k v = \varphi^k q(\varphi) v = 0$ для любого $k$.
А это значит, что $q$ зануляет $\varphi|_U$.
Но по замечанию перед утверждением мы знаем, что минимальный многочлен для $\varphi|_U$ должен быть $p$.
А значит $p$ делит $q$ и так как $\deg q < \det p$, такое возможно только если $q = 0$.
То есть если все $\alpha_i = 0$, что и требовалось.
\end{proof}

Таким образом в случае $\lambda \in \spec_F \varphi$ ему соответствует линейный многочлен $t - \lambda \in \spec^I_F\varphi$.
А значит линейному многочлену соответствуют одномерные инвариантные подпространства в $V_\lambda$.
В случае если элемент идеального спектра $p$ имеет степень больше единицы, то $V_p$ содержит инвариантные размерности $\deg p$.
На самом деле можно показать, что все такие инвариантные подпространства внутри $V_p$ <<одинаковые>> в некотором смысле и все $V_p$ есть их прямая сумма.

\subsection{Структура векторного пространства с оператором}

Изучение структуры матрицы линейного оператора в некотором базисе равносильна изучению инвариантных подпространств пространства $V$.
Теорема о жордановой нормальной форме может рассматриваться таким образом как структурная теорема для векторного пространства с оператором.
В общем виде теорема о жордановой нормальной форме доказывается в два шага: все пространство раскладывается в прямую сумму корневых, после чего задача сводится к нильпотентному оператору.
Первый шаг мы уже на самом деле проделали в утверждении~\ref{claim::GenRootDec}.

\begin{claim}
\label{claim::RootSpaceDec}
Пусть $\varphi\colon V\to V$ -- линейный оператор такой, что его характеристический (или минимальный) многочлен раскладывается на линейные множители.
Тогда $V = V^{\lambda_1} \oplus \ldots\oplus V^{\lambda_r}$, где $\spec_F \varphi = \{\lambda_1,\ldots, \lambda_r\}$.
\end{claim}
\begin{proof}
Это частный случай утверждения~\ref{claim::GenRootDec}.
\end{proof}

\paragraph{Замечание}

Давайте объясним, что мы доказали на данный момент.
Пусть $\varphi\colon V\to V$ -- некоторый линейный оператор, у которого характеристический многочлен раскладывается на линейные множители.
Тогда $V = V^{\lambda_1}\oplus\ldots\oplus V^{\lambda_r}$.
Выберем базис $e_i$ в каждом $V^{\lambda_i}$.
Тогда по одному из определений прямой суммы $e = e_1 \sqcup \ldots \sqcup e_r$ будет базисом $V$.
Так как все $V^{\lambda_i}$ инвариантны относительно $\varphi$, то когда мы запишем его матрицу в этом базисе, мы получим блочно диагональную матрицу вида
\[
A_\varphi = 
\begin{pmatrix}
{A_1}&{}&{}&{}\\
{}&{A_2}&{}&{}\\
{}&{}&{\ddots}&{}\\
{}&{}&{}&{A_r}\\
\end{pmatrix}
\]
где $A_i$ -- это матрица ограничения $\varphi|_{V^{\lambda_i}}$, то есть ее размер равен $\dim V^{\lambda_i}$.
Кроме того, из утверждения~\ref{claim::OperatorUpperTriangle} следует, что в каждом $V^{\lambda_i}$ можно найти такой базис, что матрица $A_i$ будет верхне треугольной с числом $\lambda_i$ на диагонали, то есть
\[
A_i = 
\begin{pmatrix}
{\lambda_i}&{*}&{\ldots}&{*}\\
{}&{\lambda_i}&{\ldots}&{*}\\
{}&{}&{\ddots}&{\vdots}\\
{}&{}&{}&{\lambda_i}\\
\end{pmatrix}
\]
Наша цель еще улучшить вид матриц $A_i$.
Оказывается, почти все элементы верхнего блока можно сделать нулевыми.
Что это в точности означает и как доказывается, вы узнаете в теореме о жордановой нормальной форме.


\subsection{Отношение равенства по модулю подпространства}

\begin{definition}
Пусть $V$ -- векторное пространство, а $U\subseteq V$ -- подпространство.
Тогда будем говорить, что векторы $v,w\in V$ равны по модулю $U$ и писать $v = w \pmod U$, если $v - w \in U$.%
\footnote{На самом деле равенство по модулю подпространства сводится к равенству в некотором новом пространстве, которое называется фактор пространством.
Так как мы пока не знаем, что это такое, будем пользоваться лишь отношением равенства по модулю.
Думать про него надо так же, как и про остатки в целых числах.}
\end{definition}

Например, если $V = \mathbb R^2$ -- плоскость и $U = \langle e_1\rangle$ -- горизонтальная прямая, то все векторы лежащие на горизонтальных прямых между собой равны по модулю $U$.
Например, $e_2 = e_2 + e_1 = e_2 - 3 e_1 \pmod U$.
Но $e_2 \neq 2 e_2 \pmod U$.

Заметим, что обычное равенство -- это равенство по модулю нулевого подпространства.
С другой стороны, по модулю подпространства $U = V$ любые два вектора равны.

\begin{definition}
Пусть $V$ -- векторное пространство,  $U\subseteq V$ -- некоторое подпространство, и $v_1,\ldots,v_n\in V$ -- набор векторов.
\begin{enumerate}
\item Будем говорить, что $v_1,\ldots,v_n$ линейно независимы по модулю $U$, если из равенства $\alpha_1 v_1 + \ldots + \alpha_n v_n = 0 \pmod U$ следует, что $\alpha_1 = \ldots = \alpha_ n = 0$.

\item Будем говорить, что $v_1, \ldots, v_n$ порождающие по модулю $U$, если для любого вектора $v\in V$ найдутся коэффициенты $\alpha_1,\ldots,\alpha_n\in F$ такие, что $v = \alpha_1 v_1 + \ldots + \alpha_n v_n\pmod U$.

\item Будем говорить, что $v_1,\ldots,v_n$ являются базисом $V$ по модулю $U$, если они одновременно линейно независимы и порождающие по модулю $U$.%
\footnote{Так как равенство по модулю сводится к равенству в некотором новом пространстве, то все факты про базис аналогичные обычным фактам, что мы доказывали будут верны.
Вам же я предлагаю доказать их по аналогии в качестве упражнения.}
\end{enumerate}
\end{definition}

\begin{claim}
Пусть $V$ -- векторное пространство, $U\subseteq V$ -- подпространство, и $v_1,\ldots,v_n\in V$ -- набор векторов.
Тогда
\begin{enumerate}
\item Векторы $v_1,\ldots,v_n$ линейно независимы по модулю $U$ тогда и только тогда, когда они линейно независимы и $\langle v_1,\ldots,v_n \rangle \cap U = 0$.

\item Векторы $v_1, \ldots,v_n$ порождающие по модулю $U$ тогда и только тогда, когда $\langle v_1,\ldots,v_n\rangle + U = V$.

\item Векторы $v_1, \ldots,v_n$ являются базисом по модулю $U$ тогда и только тогда, когда они линейно независимы и $\langle v_1,\ldots,v_n\rangle \oplus U = V$.
\end{enumerate}
\end{claim}
\begin{proof}
(1) $\Rightarrow$ Пусть $\alpha_1v_1+\ldots+\alpha_n v_n = 0$, тогда $\alpha_1 v_1 + \ldots+ \alpha_n v_n \in U$.
Последнее означает, что $\alpha_1v_1 + \ldots+ \alpha_n v_n = 0 \pmod U$.
А значит все $\alpha_i = 0$.
Значит $v_i$ линейно независимы.
Теперь рассмотрим вектор $v\in \langle v_1,\ldots,v_n\rangle \cap U$.
Так как $v$ лежит в первом подпространстве, то $v = \alpha_1 v_1 + \ldots+\alpha_n v_n$.
Так как он лежит в правом подпространстве, то $\alpha_1 v_1 + \ldots+ \alpha_n v_n = v \in U$.
Значит $\alpha_1 v_1 + \ldots + \alpha_n v_n = 0\pmod U$.
А следовательно все $\alpha_i =0$.
Но значит и $v = 0$, что и требовалось.

$\Leftarrow$ Пусть $\alpha_1 v_1 +\ldots+\alpha_n v_n = 0 \pmod U$.
Это значит, что $\alpha_1 v_1 + \ldots + \alpha_n v_n \in U$.
А значит $\alpha_1 v_1 +\ldots + \alpha_n v_n \in \langle v_1,\ldots,v_n\rangle \cap U = 0$.
То есть $\alpha_1 v_1 + \ldots + \alpha_n v_n = 0$.
Но так как $v_i$ линейно независимы, то $\alpha_i = 0$ для всех $i$.

(2) $\Rightarrow$ По определению, для любого $v\in V$ найдутся коэффициенты $\alpha_i$ такие, что $v = \alpha_1 v_1 + \ldots + \alpha_n v_n \pmod U$.
То есть $v - (\alpha_1 v_1 + \ldots + \alpha_n v_n) = u \in U$.
Значит, $v = \alpha_1 v_1 + \ldots + \alpha_n v_n + u$, что и требовалось.

$\Leftarrow$ Пусть $v\in V = \langle v_1, \ldots, v_n \rangle + U$.
Тогда $v = \alpha_1 v_1 + \ldots + \alpha_n v_n + u$.
По определению это означает, что $v = \alpha_1 v_1 + \ldots + \alpha_n v_n \pmod U$.

(3) Этот пункт получается из первых двух вместе взятых.
\end{proof}
