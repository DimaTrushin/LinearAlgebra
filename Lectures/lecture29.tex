\ProvidesFile{lecture29.tex}[Лекция 29]


\subsection{Объемы}

Теперь самое время прикоснуться к объемам.
Надо сказать, что существует общая теория вычисления объемов в евклидовом пространстве $V$.
Она позволяет посчитать <<объем>> любого подмножества в $V$.
Данная конструкция ведет к понятию меры Лебега и далее к интегралу Лебега.
Мы, конечно же, не будем развивать подобную теорию в такой общности, а всего лишь ограничимся вычислением объемов для простых и естественных с точки зрения линейной алгебры фигур -- многомерных параллелепипедов.

\begin{definition}
Пусть $V$ -- евклидово пространство и $v_1,\ldots,v_k\in V$ -- набор векторов, тогда $k$-мерным параллелепипедом натянутым на $v_1,\ldots,v_k$ называется следующее множество
\[
\Pi(v_1,\ldots,v_k) = \Bigl\{\sum_{i=1}^k x_i v_i\mid 0\leqslant x_i \leqslant 1\Bigl\}
\]
\end{definition}

\paragraph{VIP пример}

Я хочу разобрать один важный пример.
Пусть $V = \mathbb R^2$ -- плоскость со стандартным скалярным произведением $(x, y) = x^t y$ и $v_1,v_2\in V$ -- два ортогональных вектора.
Если я заменю $v_2$ на $v_2 + \lambda v_1$, то геометрически я наклоню мой параллелепипед вдоль направления $v_1$ как на рисунке ниже:
\[
\begin{aligned}
\xymatrix{
	{}\ar@{-}[rr]&{}&{}\\
	{}\ar[u]^{v_2}\ar[rr]^{v_1}&{}&{}\ar@{-}[u]\\
}
\end{aligned}
\quad\longrightarrow\quad
\begin{aligned}
\xymatrix{
	{}\ar@{--}[r]&{}\ar@{-}[rr]&{}&{}\\
	{}\ar@{--}[u]\ar[rr]^{v_1}\ar[ru]|{v_2 + \lambda v_1}&{}&{}\ar@{--}[u]\ar@{-}[ru]&{}\\
}
\end{aligned}
\]
На рисунке мы видим, что параллелепипед справа отличается от параллелепипеда слева перестановкой треугольника отмеченного пунктиром.
А значит их площади одинаковые.
Давайте посмотрим на матрицы Грама двух наборов векторов:
\[
G(v_1, v_2 + \lambda v_1) = 
\begin{pmatrix}
{1}&{0}\\
{\lambda}&{1}
\end{pmatrix}
G(v_1,v_2)
\begin{pmatrix}
{1}&{\lambda}\\
{0}&{1}
\end{pmatrix}
\quad\text{при этом}\quad
G(v_1,v_2) =
\begin{pmatrix}
{|v_1|^2}&{0}\\
{0}&{|v_2^2|}
\end{pmatrix}
\]
То есть $\det G(v_1,v_2 + \lambda v_1) = \det G(v_1,v_2) = |v_1|^2 |v_2|^2 = S_{\Pi(v_1,v_2)}^2$.
То есть определитель матрицы Грамма дает нам квадрат площади параллелограмма, натянутого на векторы $v_1$, $v_2$.
Этот пример подсказывает, как надо определять объемы в общем случае.

\begin{definition}
Пусть $v_1,\ldots, v_k\in V$ -- произвольный набор векторов в евклидовом пространстве.
Тогда определим $k$-мерный объем параллелепипеда $\Pi(v_1,\ldots,v_k)$ по следующей формуле:
\[
\Vol_k(\Pi(v_1,\ldots,v_k)) = \sqrt{\det G(v_1,\ldots,v_k)}
\]
\end{definition}

Давайте теперь покажем, что объем $k$-мерного параллелепипеда можно считать через площадь ($k-1$-мерный объем) основания на высоту.

\begin{claim}
Пусть $v_1,\ldots,v_k\in V$ -- произвольный набор векторов в евклидовом пространстве.
Тогда
\[
\Vol_k(\Pi(v_1,\ldots,v_k)) = \Vol_{k-1}(\Pi(v_1,\ldots,v_{k-1}))\rho(v_k, \langle v_1,\ldots,v_{k-1}\rangle)
\] 
\end{claim}
\begin{proof}
Пусть $h = \ort_{\langle v_1,\ldots,v_{k-1}\rangle} v_k$.
Тогда мы знаем, что $h = v_k - \sum_{i=1}^{k-1}\alpha_i v_i$ для каких-то коэффициентов $\alpha_i$ (например это следует из алгоритма Грама-Шмидта).
Значит по утверждению~\ref{claim::GramMatrixFull} пункт~(4),
\[
\det G(v_1,\ldots,v_{k-1}, h) = \det G(v_1,\ldots,v_{k-1},v_k)
\]
Так как $h \bot v_i$, то 
\[
G(v_1,\ldots,v_{k-1}, h) = 
\begin{pmatrix}
{G(v_1,\ldots,v_{k-1})}&{0}\\
{0}&{(h, h)}
\end{pmatrix}
\]
То есть 
\[
\det G(v_1,\ldots,v_{k-1}, h) = \det G(v_1,\ldots, v_{k-1}) (h, h)
\]
\end{proof}


Последняя формула позволяет нам вычислять расстояние от вектора до подпространства с помощью объемов.
А именно, если $v\in V$ и $L\subseteq V$ -- подпространство с базисом $e_1,\ldots,e_k$, то 
\[
\rho(v, L) = \sqrt{\frac{\det G(e_1,\ldots,e_k, v)}{\det G(e_1,\ldots,e_k)}}
\]

\paragraph{Пример}

Давайте рассмотрим векторное пространство $\mathbb R^n$ со стандартным скалярным произведением $(x, y) = x^ty$.
И пусть даны векторы $v_1,\ldots,v_n\in \mathbb R^n$.
Сложим эти векторы в матрицу $A = (v_1 | \ldots | v_n)$.
Тогда
\[
\Vol_n(\Pi(v_1,\ldots, v_n)) = \sqrt{\det(A^t A)} = |\det A|
\]
Таким образом в ортонормированном базисе теория неориентированного объема превращается в теорию вычисления модуля определителя.

\begin{claim}
Пусть $V$ -- евклидово пространство размерности $n$, $v_1,\ldots,v_n\in V$ -- некоторый набор векторов и $\varphi\colon V\to V$ -- линейный оператор.
Тогда 
\[
\Vol_n(\varphi(\Pi(v_1,\ldots,v_n))) = |\det \varphi| \Vol_n(\Pi(v_1,\ldots,v_n))
\]
\end{claim}
\begin{proof}
Давайте выберем произвольный ортонормированный базис $e_1,\ldots,e_n\in V$.
Тогда $V$ превращается в $\mathbb R^n$, скалярное произведение становится стандартным $(x, y) = x^t y$, линейное отображение превращается в умножение на матрицу $\varphi(x) = Cx$, а векторы $v_1,\ldots,v_n$ расставим по столбцам матрицы $A = (v_1|\ldots|v_n)$.%
\footnote{Заметьте, что мы избрали мучительный и болезненный путь расчета в координатах.
У вас может появиться соблазн найти другое доказательство.
Однако, увы и ах.
Так как в формулировке присутствует определитель отображения, то у нас нет другого способа, потому что невозможно определить $\det \varphi$ без базиса.
Его свойствами пользоваться можно без базиса, но посчитать нет.}

Заметим, что $\varphi(\Pi(v_1,\ldots,v_n)) =  \Pi(\varphi(v_1),\ldots,\varphi(v_n))$ -- параллелепипед натянутый на столбцы матрицы $CA$.
Теперь можно воспользоваться предыдущим примером и увидеть, что
\[
\Vol_n(\Pi(v_1,\ldots, v_n)) = |\det A|\quad\text{и}\quad
\Vol_n(\varphi(\Pi(v_1,\ldots,v_n))) = |\det (CA)|=|\det C| |\det A|
\]
А так как $\det C = \det \varphi$ по определению, то все доказано.
\end{proof}

\begin{claim}
\label{claim::Volume}
Пусть $v_1,\ldots,v_k\in V$ -- набор векторов в евклидовом пространстве и $C\in \Matrix{k}$ -- произвольная матрица.
Тогда
\[
\Vol_k\Pi((v_1,\ldots,v_k)C) = |\det C| \Vol_k\Pi(v_1,\ldots,v_k)
\]
\end{claim}
\begin{proof}
По формуле для замены матрицы Грама (раздел~\ref{subsection::Gram}) мы знаем, что
\[
G((v_1,\ldots,v_k)C) = C^t G(v_1,\ldots,v_k) C
\]
Значит $\det G((v_1,\ldots,v_k)C) = \det C^2 \det G(v_1,\ldots,v_k)$.
Откуда следует требуемое по определению объемов.
\end{proof}

Таким образом, при линейной замене образующих параллелепипеда мы получаем другой параллелепипед, объем которого меняется на модуль определителя матрицы замены.
Мы хотим избавиться от модуля в этой формуле, чтобы объем мог быть положительным и отрицательным.


\subsection{Ориентированный объем}

\paragraph{Конструкция ориентированного объема}

Пусть $V$ -- евклидово пространство размерности $n$.
Рассмотрим все возможные наборы из $n$ векторов -- $V^n$.
Множество $V^n$ разбивается на две части: когда набор $(v_1,\ldots,v_n)$ линейно зависим и когда он линейно независим.
В первом случае $\Vol_n\Pi(v_1,\ldots,v_n) = 0$, а во втором $\Vol_n\Pi(v_1,\ldots,v_n)\neq 0$.
Мы хотим поделить все линейно независимые наборы (то есть базисы) на два класса: для одного класса объемы будут положительные, а для другого -- отрицательные.

\begin{definition}
Пусть $(v_1,\ldots,v_n)$ и $(u_1,\ldots,u_n)$ -- два базиса пространства $V$.
Тогда существует единственная матрица $C\in \Matrix{n}$ такая, что $(v_1,\ldots,v_n) = (u_1,\ldots,u_n)C$.%
\footnote{Потому что любой вектор однозначно раскладывается по базису.}
Будем говорить, что $(v_1,\ldots,v_n)$ эквивалентно $(u_1,\ldots,u_n)$ и писать $(v_1,\ldots,v_n)\sim(u_1,\ldots,u_n)$, если $\det C > 0$.
\end{definition}

\begin{claim}
Пусть $V$ -- евклидово пространство размерности $n$.
Тогда
\begin{enumerate}
\item Отношение, введенное на базисах, является отношением эквивалентности на множестве всех упорядоченных базисов
\[
G_n(V) = \{(v_1,\ldots,v_n)\mid v_i\in V,\;v_i\text{ линейно независимы }\}
\]

\item Множество упорядоченных базисов $G_n(V)$ разбивается на два класса эквивалентности.
\end{enumerate}
\end{claim}
\begin{proof}
(1) Если $v \in G_n(V)$ -- некоторый набор, то $v = v E$, а $\det E = 1 > 0$.
Значит $v \sim v$.
Если $v = u C$, то $u = vC^{-1}$.
Но знак у $\det C$ такой же как и у $\det C^{-1}$.
Значит если $v \sim u$, то $u\sim v$.
Теперь пусть $v\sim u$ и $u\sim w$, то есть $v=  u C$ и $u = w D$.
В этом случае $v = wDC$, то $\det (DC) =\det D \det C >0$, то есть $v\sim w$.

(2) Пусть $v,u, w\in G_n(V))$, и пусть $v\not\sim u$ и $v\not\sim w$.
Покажем, что в этом случае $u \sim w$.
Действительно $v\not\sim u$ означает, $v = u C$ и $\det C < 0$.
Аналогично, $v = w D$ и $\det D < 0$.
Тогда $u = w DC^{-1}$ и $\det(DC^{-1}) > 0$, то есть $u\sim w$.
Таким образом классов эквивалентности не более двух.
С другой стороны.
Если набор $v$ лежит в одном классе, а $C$ -- произвольная обратимая матрица с отрицательным определителем, то набор $v C$ лежит в другом классе.
Значит их в точности два.
\end{proof}

Таким образом все базисы у нас поделились на две группы.
Какую-то из этих групп нам надо назвать положительной, другую отрицательной.
Какую выбрать -- это наша свобода.
После подобного выбора все наборы в $V^n$ делятся на три группы: (1) линейно зависимые, у них объем ноль, (2) положительные, натянутые на них параллелепипеды имеют положительный объем, (3) отрицательные, натянутые на них параллелепипеды имеют отрицательный объем.
Положительные базисы будем еще называть положительно ориентированными, а отрицательные -- отрицательно ориентированными.
Если зафиксированы положительные и отрицательные базисы, будем говорить, что на $V$ зафиксирована ориентация.
% TO DO
% Может быть это стоит разбить в определения!

\begin{definition}
Пусть $V$ -- евклидово пространство с фиксированной ориентацией.
Тогда определим ориентированный $n$-мерный объем следующим образом.
Пусть $(v_1,\ldots,v_n)\in V$, тогда
\[
\Vol^{or}_n\Pi(v_1,\ldots,v_n) = 
\left\{
\begin{aligned}
0&,& &v_1,\ldots,v_n\text{ линейно зависимы}\\
\Vol_n\Pi(v_1,\ldots,v_n)&,& &(v_1,\ldots,v_n)\text{ положительно ориентирован}\\
-\Vol_n\Pi(v_1,\ldots,v_n)&,& &(v_1,\ldots,v_n)\text{ отрицательно ориентирован}
\end{aligned}
\right.
\]
\end{definition}

\begin{claim}
Пусть $V$ -- евклидово пространство размерности $n$ с фиксированной ориентацией, $(v_1,\ldots,v_n)\in V^n$ и $C\in \Matrix{n}$.
Тогда
\[
\Vol^{or}_n\Pi((v_1,\ldots,v_n)C) = \det C \Vol^{or}_n\Pi(v_1,\ldots,v_n)
\]
\end{claim}
\begin{proof}
Если матрица $C$ вырождена, то набор векторов $(v_1,\ldots,v_n)C$ линейно зависим, а значит левый объем равен  нулю.
С другой стороны $\det C = 0$, а значит и правая часть равна нулю.
Аналогично, если $(v_1,\ldots,v_n)$ линейно зависимый набор, то и набор $(v_1,\ldots,v_n)C$ тоже линейно зависим.
А тогда объемы в левой и правой частях равны нулю.
Осталось разобраться со случаем $(v_1,\ldots, v_n)$ -- базис и $C$ -- невырожденная матрица.
В этом случае совпадение левой и правой части по модулю следует из утверждения~\ref{claim::Volume} об изменении объемов при линейной замене образующих параллелепипеда.
Осталось проверить, что у них совпадают знаки.
Если $\det C > 0$, то наборы $(v_1,\ldots,v_n)$ и $(v_1,\ldots, v_n)C$ одинаково ориентированы.
Значит знаки у $\Vol^{or}_n\Pi((v_1,\ldots, v_n)C) $ и $\Vol^{or}_n \Pi((v_1,\ldots, v_n))$ одинаковые, а к тому же $\det C > 0$.
Значит знаки обеих частей равны.
Пусть теперь $\det C < 0$.
Тогда знаки у $\Vol^{or}_n\Pi((v_1,\ldots, v_n)C) $ и $\Vol^{or}_n \Pi((v_1,\ldots, v_n))$ разные.
Но при этом $\det C < 0$, что делает знаки левой и правой части одинаковыми.
Победа!
\end{proof}

\paragraph{Пример}

Пусть $V = \mathbb R^n$ со стандартным скалярным произведением $(x, y) = x^t y$.
И пусть ориентация зафиксирована так, что стандартный базис является положительным.
Возьмем $v_1,\ldots,v_n\in \mathbb R^n$ произвольный набор векторов.
Образуем матрицу $A = (v_1|\ldots|v_n)\in \Matrix{n}$.
Тогда 
\begin{itemize}
\item $G(v_1,\ldots,v_n) = A^t A$.

\item $\det G(v_1,\ldots,v_n) = \det A^2$.

\item $\Vol_n\Pi(v_1,\ldots,v_n) = |\det A|$.

\item $\Vol^{or}_n\Pi(v_1,\ldots,v_n) = \det A$.
\end{itemize}
Заметьте, что ориентация набора (как и знак соответствующего объема) меняется при перестановке векторов в наборе на знак совершенной перестановки.
Другая причина знака объема -- смена направления вектора, то есть когда вектор $v_i$ в наборе меняется на вектор $-v_i$.

\begin{claim}
Пусть $V$ -- ориентированное евклидово пространство размерности $n$, $v_1,\ldots,v_n\in V$ -- некоторый набор векторов и $\varphi\colon V\to V$ -- линейный оператор.
Тогда 
\[
\Vol^{or}_n(\varphi(\Pi(v_1,\ldots,v_n))) = \det \varphi \Vol^{or}_n(\Pi(v_1,\ldots,v_n))
\]
\end{claim}
\begin{proof}
Выберем положительно ориентированный ортонормированный базис $e_1,\ldots,e_n$.
Тогда $V$ превращается в $\mathbb R^n$, скалярное произведение становится стандартным $(x, y) = x^t y$, линейное отображение превращается в умножение на матрицу $\varphi(x) = Cx$, а векторы $v_1,\ldots,v_n$ расставим по столбцам матрицы $A = (v_1|\ldots|v_n)$.

Как мы видели в предыдущем примере
\[
\Vol^{or}_n(\Pi(v_1,\ldots,v_n)) = \det A,\quad
\Vol^{or}_n(\varphi(\Pi(v_1,\ldots,v_n))) = \det CA
\]
А так как $\det C = \det \varphi$ по определению, утверждение вытекает из мультипликативности определителя.
\end{proof}


\newpage
\section{Комплексные векторные пространства}

В начале несколько общих слов о том, зачем все это надо и куда оно нас заведет.
Этот раздел будет полностью посвящен векторным пространствам над полем $\mathbb C$.
Основная задача этого поля -- помогать решать трудности поля $\mathbb R$.
Но для этого нам надо уметь заменять вещественные объекты комплексными.
Например, вещественное векторное пространство превращать в комплексное векторное пространство.
Беда с евклидовыми пространствами в том, что среди билинейных форм над $\mathbb C$ нет аналога скалярного произведения.
Потому приходится от билинейных форм переходить к так называемым полуторалинейным.
Оказывается, что в этом случае можно построить полноценный аналог евклидовых пространств в комплексном мире, такие пространства называются эрмитовыми (но я иногда буду их называть евклидовыми над $\mathbb C$, чтобы подчеркнуть аналогию).
Кроме этого я покажу как переходить от вещественных объектов к комплексным и наоборот.
Мы будем менять пространства, операторы, билинейные формы и прочее.
Основная идея будет уследить за тем, как при подобных заменах меняются характеристики этих объектов.
Этому будет посвящен раздел комплексификации и овеществления.

\subsection{Полуторалинейные формы}

\begin{definition}
Пусть $V$ -- векторное пространство над полем $\mathbb C$, отображение $\beta \colon V\times V\to \mathbb C$ называется полуторалинейным, если выполнены следующие свойства:
\begin{enumerate}
\item $\beta(v_1+v_2, u) = \beta(v_1,u) + \beta(v_2, u)$, для любых $v_1, v_2, u\in V$

\item $\beta(\lambda v, u) = \bar \lambda \beta(v, u)$, для любых $v, u \in V$ и $\lambda\in \mathbb C$.

\item $\beta(v, u_1 + u_2) = \beta(v, u_1) + \beta(v, u_2)$, для любых $v, u_1, u_2\in V$.

\item $\beta(v, \lambda u) = \lambda\beta(v, u)$, для любых $v,u \in V$ и $\lambda \in \mathbb C$.
\end{enumerate}
\end{definition}

В этом случае говорят, что $\beta$ полулинейна по первому аргументу и линейна по второму.
Обратите внимание, что выбор первого аргумента для полулинейности является случайным и в разной литературе принято по-разному определять полуторалинейность.
В одних источниках полулинейность в первом аргументе, в других -- во втором.

\begin{definition}
Пусть $V$ -- векторное пространство над $\mathbb C$.
Множество всех полуторалинейных форм на пространстве $V$ я буду обозначать через $\Bil_{1\frac{1}{2}}(V)$.
Это множество является векторным пространством над $\mathbb C$ относительно операций:
\[
(\beta_1+\beta_2)(v, u) := \beta_1(v,u) + \beta_2(v,u)\quad\text{и}\quad (\lambda\beta)(v,u) = \lambda\beta(v,u),\quad\text{где}\quad v,u\in V,\;\lambda\in \mathbb C
\]
\end{definition}

Для удобства введем следующее техническое определение.
\begin{definition}
Пусть $A\in \operatorname{M}_{m\,n}(\mathbb C)$, тогда определим матрицу $\bar A$ как матрицу, в которой мы применили сопряжение ко всем элементам матрицы $A$, то есть $(\bar A)_{ij} = \overline{A_{ij}}$.
Кроме того, определим $A^* = \bar A^t$ и будем называть матрицу $A^*$ эрмитово сопряженной к матрице $A$.%
\footnote{Вообще говоря, правильный подход к определению матрицы $A^*$ следующий.
В случае вещественного поля надо положить $A^* = A^t$, а в случае комплексного $A^* = \bar A^t$.
Тогда у нас одно обозначение и надо лишь упоминать в каком мире мы живем -- комплексном или вещественном.
Я предпочитаю единую технику, которая работает в разных ситуациях, чем иметь миллион разных способов под каждую конкретную задачу.}
\end{definition}

Как обычно начинаем с конструкторов и правил, как работать с новыми объектами.

\begin{claim}
Пусть $V$ -- векторное пространство над $\mathbb C$ и $\beta \colon V\times V \to \mathbb C$ -- полуторалинейная форма.
Тогда
\begin{enumerate}
\item Для любого базиса $e_1,\ldots,e_n\in V$ определим матрицу $B$ с коэффициентами $\beta(e_i, e_j)$.
Тогда в координатах базиса $e_1,\ldots,e_n$ полуторалинейная форма записывается в виде $\beta(x, y) = \bar x^t B y$.

\item Если $(e'_1,\ldots,e'_n) = (e_1,\ldots,e_n)C$ -- некоторый другой базис, в котором матрица полуторалинейной формы есть $B'$ с коэффициентами $\beta(e'_i,e'_j)$, то $B' = \bar C^t B C = C^* B C$.

\item Для любого фиксированного базиса $e_1,\ldots,e_n\in V$ отображение $\Bil_{1\frac{1}{2}}(V)\to \operatorname{M}_n(\mathbb C)$ сопоставляющее полуторалинейной форме $\beta$ ее матрицу $B = (\beta(e_i,e_j))$ является изоморфизмом векторных пространств.
\end{enumerate}
\end{claim}
\begin{proof}
(1) Если $e = (e_1,\ldots,e_n)$ -- базис в $V$ и $v,u\in V$, тогда $v = ex$ и $u = ey$, где $x,y\in \mathbb C^n$.
В этом случае
\[
\beta(v,u) = \beta(\sum_i x_i e_i, \sum_j y_j e_j) = \sum_{ij}\bar x_i y_j \beta(e_i, e_j) = \bar x^t B y
\]

(2) Пусть в этом случае $v,u\in V$ и при этом $v = ex = e' x'$ и $u = ey = e'y'$, где $x,y,x',y'\in \mathbb C^n$.
Тогда по формулам замены координат вектора (формулы в конце раздела~\ref{subsection::FnSpace}) $x = Cx'$ и $y = Cy'$.
Тогда
\[
\beta(v,u) =\beta(x,y) = \bar x^t B y = \overline{Cx'}^t B C y' = (\bar x')^t \bar C^t B C y' = \beta(x',y') = (\bar x')^t B' y'
\]
Значит $B' = \bar C^t B C = C^* B C$.

(3) Линейность правила $\beta\mapsto B = (\beta(e_i, e_j))$ проверяется непосредственно.
Также непосредственно проверяется, что правило $B\mapsto \beta(x,y) = \bar x^t B y$ линейно и задает обратное отображение.
Я оставлю эту рутину на радость читателю.
\end{proof}

